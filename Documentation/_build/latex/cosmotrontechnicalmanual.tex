%% Generated by Sphinx.
\def\sphinxdocclass{report}
\documentclass[letterpaper,10pt,english]{sphinxmanual}
\ifdefined\pdfpxdimen
   \let\sphinxpxdimen\pdfpxdimen\else\newdimen\sphinxpxdimen
\fi \sphinxpxdimen=.75bp\relax

\PassOptionsToPackage{warn}{textcomp}
\usepackage[utf8]{inputenc}
\ifdefined\DeclareUnicodeCharacter
% support both utf8 and utf8x syntaxes
  \ifdefined\DeclareUnicodeCharacterAsOptional
    \def\sphinxDUC#1{\DeclareUnicodeCharacter{"#1}}
  \else
    \let\sphinxDUC\DeclareUnicodeCharacter
  \fi
  \sphinxDUC{00A0}{\nobreakspace}
  \sphinxDUC{2500}{\sphinxunichar{2500}}
  \sphinxDUC{2502}{\sphinxunichar{2502}}
  \sphinxDUC{2514}{\sphinxunichar{2514}}
  \sphinxDUC{251C}{\sphinxunichar{251C}}
  \sphinxDUC{2572}{\textbackslash}
\fi
\usepackage{cmap}
\usepackage[T1]{fontenc}
\usepackage{amsmath,amssymb,amstext}
\usepackage{babel}



\usepackage{times}
\expandafter\ifx\csname T@LGR\endcsname\relax
\else
% LGR was declared as font encoding
  \substitutefont{LGR}{\rmdefault}{cmr}
  \substitutefont{LGR}{\sfdefault}{cmss}
  \substitutefont{LGR}{\ttdefault}{cmtt}
\fi
\expandafter\ifx\csname T@X2\endcsname\relax
  \expandafter\ifx\csname T@T2A\endcsname\relax
  \else
  % T2A was declared as font encoding
    \substitutefont{T2A}{\rmdefault}{cmr}
    \substitutefont{T2A}{\sfdefault}{cmss}
    \substitutefont{T2A}{\ttdefault}{cmtt}
  \fi
\else
% X2 was declared as font encoding
  \substitutefont{X2}{\rmdefault}{cmr}
  \substitutefont{X2}{\sfdefault}{cmss}
  \substitutefont{X2}{\ttdefault}{cmtt}
\fi


\usepackage[Bjarne]{fncychap}
\usepackage{sphinx}

\fvset{fontsize=\small}
\usepackage{geometry}


% Include hyperref last.
\usepackage{hyperref}
% Fix anchor placement for figures with captions.
\usepackage{hypcap}% it must be loaded after hyperref.
% Set up styles of URL: it should be placed after hyperref.
\urlstyle{same}

\addto\captionsenglish{\renewcommand{\contentsname}{Contents:}}

\usepackage{sphinxmessages}
\setcounter{tocdepth}{2}



\title{Cosmotron Technical Manual}
\date{Apr 14, 2021}
\release{2021}
\author{Parker King}
\newcommand{\sphinxlogo}{\vbox{}}
\renewcommand{\releasename}{Release}
\makeindex
\begin{document}

\pagestyle{empty}
\sphinxmaketitle
\pagestyle{plain}
\sphinxtableofcontents
\pagestyle{normal}
\phantomsection\label{\detokenize{index::doc}}



\chapter{Introduction}
\label{\detokenize{intro:introduction}}\label{\detokenize{intro::doc}}
\sphinxAtStartPar
This document is meant to be a technical guide for accessing,
modifying, and expanding Cosmotron’s software. This is not meant
to be API documentation, but rather a guide for modifying the
software on Cosmotron to adjust how the motors move, change the
controller scheme, and possibly adjust the code to match
modifications made to the physical movement mechanisms.

\sphinxAtStartPar
The following is a brief summarry of each section.

\sphinxAtStartPar
\sphinxstylestrong{Technical Manual}
\begin{quote}

\sphinxAtStartPar
\sphinxstylestrong{Accessing the Raspberry Pi}
\begin{quote}

\sphinxAtStartPar
How to access the files on the Raspberry Pi so they can be changed.
\end{quote}

\sphinxAtStartPar
\sphinxstylestrong{Changing Movement Range}
\begin{quote}

\sphinxAtStartPar
How to change the ranges the motors on Cosmotron move.
\end{quote}

\sphinxAtStartPar
\sphinxstylestrong{Changing Movement Map}
\begin{quote}

\sphinxAtStartPar
How to change the mapping of inputs to outputs.
\end{quote}
\end{quote}

\sphinxAtStartPar
\sphinxstylestrong{Input Event Handling}
\begin{quote}

\sphinxAtStartPar
Brief documentation of the code that reads input events from the PS4 controllers.
\end{quote}

\sphinxAtStartPar
\sphinxstylestrong{Movement Mapping}
\begin{quote}

\sphinxAtStartPar
Brief documentation of the code that maps the PS4 inputs to the Maestro outputs.
\end{quote}

\sphinxAtStartPar
\sphinxstylestrong{Output Objects}
\begin{quote}

\sphinxAtStartPar
Brief documentation for each of the output objects used to calculate the Maestro outputs.
\end{quote}


\chapter{Technical Manual}
\label{\detokenize{technical:technical-manual}}\label{\detokenize{technical::doc}}

\section{Accessing the Raspberry Pi}
\label{\detokenize{access:accessing-the-raspberry-pi}}\label{\detokenize{access::doc}}

\subsection{Using SSH}
\label{\detokenize{access:using-ssh}}
\sphinxAtStartPar
Since the Raspberry Pi computer is used headless (without a monitor) to
control the animatronic cougar, one of the easier ways to access the files
on the Pi is to use SSH. There are various client programs that can be
used to SSH. Using the SSH extension on VS Code is a good option for this
because it provides the ability to edit files within VS Code instead of
having to use a terminal text editor such as VIM or nano.

\sphinxAtStartPar
Using the installed SSH client, the connection can be made with the
following information: domain, username, and password. How exactly this
information is used depends on the SSH client.

\sphinxAtStartPar
The domain is the IP address of the Pi on the network it is on. Most likely,
this will be the IP address of the Pi on BYU’s public WiFi. This address
can be found by pressing on the up arrow of the Pi\sphinxhyphen{}Top case until the WiFi
section is reached. This option will display the Pi’s current IP address,
which takes the form of four numbers separated by periods. Such as:
“10.37.49.83”.

\sphinxAtStartPar
The default username on all Raspberry Pis is “pi”. This has not been changed
on the Pi used for Cosmotron.

\sphinxAtStartPar
The password for the Raspberry Pi being used for Cosmotron is currently
“pi\sphinxhyphen{}top”.

\sphinxAtStartPar
Unless the SSH client provides specific fields to enter the needed information
into, the command to make an SSH connection takes the following form:

\sphinxAtStartPar
\sphinxcode{\sphinxupquote{ssh username@ip\_address}}

\sphinxAtStartPar
Here is an example of what this command could look like for accessing
Cosmotron’s Raspberry Pi:

\sphinxAtStartPar
\sphinxcode{\sphinxupquote{ssh pi@10.37.49.83}}

\sphinxAtStartPar
The user will be prompted to enter the password following a successful connection.


\subsection{Using a Monitor}
\label{\detokenize{access:using-a-monitor}}
\sphinxAtStartPar
An alternative method that is more user friendly, but requires a bit more hardware
than what is provided (and assumed to be owned by the user), is to connect the
Raspberry Pi to monitor and use it as a normal computer. This requires an HDMI
compatible monitor, an HDMI to Micro HDMI to connect the Pi to the monitor, a mouse,
and a keyboard.

\sphinxAtStartPar
Ensure the Pi is connected to the monitor before powering it on or else it won’t
start up the user interface to display on the monitor. The keybaord and mouse can be
plugged in either before or after the Pi is powered on.

\sphinxAtStartPar
After the Pi has been turned on, files can be editted using the local programs such
as the programming IDE’s (Geany and Thonny), the general text editor (Mousepad),
and command line text editors (nano).


\section{Changing Movement Range}
\label{\detokenize{range:changing-movement-range}}\label{\detokenize{range::doc}}
\sphinxAtStartPar
There are two loactions that the range of the outputs can be adjusted. The first is
in Python code and the second is in the settings of the Pololu Maestro Board. These
two sources do not change the range in the same way. The Python code determines the
output value used in order to move the motors. The Maestro board provides a limit to
the allowable range for each output channel. If the Maestro board receives an output
that is outside its defined range, it will limit the value to the edge of its range.
So the settings on the Maestro functions more like a safety in how the Maestro board
is being utilized for Cosmotron and the Python code defines the range which impacts
how the input values from the PlayStation 4 (PS4) Dual Shock controllers are mapped
to the output values. How to change both are outlined in the following sections.


\subsection{Defining Output Ranges in Python Code}
\label{\detokenize{range:defining-output-ranges-in-python-code}}\label{\detokenize{range:pyrange}}
\sphinxAtStartPar
The class, MovementMap, contained in the file MovementMap.py, is were the ranges for
each output is defined. Thes values are defined in the beginning of the class’
constructor. Output objects are created for each movement group. After each object is
created, it has it’s minimum, default, and maximum output values defined by using the
set\_outputs method of the output object. The object is then added to a list used for
a sequential startup of the Maestro channels, but this is irrelavant to the output
ranges. The following code block is an excerpty from MovementMap.py:

\begin{sphinxVerbatim}[commandchars=\\\{\},numbers=left,firstnumber=78,stepnumber=1]
\PYG{c+c1}{\PYGZsh{} Left Ear}
\PYG{n+nb+bp}{self}\PYG{o}{.}\PYG{n}{left\PYGZus{}ear} \PYG{o}{=} \PYG{n}{EarOutput}\PYG{p}{(}\PYG{l+s+s2}{\PYGZdq{}}\PYG{l+s+s2}{left ear}\PYG{l+s+s2}{\PYGZdq{}}\PYG{p}{,} \PYG{l+m+mi}{1}\PYG{p}{,} \PYG{p}{[}\PYG{l+m+mi}{1}\PYG{p}{]}\PYG{p}{,} \PYG{p}{[}\PYG{l+s+s2}{\PYGZdq{}}\PYG{l+s+s2}{down\PYGZus{}button\PYGZus{}1}\PYG{l+s+s2}{\PYGZdq{}}\PYG{p}{,} \PYG{l+s+s2}{\PYGZdq{}}\PYG{l+s+s2}{left\PYGZus{}button\PYGZus{}1}\PYG{l+s+s2}{\PYGZdq{}}\PYG{p}{]}\PYG{p}{)}
\PYG{n+nb+bp}{self}\PYG{o}{.}\PYG{n}{left\PYGZus{}ear}\PYG{o}{.}\PYG{n}{set\PYGZus{}outputs}\PYG{p}{(}\PYG{p}{[}\PYG{l+m+mi}{4000}\PYG{p}{]}\PYG{p}{,} \PYG{p}{[}\PYG{l+m+mi}{4000}\PYG{p}{]}\PYG{p}{,} \PYG{p}{[}\PYG{l+m+mi}{8000}\PYG{p}{]}\PYG{p}{)}
\PYG{n+nb+bp}{self}\PYG{o}{.}\PYG{n}{output\PYGZus{}objects}\PYG{o}{.}\PYG{n}{append}\PYG{p}{(}\PYG{n+nb+bp}{self}\PYG{o}{.}\PYG{n}{left\PYGZus{}ear}\PYG{p}{)}

\PYG{c+c1}{\PYGZsh{} Eyelids}
\PYG{n+nb+bp}{self}\PYG{o}{.}\PYG{n}{eyelids} \PYG{o}{=} \PYG{n}{AnalogOutputObject}\PYG{p}{(}\PYG{l+s+s2}{\PYGZdq{}}\PYG{l+s+s2}{eylids}\PYG{l+s+s2}{\PYGZdq{}}\PYG{p}{,} \PYG{l+m+mi}{2}\PYG{p}{,} \PYG{p}{[}\PYG{l+m+mi}{2}\PYG{p}{,} \PYG{l+m+mi}{3}\PYG{p}{]}\PYG{p}{)}
\PYG{n+nb+bp}{self}\PYG{o}{.}\PYG{n}{eyelids}\PYG{o}{.}\PYG{n}{set\PYGZus{}outputs}\PYG{p}{(}\PYG{p}{[}\PYG{l+m+mi}{6800}\PYG{p}{,} \PYG{l+m+mi}{5300}\PYG{p}{]}\PYG{p}{,} \PYG{p}{[}\PYG{l+m+mi}{6800}\PYG{p}{,} \PYG{l+m+mi}{5300}\PYG{p}{]}\PYG{p}{,} \PYG{p}{[}\PYG{l+m+mi}{6050}\PYG{p}{,} \PYG{l+m+mi}{6550}\PYG{p}{]}\PYG{p}{)}
\PYG{n+nb+bp}{self}\PYG{o}{.}\PYG{n}{output\PYGZus{}objects}\PYG{o}{.}\PYG{n}{append}\PYG{p}{(}\PYG{n+nb+bp}{self}\PYG{o}{.}\PYG{n}{eyelids}\PYG{p}{)}
\end{sphinxVerbatim}

\sphinxAtStartPar
This format is used for each movement group contained in the MovementMap class. To
accomodate the varying circumstances of each movement group and how many motors it
will output to, the inputs to the set\_outputs method takes three lists. The first
list contains the minimum output values. The second list contains the default output
values which are used on startup to match the motors with the default inputs from
PS4 controllers and for some complicated movement groups. The third list contains
the maximum output values.

\sphinxAtStartPar
The code above provides two examples of setting the range. One with only one output
channel and one with two output channels. The number of output channels, as well as
which channels to output to, are determined by the inputs passed to the constructor
of the output object. The exact inputs differ depending on the output object, but
each output object constructor takes the same first three inputs. These are, in
order, movement group name, number of output channels, and which channels to output
to.

\begin{sphinxadmonition}{note}{Note:}
\sphinxAtStartPar
Order of the elements in the list matters.
\end{sphinxadmonition}

\sphinxAtStartPar
The number of elements in each list passed to the set\_outputs method needs to match
the number output channels for that object. So for the left ear, which only has one
output, each list has one value which correspond directly to the max, default, and min
output values. The eyelids have two outputs, so each list in the set\_outputs method has
two values. For these situations with more than one output for the movement group, the
order of the values in each list matters. The first value in the min, default, and max
lists corresponds to the first value in the list defining the output channels. The
second value in the min, default, and max lists corresponds to the second value in the
list defining the output channels and so on.

\sphinxAtStartPar
This means that for the left ear, it only sends outputs to channel one on the Maestro
board. The minimum value that is output is 4000 and the maximum is 8000. By default,
the value 4000 is sent to this channel.

\sphinxAtStartPar
Looking at the eyelids, this movement group outputs to channels 2 and 3 on the
Maestro. For channel 2, the minimum value is 6800 and the max is 6050. For channel
3, the minimum value is 5300 and the max is 6550. The defaults for channels 2 and 3
are 6800 and 5300 respectively.

\begin{sphinxVerbatim}[commandchars=\\\{\},numbers=left,firstnumber=117,stepnumber=1]
\PYG{c+c1}{\PYGZsh{} Right Lip}
\PYG{n+nb+bp}{self}\PYG{o}{.}\PYG{n}{right\PYGZus{}lip} \PYG{o}{=} \PYG{n}{SideLipOutput}\PYG{p}{(}\PYG{l+s+s2}{\PYGZdq{}}\PYG{l+s+s2}{right lip}\PYG{l+s+s2}{\PYGZdq{}}\PYG{p}{,} \PYG{l+m+mi}{2}\PYG{p}{,} \PYG{p}{[}\PYG{l+m+mi}{14}\PYG{p}{,} \PYG{l+m+mi}{15}\PYG{p}{]}\PYG{p}{,} \PYG{p}{[}\PYG{l+s+s2}{\PYGZdq{}}\PYG{l+s+s2}{circle\PYGZus{}button\PYGZus{}2}\PYG{l+s+s2}{\PYGZdq{}}\PYG{p}{,} \PYG{l+s+s2}{\PYGZdq{}}\PYG{l+s+s2}{x\PYGZus{}button\PYGZus{}2}\PYG{l+s+s2}{\PYGZdq{}}\PYG{p}{]}\PYG{p}{)}
\PYG{n+nb+bp}{self}\PYG{o}{.}\PYG{n}{right\PYGZus{}lip}\PYG{o}{.}\PYG{n}{set\PYGZus{}outputs}\PYG{p}{(}\PYG{p}{[}\PYG{l+m+mi}{6500}\PYG{p}{,} \PYG{l+m+mi}{5000}\PYG{p}{]}\PYG{p}{,} \PYG{p}{[}\PYG{l+m+mi}{6000}\PYG{p}{,} \PYG{l+m+mi}{6000}\PYG{p}{]}\PYG{p}{,} \PYG{p}{[}\PYG{l+m+mi}{5500}\PYG{p}{,} \PYG{l+m+mi}{7000}\PYG{p}{]}\PYG{p}{)}
\PYG{n+nb+bp}{self}\PYG{o}{.}\PYG{n}{output\PYGZus{}objects}\PYG{o}{.}\PYG{n}{append}\PYG{p}{(}\PYG{n+nb+bp}{self}\PYG{o}{.}\PYG{n}{right\PYGZus{}lip}\PYG{p}{)}
\end{sphinxVerbatim}

\sphinxAtStartPar
Here is another example to illustrate that sometimes the minimum value is set higher
than the maximum value. Doing this “inverts” the output. More specifially, a minimum
input value maps to the value set as the minimum for the output value. The same goes
for the maximums. Maximum input value maps to the value set for the maximum output
value. So the right lip outputs to channels 14 and 15 (set by the list {[}14, 15{]}
passed to the constructor). Channel 14 has its minimum output set to 6050 and its
maximum output set to 5500. Channel 15 has its minimum output set to 5000 and its
maximum output set to 7000. The defaults for channels 14 and 15 are both 6000.


\subsection{Defining Output Limits in Maestro Settings}
\label{\detokenize{range:defining-output-limits-in-maestro-settings}}
\sphinxAtStartPar
\sphinxstylestrong{Maestro Servo Controller}

\sphinxAtStartPar
There are two main ways to change the range limits for the Maestro board. One is
to use the Maestro Control Center software. The other is to use UscCmd command
line software that comes with the linux\sphinxhyphen{}maestro files. Ultimately, both depend on
loading or modifying a settings file. The control center software is a GUI that
lets you change the settings interactively and thus is more user friendly.

\sphinxAtStartPar
As mentioned, the most user friendly option to change the channel range limits is
through the control center software. This can be download for the needed operating
system from the \sphinxhref{https://www.pololu.com/docs/0J40/all\#3}{Maestro user manual on Pololu’s website}. Any questions on how to use the
control center properly that is not answered by this document for Cosmotron can be
found in the user manual for the Maestro board from Pololu.

\begin{figure}[htbp]
\centering
\capstart

\noindent\sphinxincludegraphics{{MaestroControlCenter}.png}
\caption{The status tab of the Maestro Control Center GUI}\label{\detokenize{range:id1}}\end{figure}

\sphinxAtStartPar
The two main tabs on the control center that are important for configuring settings
are the following:
\begin{itemize}
\item {} 
\sphinxAtStartPar
Channel Settings

\item {} 
\sphinxAtStartPar
Serial Settings

\end{itemize}

\sphinxAtStartPar
In the \sphinxstyleemphasis{Channel Settings} tab you are able to set the min and max allowable pulse
range, in microseconds, of each output channel. In the \sphinxstyleemphasis{Serial Settings} tab, the
serial mode must be set to \sphinxstyleemphasis{USB Dual Port} in order for the commands to be properly
sent from the Raspberry Pi.

\sphinxAtStartPar
The changes to the settings are applied to the Maestro board by clicking on the
\sphinxstyleemphasis{Apply Settings} button on the bottom right part of the control center window.
These settings persist on the board until they are changed again in the control
center or by loading a settings file. The settings file is generated by selecting
the \sphinxstyleemphasis{Save settings file…} option in the File drop down menu at the top of the
control center. To load a settings file, use the \sphinxstyleemphasis{Open settings file…} option
from the same menu. It is recommended to keep a file of the desired settings on
the Maestro used for Cosmotron in case the Maestro board needs to be changed.

\sphinxAtStartPar
The settings file is a simple text file that can easily be edited using any text
editor. It uses tags like HTML to specify the different settings and their values.

\sphinxAtStartPar
\sphinxstylestrong{Tic Stepper Driver Boards}

\sphinxAtStartPar
It is important to mention the Tic Stepper Motor Driver boards in this section
since they also have settings that impact the motor movement of the few stepper
motors being used. The Tic boards also use settings files that can be saved and
loaded like the Maestro boards do. There is also control center software that is
used to set the settings on the Tic board. The software can be found for various
operating systems in the \sphinxhref{https://www.pololu.com/docs/0J71/all\#3}{Tic user manual on Pololu’s website}.

\sphinxAtStartPar
Currently, the Tic boards have their \sphinxstyleemphasis{Control Mode} set to \sphinxstyleemphasis{RC Position Control}
so the steppers are moved to specific positions based off of the output the Tic
board receives from the Maestro board. The settings for this mode are set in the
\sphinxstyleemphasis{RC and analog scaling} section of the \sphinxstyleemphasis{Input and motor settings} tab. Using the
learn button, the min, neutral, and max values can be set for what the Maestro
board is set to output. However, the important setting that really determines the
range of outputs for the steppers is the target values for the min and max inputs.
These targets correlate to the position of the motor. If the target for minimum
input is set to \sphinxhyphen{}200 and the target for maximum input is set to 200, the motor will
travel 400 steps to go between the minimum and maximum positions.

\begin{figure}[htbp]
\centering
\capstart

\noindent\sphinxincludegraphics{{TicControlCenter}.png}
\caption{The Input and motor settings tab of the Tic Control Center}\label{\detokenize{range:id2}}\end{figure}

\begin{sphinxadmonition}{note}{Note:}
\sphinxAtStartPar
To connect Cosmotron’s Maestro to a computer, the USB port in Cosmotron’s
access panel can be used. However, to connect to the Tic boards in order to
change their settings, the boards must currently be removed from inside
Cosmotron.

\sphinxAtStartPar
The Maestro has a USB\sphinxhyphen{}B Mini port, but the access panel it is connected to
has a USB\sphinxhyphen{}A port. The Tic boards have a USB\sphinxhyphen{}Micro port. Before trying to change
settings on either board make sure you have the proper USB cable.
\end{sphinxadmonition}


\section{Changing the Movement Map}
\label{\detokenize{mapchange:changing-the-movement-map}}\label{\detokenize{mapchange::doc}}
\sphinxAtStartPar
The movement mapping code is constructed in a way to be easily adaptable.
For example, if the number of servos used in a movement group change, this
can be adjusted using the concepts discussed in {\hyperref[\detokenize{range:pyrange}]{\sphinxcrossref{\DUrole{std,std-ref}{Defining Output Ranges in Python Code}}}}. The two
main areas of interest that could be changed are if a movement group is
added, changed, or removed, and the way inputs are mapped to outputs need
to be changed.

\begin{sphinxadmonition}{note}{Note:}
\sphinxAtStartPar
This section assumes familiarity with Python classes and objects.
\end{sphinxadmonition}


\subsection{Movement Groups}
\label{\detokenize{mapchange:movement-groups}}
\sphinxAtStartPar
The output objects are designed to easily have the following attributes
changed:
\begin{enumerate}
\sphinxsetlistlabels{\arabic}{enumi}{enumii}{}{.}%
\item {} 
\sphinxAtStartPar
Number of output channels

\item {} 
\sphinxAtStartPar
Which output channels to use

\item {} 
\sphinxAtStartPar
Min, default, and max output values

\item {} 
\sphinxAtStartPar
Correlated PS4 input(s)

\end{enumerate}

\sphinxAtStartPar
\sphinxstylestrong{Attributes 1 and 2}

\sphinxAtStartPar
These attributes are always set using the constructor of the object.
The following example has the “jaw” movement sending outputs on two
channels. These channels are 18 and 19.

\begin{sphinxVerbatim}[commandchars=\\\{\}]
\PYG{n+nb+bp}{self}\PYG{o}{.}\PYG{n}{jaw} \PYG{o}{=} \PYG{n}{AnalogOutputObject}\PYG{p}{(}\PYG{l+s+s2}{\PYGZdq{}}\PYG{l+s+s2}{jaw}\PYG{l+s+s2}{\PYGZdq{}}\PYG{p}{,} \PYG{l+m+mi}{2}\PYG{p}{,} \PYG{p}{[}\PYG{l+m+mi}{18}\PYG{p}{,} \PYG{l+m+mi}{19}\PYG{p}{]}\PYG{p}{)}
\end{sphinxVerbatim}

\sphinxAtStartPar
\sphinxstylestrong{Attribute 3}

\sphinxAtStartPar
This attribute is always dependant on the \sphinxstyleemphasis{set\_outputs} method that each
output object class inherits. This was already discussed in the section,
{\hyperref[\detokenize{range:pyrange}]{\sphinxcrossref{\DUrole{std,std-ref}{Defining Output Ranges in Python Code}}}}. However, here is an example of the method being used.

\begin{sphinxVerbatim}[commandchars=\\\{\}]
\PYG{n+nb+bp}{self}\PYG{o}{.}\PYG{n}{jaw}\PYG{o}{.}\PYG{n}{set\PYGZus{}outputs}\PYG{p}{(}\PYG{p}{[}\PYG{l+m+mi}{6000}\PYG{p}{,} \PYG{l+m+mi}{6000}\PYG{p}{]}\PYG{p}{,} \PYG{p}{[}\PYG{l+m+mi}{6000}\PYG{p}{,} \PYG{l+m+mi}{6000}\PYG{p}{]}\PYG{p}{,} \PYG{p}{[}\PYG{l+m+mi}{7000}\PYG{p}{,} \PYG{l+m+mi}{5000}\PYG{p}{]}\PYG{p}{)}
\end{sphinxVerbatim}

\sphinxAtStartPar
\sphinxstylestrong{Attribute 4}

\sphinxAtStartPar
This attribute is mostly determined by an attribute of the MovementMap
class called \sphinxstyleemphasis{input\_map}. This MovementMap attribute is a python
dictionary that links input names from the two PS4 controllers to the
movement groups they control. This enables the MovementMap class to pass
the inputs to the correct output objects quickly based on the input name.

\sphinxAtStartPar
The following is the mapping for the first controller:

\begin{sphinxVerbatim}[commandchars=\\\{\}]
\PYG{l+s+s2}{\PYGZdq{}}\PYG{l+s+s2}{x\PYGZus{}button\PYGZus{}1}\PYG{l+s+s2}{\PYGZdq{}}\PYG{p}{:} \PYG{n+nb+bp}{self}\PYG{o}{.}\PYG{n}{right\PYGZus{}ear}\PYG{p}{,}
\PYG{l+s+s2}{\PYGZdq{}}\PYG{l+s+s2}{circle\PYGZus{}button\PYGZus{}1}\PYG{l+s+s2}{\PYGZdq{}}\PYG{p}{:} \PYG{n+nb+bp}{self}\PYG{o}{.}\PYG{n}{right\PYGZus{}ear}\PYG{p}{,}
\PYG{l+s+s2}{\PYGZdq{}}\PYG{l+s+s2}{triangle\PYGZus{}button\PYGZus{}1}\PYG{l+s+s2}{\PYGZdq{}}\PYG{p}{:} \PYG{n+nb+bp}{None}\PYG{p}{,}
\PYG{l+s+s2}{\PYGZdq{}}\PYG{l+s+s2}{square\PYGZus{}button\PYGZus{}1}\PYG{l+s+s2}{\PYGZdq{}}\PYG{p}{:} \PYG{n+nb+bp}{None}\PYG{p}{,}
\PYG{l+s+s2}{\PYGZdq{}}\PYG{l+s+s2}{down\PYGZus{}button\PYGZus{}1}\PYG{l+s+s2}{\PYGZdq{}}\PYG{p}{:} \PYG{n+nb+bp}{self}\PYG{o}{.}\PYG{n}{left\PYGZus{}ear}\PYG{p}{,}
\PYG{l+s+s2}{\PYGZdq{}}\PYG{l+s+s2}{up\PYGZus{}button\PYGZus{}1}\PYG{l+s+s2}{\PYGZdq{}}\PYG{p}{:} \PYG{n+nb+bp}{None}\PYG{p}{,}
\PYG{l+s+s2}{\PYGZdq{}}\PYG{l+s+s2}{right\PYGZus{}button\PYGZus{}1}\PYG{l+s+s2}{\PYGZdq{}}\PYG{p}{:} \PYG{n+nb+bp}{None}\PYG{p}{,}
\PYG{l+s+s2}{\PYGZdq{}}\PYG{l+s+s2}{left\PYGZus{}button\PYGZus{}1}\PYG{l+s+s2}{\PYGZdq{}}\PYG{p}{:} \PYG{n+nb+bp}{self}\PYG{o}{.}\PYG{n}{left\PYGZus{}ear}\PYG{p}{,}
\PYG{l+s+s2}{\PYGZdq{}}\PYG{l+s+s2}{r1\PYGZus{}button\PYGZus{}1}\PYG{l+s+s2}{\PYGZdq{}}\PYG{p}{:} \PYG{n+nb+bp}{self}\PYG{o}{.}\PYG{n}{nose}\PYG{p}{,}
\PYG{l+s+s2}{\PYGZdq{}}\PYG{l+s+s2}{l1\PYGZus{}button\PYGZus{}1}\PYG{l+s+s2}{\PYGZdq{}}\PYG{p}{:} \PYG{n+nb+bp}{None}\PYG{p}{,}
\PYG{l+s+s2}{\PYGZdq{}}\PYG{l+s+s2}{r2\PYGZus{}analog\PYGZus{}1}\PYG{l+s+s2}{\PYGZdq{}}\PYG{p}{:} \PYG{n+nb+bp}{None}\PYG{p}{,}
\PYG{l+s+s2}{\PYGZdq{}}\PYG{l+s+s2}{l2\PYGZus{}analog\PYGZus{}1}\PYG{l+s+s2}{\PYGZdq{}}\PYG{p}{:} \PYG{n+nb+bp}{self}\PYG{o}{.}\PYG{n}{eyelids}\PYG{p}{,}
\PYG{l+s+s2}{\PYGZdq{}}\PYG{l+s+s2}{r3\PYGZus{}button\PYGZus{}1}\PYG{l+s+s2}{\PYGZdq{}}\PYG{p}{:} \PYG{n+nb+bp}{None}\PYG{p}{,}
\PYG{l+s+s2}{\PYGZdq{}}\PYG{l+s+s2}{l3\PYGZus{}button\PYGZus{}1}\PYG{l+s+s2}{\PYGZdq{}}\PYG{p}{:} \PYG{n+nb+bp}{None}\PYG{p}{,}
\PYG{l+s+s2}{\PYGZdq{}}\PYG{l+s+s2}{r\PYGZus{}joystick\PYGZus{}x\PYGZus{}analog\PYGZus{}1}\PYG{l+s+s2}{\PYGZdq{}}\PYG{p}{:} \PYG{n+nb+bp}{self}\PYG{o}{.}\PYG{n}{eyebrows}\PYG{p}{,}
\PYG{l+s+s2}{\PYGZdq{}}\PYG{l+s+s2}{r\PYGZus{}joystick\PYGZus{}y\PYGZus{}analog\PYGZus{}1}\PYG{l+s+s2}{\PYGZdq{}}\PYG{p}{:} \PYG{n+nb+bp}{self}\PYG{o}{.}\PYG{n}{eyebrows}\PYG{p}{,}
\PYG{l+s+s2}{\PYGZdq{}}\PYG{l+s+s2}{l\PYGZus{}joystick\PYGZus{}x\PYGZus{}analog\PYGZus{}1}\PYG{l+s+s2}{\PYGZdq{}}\PYG{p}{:} \PYG{n+nb+bp}{self}\PYG{o}{.}\PYG{n}{eyes\PYGZus{}horizontal}\PYG{p}{,}
\PYG{l+s+s2}{\PYGZdq{}}\PYG{l+s+s2}{l\PYGZus{}joystick\PYGZus{}y\PYGZus{}analog\PYGZus{}1}\PYG{l+s+s2}{\PYGZdq{}}\PYG{p}{:} \PYG{n+nb+bp}{self}\PYG{o}{.}\PYG{n}{eyes\PYGZus{}vertical}\PYG{p}{,}
\PYG{l+s+s2}{\PYGZdq{}}\PYG{l+s+s2}{share\PYGZus{}button\PYGZus{}1}\PYG{l+s+s2}{\PYGZdq{}}\PYG{p}{:} \PYG{n+nb+bp}{None}\PYG{p}{,}
\PYG{l+s+s2}{\PYGZdq{}}\PYG{l+s+s2}{options\PYGZus{}button\PYGZus{}1}\PYG{l+s+s2}{\PYGZdq{}}\PYG{p}{:} \PYG{n+nb+bp}{None}\PYG{p}{,}
\PYG{l+s+s2}{\PYGZdq{}}\PYG{l+s+s2}{ps\PYGZus{}symbol\PYGZus{}button\PYGZus{}1}\PYG{l+s+s2}{\PYGZdq{}}\PYG{p}{:} \PYG{n+nb+bp}{None}
\end{sphinxVerbatim}

\sphinxAtStartPar
For the dictionary keys that have a value of \sphinxstyleemphasis{None}, those inputs
don’t control anything on Cosmotron and will be ignored by the
MovementMap class. To make a new mapping between a PS4 input and an
movement group, replace \sphinxstyleemphasis{None} with the corresponding output object.
To remove a mapping, change the value for the input key to \sphinxstyleemphasis{None}.
To change a currently used mapping, change the output object in the
value to the output object for the desired movement group.

\sphinxAtStartPar
There is a secondary place where attribute 4 is set. This is in the
constructor for some of the more complex output objects, like the
eyebrows. These output object determine their output based off of
two PS4 inputs. These output objects take an additional parameter in
their constructor. This parameter is a list of the input names. The
order of this list matters. To see how to order the list, look at
the code for the corresponding output object.

\begin{sphinxadmonition}{note}{Note:}
\sphinxAtStartPar
These complex objects that use multiple PS4 inputs need to
have the input names passed into their constructor be the same
as the key\sphinxhyphen{}value pairs set be the \sphinxstyleemphasis{input\_map} attribute.
\end{sphinxadmonition}

\sphinxAtStartPar
Here is the constructor being called for the eyebrow output object
as an example:

\begin{sphinxVerbatim}[commandchars=\\\{\}]
\PYG{n+nb+bp}{self}\PYG{o}{.}\PYG{n}{eyebrows} \PYG{o}{=} \PYG{n}{EyebrowsOutput}\PYG{p}{(}\PYG{l+s+s2}{\PYGZdq{}}\PYG{l+s+s2}{eyebrows}\PYG{l+s+s2}{\PYGZdq{}}\PYG{p}{,} \PYG{l+m+mi}{4}\PYG{p}{,} \PYG{p}{[}\PYG{l+m+mi}{8}\PYG{p}{,} \PYG{l+m+mi}{9}\PYG{p}{,} \PYG{l+m+mi}{10}\PYG{p}{,} \PYG{l+m+mi}{11}\PYG{p}{]}\PYG{p}{,} \PYG{p}{[}\PYG{l+s+s2}{\PYGZdq{}}\PYG{l+s+s2}{r\PYGZus{}joystick\PYGZus{}y\PYGZus{}analog\PYGZus{}1}\PYG{l+s+s2}{\PYGZdq{}}\PYG{p}{,} \PYG{l+s+s2}{\PYGZdq{}}\PYG{l+s+s2}{r\PYGZus{}joystick\PYGZus{}x\PYGZus{}analog\PYGZus{}1}\PYG{l+s+s2}{\PYGZdq{}}\PYG{p}{]}\PYG{p}{)}
\end{sphinxVerbatim}

\sphinxAtStartPar
Notice how in the section of the \sphinxstyleemphasis{input\_map} attribute shown above,
the object, \sphinxstyleemphasis{eyebrows}, is associated with “r\_joystick\_y\_analog\_1”
and “r\_joystick\_x\_analog\_1”, which are the same input names passed
into the constructor for the eyebrows movement group.


\subsection{Output Mapping}
\label{\detokenize{mapchange:output-mapping}}
\sphinxAtStartPar
Currently each output object class has a linear mapping of inputs
to outputs. The following is the default mapping equation:

\sphinxAtStartPar
\sphinxcode{\sphinxupquote{output\_value = (input\_value \sphinxhyphen{} input\_min) * (out\_max \sphinxhyphen{} out\_min) / (input\_max \sphinxhyphen{} input\_min) + out\_min}}

\sphinxAtStartPar
Out max and min are set by the \sphinxstyleemphasis{set\_outputs} method for the output
object. Input max and min are dependant on whether the output object
accepts a digital or analog input. For digital, the max is 1 and the
min is 0. For analog, the max is 255 and the min is 0. Input value is
the value associated with the corresponding PS4 input being processed.
The output value is what is sent to the Maestro board.

\sphinxAtStartPar
For the output objects that take multiple PS4 inputs, this linear
mapping is not changed. However, the mixing of inputs is done before
the input value is sent to the mapping equation.

\sphinxAtStartPar
To change this mapping to follow an exponential growth or decay, or
any other behavior, it is recommended to create a new output object
for the desired movement group which inherits from the object that
movement group was previously set to. Then overide the \sphinxstyleemphasis{map\_values}
method which contains the mapping equation. In the overidden method,
create the new desired mapping equation. This will enable custom
mappings to be set without changing the behavior of the other
movement groups.


\chapter{Input Event Handling}
\label{\detokenize{input:input-event-handling}}\label{\detokenize{input::doc}}

\section{Manual Control}
\label{\detokenize{manualcontrol:module-manualControl}}\label{\detokenize{manualcontrol:manual-control}}\label{\detokenize{manualcontrol::doc}}\index{module@\spxentry{module}!manualControl@\spxentry{manualControl}}\index{manualControl@\spxentry{manualControl}!module@\spxentry{module}}
\sphinxAtStartPar
This module will handle all input from the Bluetooth connected PS4 controllers in an asynchronous manner. It expects two controllers to be connected 
and will not run until then. If either controller is disconnected the program will need to be restarted in order to reconnect.
Sends modified controller input to the MovementMap.
\index{ControllerEvent (class in manualControl)@\spxentry{ControllerEvent}\spxextra{class in manualControl}}

\begin{fulllineitems}
\phantomsection\label{\detokenize{manualcontrol:manualControl.ControllerEvent}}\pysiglinewithargsret{\sphinxbfcode{\sphinxupquote{class }}\sphinxcode{\sphinxupquote{manualControl.}}\sphinxbfcode{\sphinxupquote{ControllerEvent}}}{\emph{\DUrole{n}{name}}, \emph{\DUrole{n}{value}}}{}
\sphinxAtStartPar
Controller event object. To be sent to the servo handler to convert a button press to movement, for example. 
Contains a predetermined name and the current event’s value. 
For buttons the value is 1 or 0. For analog axes it is 0\sphinxhyphen{}255. For d\sphinxhyphen{}pad hat axes it is \sphinxhyphen{}1,0,1.

\end{fulllineitems}

\index{process\_events() (in module manualControl)@\spxentry{process\_events()}\spxextra{in module manualControl}}

\begin{fulllineitems}
\phantomsection\label{\detokenize{manualcontrol:manualControl.process_events}}\pysiglinewithargsret{\sphinxbfcode{\sphinxupquote{async }}\sphinxcode{\sphinxupquote{manualControl.}}\sphinxbfcode{\sphinxupquote{process\_events}}}{\emph{\DUrole{n}{device}}}{}
\sphinxAtStartPar
Async helper function. Process PS4 events and create ControllerEvent objects to be passed to the movementMap

\end{fulllineitems}



\chapter{Movement Mapping}
\label{\detokenize{mapping:movement-mapping}}\label{\detokenize{mapping::doc}}

\section{Movement Map}
\label{\detokenize{movementmap:module-MovementMap}}\label{\detokenize{movementmap:movement-map}}\label{\detokenize{movementmap::doc}}\index{module@\spxentry{module}!MovementMap@\spxentry{MovementMap}}\index{MovementMap@\spxentry{MovementMap}!module@\spxentry{module}}\index{MovementMap (class in MovementMap)@\spxentry{MovementMap}\spxextra{class in MovementMap}}

\begin{fulllineitems}
\phantomsection\label{\detokenize{movementmap:MovementMap.MovementMap}}\pysigline{\sphinxbfcode{\sphinxupquote{class }}\sphinxcode{\sphinxupquote{MovementMap.}}\sphinxbfcode{\sphinxupquote{MovementMap}}}
\sphinxAtStartPar
A class that handles controller inputs. Contains a mapping of input 
objects to output objects.

\sphinxAtStartPar
…
\begin{quote}\begin{description}
\item[{Attributes}] \leavevmode\begin{itemize}
\item {} 
\sphinxAtStartPar
\sphinxstylestrong{servoBoard}(\sphinxstyleemphasis{maestro.Controller}) \textendash{} object that 
establishes serial conection to the Maestro and has functions 
to send commands to the board

\item {} 
\sphinxAtStartPar
\sphinxstylestrong{output\_objects}(\sphinxstyleemphasis{{[}OutputObject{]}}) \textendash{} list of all the 
output objects, used to activate each output on the Maestro 
board on startup

\item {} 
\sphinxAtStartPar
\sphinxstylestrong{right\_ear}(\sphinxstyleemphasis{AnalogOutputObject}) \textendash{} object to map 
inputs to output for the right ear movements

\item {} 
\sphinxAtStartPar
\sphinxstylestrong{left\_ear}(\sphinxstyleemphasis{AnalogOutputObject}) \textendash{} object to map 
inputs to output for the left ear movements

\item {} 
\sphinxAtStartPar
\sphinxstylestrong{eyelids}(\sphinxstyleemphasis{AnalogOutputObject}) \textendash{} object to map input 
to outputs for the top and bottom eyelid movements

\item {} 
\sphinxAtStartPar
\sphinxstylestrong{eyes\_horizontal}(\sphinxstyleemphasis{AnalogOutputObject}) \textendash{} object to 
map input to outputs for the left and right eye horizontal 
movements

\item {} 
\sphinxAtStartPar
\sphinxstylestrong{eyes\_vertical}(\sphinxstyleemphasis{AnalogOutputObject}) \textendash{} object to map 
input to outputs for the left and right eye vertical movements

\item {} 
\sphinxAtStartPar
\sphinxstylestrong{eyebrows}(\sphinxstyleemphasis{EybrowsOutput}) \textendash{} object to map inputs to 
outputs for the eyebrow movements

\item {} 
\sphinxAtStartPar
\sphinxstylestrong{nose}(\sphinxstyleemphasis{DigitalOutputObject}) \textendash{} object to map input to 
output for the nose movements

\item {} 
\sphinxAtStartPar
\sphinxstylestrong{top\_lip}(\sphinxstyleemphasis{DigitalOutputObject}) \textendash{} object to map input 
to output for the top lip movements

\item {} 
\sphinxAtStartPar
\sphinxstylestrong{right\_lip}(\sphinxstyleemphasis{SideLipOutput}) \textendash{} object to map inputs to 
outputs for the right lip movements

\item {} 
\sphinxAtStartPar
\sphinxstylestrong{left\_lip}(\sphinxstyleemphasis{SideLipOutput}) \textendash{} object to map inputs to 
outputs for the left lip movements

\item {} 
\sphinxAtStartPar
\sphinxstylestrong{jaw}(\sphinxstyleemphasis{AnalogOutputObject}) \textendash{} object to map input to 
outputs for the jaw movements

\item {} 
\sphinxAtStartPar
\sphinxstylestrong{neck\_twist}(\sphinxstyleemphasis{AnalogOutputObject}) \textendash{} object to map input 
to output for the neck twist movements

\item {} 
\sphinxAtStartPar
\sphinxstylestrong{neck\_tilt}(\sphinxstyleemphasis{NeckTiltOutput}) \textendash{} object to map inputs to 
outpus for the neck tilt movements

\item {} 
\sphinxAtStartPar
\sphinxstylestrong{input\_map}(\sphinxstyleemphasis{dictionary {[}input\_name (string): output\_object 
(OutputObject){]}}) \textendash{} mapping of the controller inputs to movement 
outputs

\end{itemize}

\end{description}\end{quote}

\sphinxAtStartPar
…

\sphinxAtStartPar
\sphinxstylestrong{Methods}
\index{\_\_init\_\_() (MovementMap.MovementMap method)@\spxentry{\_\_init\_\_()}\spxextra{MovementMap.MovementMap method}}

\begin{fulllineitems}
\phantomsection\label{\detokenize{movementmap:MovementMap.MovementMap.__init__}}\pysiglinewithargsret{\sphinxbfcode{\sphinxupquote{\_\_init\_\_}}}{}{}
\sphinxAtStartPar
Class constructor. Creates needed output objects and sets their parameters. Also 
creates the input map.

\end{fulllineitems}

\index{process\_input() (MovementMap.MovementMap method)@\spxentry{process\_input()}\spxextra{MovementMap.MovementMap method}}

\begin{fulllineitems}
\phantomsection\label{\detokenize{movementmap:MovementMap.MovementMap.process_input}}\pysiglinewithargsret{\sphinxbfcode{\sphinxupquote{process\_input}}}{\emph{\DUrole{n}{input\_object}}}{}
\sphinxAtStartPar
Using the input map, determines the correct output object to send the input value to. 
The returned outputs are passed to the maestro board.
\begin{quote}\begin{description}
\item[{Parameters}] \leavevmode
\sphinxAtStartPar
\sphinxstyleliteralstrong{\sphinxupquote{input\_object}} ({\hyperref[\detokenize{manualcontrol:manualControl.ControllerEvent}]{\sphinxcrossref{\sphinxstyleliteralemphasis{\sphinxupquote{ControllerEvent}}}}}) \textendash{} object containing the name of the input and its associated value 
to be processed

\end{description}\end{quote}

\end{fulllineitems}

\index{send\_outputs() (MovementMap.MovementMap method)@\spxentry{send\_outputs()}\spxextra{MovementMap.MovementMap method}}

\begin{fulllineitems}
\phantomsection\label{\detokenize{movementmap:MovementMap.MovementMap.send_outputs}}\pysiglinewithargsret{\sphinxbfcode{\sphinxupquote{send\_outputs}}}{\emph{\DUrole{n}{num\_outputs}}, \emph{\DUrole{n}{channel}}, \emph{\DUrole{n}{output}}}{}
\sphinxAtStartPar
Sends the output to the corresponding channel from the given lists of channels and
outputs.
\begin{quote}\begin{description}
\item[{Parameters}] \leavevmode\begin{itemize}
\item {} 
\sphinxAtStartPar
\sphinxstyleliteralstrong{\sphinxupquote{num\_outputs}} (\sphinxstyleliteralemphasis{\sphinxupquote{int}}) \textendash{} number of channels and outputs to loop through

\item {} 
\sphinxAtStartPar
\sphinxstyleliteralstrong{\sphinxupquote{channel}} (\sphinxstyleliteralemphasis{\sphinxupquote{{[}}}\sphinxstyleliteralemphasis{\sphinxupquote{int}}\sphinxstyleliteralemphasis{\sphinxupquote{{]}}}) \textendash{} channel numbers on the Maestro board to send outputs to

\item {} 
\sphinxAtStartPar
\sphinxstyleliteralstrong{\sphinxupquote{output}} (\sphinxstyleliteralemphasis{\sphinxupquote{{[}}}\sphinxstyleliteralemphasis{\sphinxupquote{double}}\sphinxstyleliteralemphasis{\sphinxupquote{{]}}}) \textendash{} values to output on the Maestro board, cast to int to ensure proper
typing

\end{itemize}

\end{description}\end{quote}

\end{fulllineitems}

\index{start\_outputs() (MovementMap.MovementMap method)@\spxentry{start\_outputs()}\spxextra{MovementMap.MovementMap method}}

\begin{fulllineitems}
\phantomsection\label{\detokenize{movementmap:MovementMap.MovementMap.start_outputs}}\pysiglinewithargsret{\sphinxbfcode{\sphinxupquote{start\_outputs}}}{}{}~\begin{description}
\item[{Sends the starting outputs for each motor to the Maestro board to activate each channel}] \leavevmode
\sphinxAtStartPar
in default positions.

\end{description}

\end{fulllineitems}


\end{fulllineitems}



\chapter{Output Objects}
\label{\detokenize{output:output-objects}}\label{\detokenize{output::doc}}

\section{Base Objects}
\label{\detokenize{base:base-objects}}\label{\detokenize{base::doc}}

\subsection{Output Object}
\label{\detokenize{base:module-OutputObject}}\label{\detokenize{base:output-object}}\index{module@\spxentry{module}!OutputObject@\spxentry{OutputObject}}\index{OutputObject@\spxentry{OutputObject}!module@\spxentry{module}}\index{OutputObject (class in OutputObject)@\spxentry{OutputObject}\spxextra{class in OutputObject}}

\begin{fulllineitems}
\phantomsection\label{\detokenize{base:OutputObject.OutputObject}}\pysiglinewithargsret{\sphinxbfcode{\sphinxupquote{class }}\sphinxcode{\sphinxupquote{OutputObject.}}\sphinxbfcode{\sphinxupquote{OutputObject}}}{\emph{\DUrole{n}{name}}, \emph{\DUrole{n}{num\_outputs}}, \emph{\DUrole{n}{channels\_output}}}{}
\sphinxAtStartPar
A base class to represent a an output object for a controller input.

\sphinxAtStartPar
…
\begin{quote}\begin{description}
\item[{Attributes}] \leavevmode\begin{itemize}
\item {} 
\sphinxAtStartPar
\sphinxstylestrong{name}(\sphinxstyleemphasis{str}) \textendash{} name of the servo group for the output

\item {} 
\sphinxAtStartPar
\sphinxstylestrong{num\_outputs}(\sphinxstyleemphasis{int}) \textendash{} total number of outputs 
calculated from the given input

\item {} 
\sphinxAtStartPar
\sphinxstylestrong{channels\_output}(\sphinxstyleemphasis{{[}int{]}}) \textendash{} channels corresponding to 
the servos controlled by the outputs

\item {} 
\sphinxAtStartPar
\sphinxstylestrong{maximums\_output}(\sphinxstyleemphasis{{[}int{]}}) \textendash{} maximum pulse width values 
for the corresponding servo channel

\item {} 
\sphinxAtStartPar
\sphinxstylestrong{minimums\_output}(\sphinxstyleemphasis{{[}int{]}}) \textendash{} minimum pulse width values 
for the corresponding servo channel

\item {} 
\sphinxAtStartPar
\sphinxstylestrong{default\_output}(\sphinxstyleemphasis{{[}int{]}}) \textendash{} default output position of 
the servos for start up position and used for some multi input 
objects

\item {} 
\sphinxAtStartPar
\sphinxstylestrong{current\_output}(\sphinxstyleemphasis{{[}int{]}}) \textendash{} current output position of the 
servos to be used for increment mode or multi input objects

\item {} 
\sphinxAtStartPar
\sphinxstylestrong{maximum\_input}(\sphinxstyleemphasis{int}) \textendash{} maximum input value used for mapping 
inputs to outputs, 255 for analog and 1 for digital

\item {} 
\sphinxAtStartPar
\sphinxstylestrong{minimum\_input}(\sphinxstyleemphasis{int}) \textendash{} minimum input value used for mapping 
inputs to outputs, 0 for both analog and digital

\item {} 
\sphinxAtStartPar
\sphinxstylestrong{is\_inverted}(\sphinxstyleemphasis{{[}Boolean{]}}) \textendash{} whether to invert the output 
mapping

\item {} 
\sphinxAtStartPar
\sphinxstylestrong{control\_type}(\sphinxstyleemphasis{Enum ControlType}) \textendash{} mode on how to determine 
the output

\item {} 
\sphinxAtStartPar
\sphinxstylestrong{toggle\_state}(\sphinxstyleemphasis{Enum ToggleState}) \textendash{} used when control mode 
is set to TOGGLE to determine the current output state

\end{itemize}

\end{description}\end{quote}

\sphinxAtStartPar
…

\sphinxAtStartPar
\sphinxstylestrong{Methods}
\index{\_\_init\_\_() (OutputObject.OutputObject method)@\spxentry{\_\_init\_\_()}\spxextra{OutputObject.OutputObject method}}

\begin{fulllineitems}
\phantomsection\label{\detokenize{base:OutputObject.OutputObject.__init__}}\pysiglinewithargsret{\sphinxbfcode{\sphinxupquote{\_\_init\_\_}}}{\emph{\DUrole{n}{name}}, \emph{\DUrole{n}{num\_outputs}}, \emph{\DUrole{n}{channels\_output}}}{}
\sphinxAtStartPar
Class constructor. Assigns the values passed in and initalizes remaining 
members to default values.
\begin{quote}\begin{description}
\item[{Parameters}] \leavevmode\begin{itemize}
\item {} 
\sphinxAtStartPar
\sphinxstyleliteralstrong{\sphinxupquote{name}} (\sphinxstyleliteralemphasis{\sphinxupquote{string}}) \textendash{} name of the output group represented by output object

\item {} 
\sphinxAtStartPar
\sphinxstyleliteralstrong{\sphinxupquote{num\_outputs}} (\sphinxstyleliteralemphasis{\sphinxupquote{int}}) \textendash{} number of output channels controlled by the output 
object

\item {} 
\sphinxAtStartPar
\sphinxstyleliteralstrong{\sphinxupquote{channels\_output}} (\sphinxstyleliteralemphasis{\sphinxupquote{{[}}}\sphinxstyleliteralemphasis{\sphinxupquote{int}}\sphinxstyleliteralemphasis{\sphinxupquote{{]}}}) \textendash{} list of the output channels used by the object

\end{itemize}

\end{description}\end{quote}

\end{fulllineitems}

\index{get\_default\_outputs() (OutputObject.OutputObject method)@\spxentry{get\_default\_outputs()}\spxextra{OutputObject.OutputObject method}}

\begin{fulllineitems}
\phantomsection\label{\detokenize{base:OutputObject.OutputObject.get_default_outputs}}\pysiglinewithargsret{\sphinxbfcode{\sphinxupquote{get\_default\_outputs}}}{}{}
\sphinxAtStartPar
Returns the default output values for the object.
\begin{quote}\begin{description}
\item[{Returns}] \leavevmode
\sphinxAtStartPar
Two lists. The first list is the output channels and the second 
is the default output for those channels.

\item[{Return type}] \leavevmode
\sphinxAtStartPar
{[}{[}int{]}, {[}int{]}{]}

\end{description}\end{quote}

\end{fulllineitems}

\index{get\_num\_channels() (OutputObject.OutputObject method)@\spxentry{get\_num\_channels()}\spxextra{OutputObject.OutputObject method}}

\begin{fulllineitems}
\phantomsection\label{\detokenize{base:OutputObject.OutputObject.get_num_channels}}\pysiglinewithargsret{\sphinxbfcode{\sphinxupquote{get\_num\_channels}}}{}{}
\sphinxAtStartPar
Returns the number of output channels for the object.
\begin{quote}\begin{description}
\item[{Returns}] \leavevmode
\sphinxAtStartPar
number of output channels the object sends output to

\item[{Return type}] \leavevmode
\sphinxAtStartPar
int

\end{description}\end{quote}

\end{fulllineitems}

\index{map\_values() (OutputObject.OutputObject method)@\spxentry{map\_values()}\spxextra{OutputObject.OutputObject method}}

\begin{fulllineitems}
\phantomsection\label{\detokenize{base:OutputObject.OutputObject.map_values}}\pysiglinewithargsret{\sphinxbfcode{\sphinxupquote{map\_values}}}{\emph{\DUrole{n}{value}}, \emph{\DUrole{n}{input\_min}}, \emph{\DUrole{n}{input\_max}}, \emph{\DUrole{n}{out\_min}}, \emph{\DUrole{n}{out\_max}}}{}
\sphinxAtStartPar
Maps an input value to its output.
\begin{quote}\begin{description}
\item[{Parameters}] \leavevmode\begin{itemize}
\item {} 
\sphinxAtStartPar
\sphinxstyleliteralstrong{\sphinxupquote{value}} (\sphinxstyleliteralemphasis{\sphinxupquote{float}}) \textendash{} value of the input to map to an output value

\item {} 
\sphinxAtStartPar
\sphinxstyleliteralstrong{\sphinxupquote{input\_min}} (\sphinxstyleliteralemphasis{\sphinxupquote{float}}) \textendash{} minimun input value in input range

\item {} 
\sphinxAtStartPar
\sphinxstyleliteralstrong{\sphinxupquote{input\_max}} (\sphinxstyleliteralemphasis{\sphinxupquote{float}}) \textendash{} maximun input value in input range

\item {} 
\sphinxAtStartPar
\sphinxstyleliteralstrong{\sphinxupquote{out\_min}} (\sphinxstyleliteralemphasis{\sphinxupquote{int}}) \textendash{} minimun output value in output range

\item {} 
\sphinxAtStartPar
\sphinxstyleliteralstrong{\sphinxupquote{out\_max}} (\sphinxstyleliteralemphasis{\sphinxupquote{int}}) \textendash{} maximun output value in output range

\end{itemize}

\item[{Returns}] \leavevmode
\sphinxAtStartPar
pulse width to output for the given input value

\item[{Return type}] \leavevmode
\sphinxAtStartPar
int

\end{description}\end{quote}

\end{fulllineitems}

\index{set\_control\_direct() (OutputObject.OutputObject method)@\spxentry{set\_control\_direct()}\spxextra{OutputObject.OutputObject method}}

\begin{fulllineitems}
\phantomsection\label{\detokenize{base:OutputObject.OutputObject.set_control_direct}}\pysiglinewithargsret{\sphinxbfcode{\sphinxupquote{set\_control\_direct}}}{}{}
\sphinxAtStartPar
Sets control mode to direct.

\end{fulllineitems}

\index{set\_control\_increment() (OutputObject.OutputObject method)@\spxentry{set\_control\_increment()}\spxextra{OutputObject.OutputObject method}}

\begin{fulllineitems}
\phantomsection\label{\detokenize{base:OutputObject.OutputObject.set_control_increment}}\pysiglinewithargsret{\sphinxbfcode{\sphinxupquote{set\_control\_increment}}}{}{}
\sphinxAtStartPar
Sets control mode to increment.

\end{fulllineitems}

\index{set\_control\_toggle() (OutputObject.OutputObject method)@\spxentry{set\_control\_toggle()}\spxextra{OutputObject.OutputObject method}}

\begin{fulllineitems}
\phantomsection\label{\detokenize{base:OutputObject.OutputObject.set_control_toggle}}\pysiglinewithargsret{\sphinxbfcode{\sphinxupquote{set\_control\_toggle}}}{}{}
\sphinxAtStartPar
Sets control mode to toggle.

\end{fulllineitems}

\index{set\_inversion() (OutputObject.OutputObject method)@\spxentry{set\_inversion()}\spxextra{OutputObject.OutputObject method}}

\begin{fulllineitems}
\phantomsection\label{\detokenize{base:OutputObject.OutputObject.set_inversion}}\pysiglinewithargsret{\sphinxbfcode{\sphinxupquote{set\_inversion}}}{\emph{\DUrole{n}{is\_inverted}}}{}
\sphinxAtStartPar
Sets whether to invert the output signal or not.
\begin{quote}\begin{description}
\item[{Parameters}] \leavevmode
\sphinxAtStartPar
\sphinxstyleliteralstrong{\sphinxupquote{is\_inverted}} (\sphinxstyleliteralemphasis{\sphinxupquote{boolean}}) \textendash{} the state to set the attribute is\_inverted to

\end{description}\end{quote}

\end{fulllineitems}

\index{set\_outputs() (OutputObject.OutputObject method)@\spxentry{set\_outputs()}\spxextra{OutputObject.OutputObject method}}

\begin{fulllineitems}
\phantomsection\label{\detokenize{base:OutputObject.OutputObject.set_outputs}}\pysiglinewithargsret{\sphinxbfcode{\sphinxupquote{set\_outputs}}}{\emph{\DUrole{n}{minimums\_output}}, \emph{\DUrole{n}{default\_output}}, \emph{\DUrole{n}{maximums\_output}}}{}
\sphinxAtStartPar
Sets which channels to output to and the minimun, default, and maximum 
pulse width for each of those channels.

\sphinxAtStartPar
Also sets current outputs to the same values as the default.
\begin{quote}\begin{description}
\item[{Parameters}] \leavevmode\begin{itemize}
\item {} 
\sphinxAtStartPar
\sphinxstyleliteralstrong{\sphinxupquote{minimums\_output}} (\sphinxstyleliteralemphasis{\sphinxupquote{{[}}}\sphinxstyleliteralemphasis{\sphinxupquote{int}}\sphinxstyleliteralemphasis{\sphinxupquote{{]}}}) \textendash{} minimum pulse width values for the corresponding 
servo channel

\item {} 
\sphinxAtStartPar
\sphinxstyleliteralstrong{\sphinxupquote{default\_output}} (\sphinxstyleliteralemphasis{\sphinxupquote{{[}}}\sphinxstyleliteralemphasis{\sphinxupquote{int}}\sphinxstyleliteralemphasis{\sphinxupquote{{]}}}) \textendash{} neutral pulse width values for the corresponding 
servo channel, also determines starting position

\item {} 
\sphinxAtStartPar
\sphinxstyleliteralstrong{\sphinxupquote{maximums\_output}} (\sphinxstyleliteralemphasis{\sphinxupquote{{[}}}\sphinxstyleliteralemphasis{\sphinxupquote{int}}\sphinxstyleliteralemphasis{\sphinxupquote{{]}}}) \textendash{} maximum pulse width values for the corresponding 
servo channel

\end{itemize}

\end{description}\end{quote}

\end{fulllineitems}


\end{fulllineitems}

\index{ControlType (class in OutputObject)@\spxentry{ControlType}\spxextra{class in OutputObject}}

\begin{fulllineitems}
\phantomsection\label{\detokenize{base:OutputObject.ControlType}}\pysiglinewithargsret{\sphinxbfcode{\sphinxupquote{class }}\sphinxcode{\sphinxupquote{OutputObject.}}\sphinxbfcode{\sphinxupquote{ControlType}}}{\emph{\DUrole{n}{value}}}{}
\sphinxAtStartPar
The output mode of the input. This determines how output is calculated 
from the inputs.
\begin{quote}\begin{description}
\item[{States}] \leavevmode\begin{itemize}
\item {} 
\sphinxAtStartPar
\sphinxstylestrong{DIRECT} : output is mapped directly to the input value

\item {} \begin{description}
\item[{\sphinxstylestrong{TOGGLE}}] \leavevmode{[}output is switched between states and is triggered by a {]}
\sphinxAtStartPar
zero input value

\end{description}

\item {} 
\sphinxAtStartPar
\sphinxstylestrong{INCREMENT} : output is incremented when input is received

\end{itemize}

\end{description}\end{quote}

\end{fulllineitems}

\index{ToggleState (class in OutputObject)@\spxentry{ToggleState}\spxextra{class in OutputObject}}

\begin{fulllineitems}
\phantomsection\label{\detokenize{base:OutputObject.ToggleState}}\pysiglinewithargsret{\sphinxbfcode{\sphinxupquote{class }}\sphinxcode{\sphinxupquote{OutputObject.}}\sphinxbfcode{\sphinxupquote{ToggleState}}}{\emph{\DUrole{n}{value}}}{}
\sphinxAtStartPar
Used to determine output behavior when the control type is set to toggle mode.
\begin{quote}\begin{description}
\item[{States}] \leavevmode\begin{itemize}
\item {} 
\sphinxAtStartPar
\sphinxstylestrong{ON} : the output is toggled “on”

\item {} 
\sphinxAtStartPar
\sphinxstylestrong{OFF} : the output is toggled “off”

\end{itemize}

\end{description}\end{quote}

\end{fulllineitems}



\subsection{Multi Input Output Object}
\label{\detokenize{base:module-MultiInputOutputObject}}\label{\detokenize{base:multi-input-output-object}}\index{module@\spxentry{module}!MultiInputOutputObject@\spxentry{MultiInputOutputObject}}\index{MultiInputOutputObject@\spxentry{MultiInputOutputObject}!module@\spxentry{module}}\index{MultiInputOutputObject (class in MultiInputOutputObject)@\spxentry{MultiInputOutputObject}\spxextra{class in MultiInputOutputObject}}

\begin{fulllineitems}
\phantomsection\label{\detokenize{base:MultiInputOutputObject.MultiInputOutputObject}}\pysiglinewithargsret{\sphinxbfcode{\sphinxupquote{class }}\sphinxcode{\sphinxupquote{MultiInputOutputObject.}}\sphinxbfcode{\sphinxupquote{MultiInputOutputObject}}}{\emph{\DUrole{n}{name}}, \emph{\DUrole{n}{num\_outputs}}, \emph{\DUrole{n}{channels\_output}}, \emph{\DUrole{n}{names\_input}}}{}
\sphinxAtStartPar
A base class to represent a an output object for movement groups 
that take multiple inputs.

\sphinxAtStartPar
It inherits attributes and methods from OutputObject.

\sphinxAtStartPar
…
\begin{quote}\begin{description}
\item[{Attributes}] \leavevmode\begin{itemize}
\item {} 
\sphinxAtStartPar
\sphinxstylestrong{name}(\sphinxstyleemphasis{str}) \textendash{} name of the servo group for the output

\item {} 
\sphinxAtStartPar
\sphinxstylestrong{num\_outputs}(\sphinxstyleemphasis{int}) \textendash{} total number of outputs 
calculated from the given input

\item {} 
\sphinxAtStartPar
\sphinxstylestrong{channels\_output}(\sphinxstyleemphasis{{[}int{]}}) \textendash{} channels corresponding to 
the servos controlled by the outputs

\item {} 
\sphinxAtStartPar
\sphinxstylestrong{maximums\_output}(\sphinxstyleemphasis{{[}int{]}}) \textendash{} maximum pulse width values 
for the corresponding servo channel

\item {} 
\sphinxAtStartPar
\sphinxstylestrong{minimums\_output}(\sphinxstyleemphasis{{[}int{]}}) \textendash{} minimum pulse width values 
for the corresponding servo channel

\item {} 
\sphinxAtStartPar
\sphinxstylestrong{default\_output}(\sphinxstyleemphasis{{[}int{]}}) \textendash{} default output position of 
the servos for start up position and used for some multi input 
objects

\item {} 
\sphinxAtStartPar
\sphinxstylestrong{current\_output}(\sphinxstyleemphasis{{[}int{]}}) \textendash{} current output position of the 
servos to be used for increment mode or multi input objects

\item {} 
\sphinxAtStartPar
\sphinxstylestrong{maximum\_input}(\sphinxstyleemphasis{int}) \textendash{} maximum input value used for mapping 
inputs to outputs, 255 for analog and 1 for digital

\item {} 
\sphinxAtStartPar
\sphinxstylestrong{minimum\_input}(\sphinxstyleemphasis{int}) \textendash{} minimum input value used for mapping 
inputs to outputs, 0 for both analog and digital

\item {} 
\sphinxAtStartPar
\sphinxstylestrong{is\_inverted}(\sphinxstyleemphasis{{[}Boolean{]}}) \textendash{} whether to invert the output 
mapping

\item {} 
\sphinxAtStartPar
\sphinxstylestrong{control\_type}(\sphinxstyleemphasis{Enum ControlType}) \textendash{} mode on how to determine 
the output

\item {} 
\sphinxAtStartPar
\sphinxstylestrong{toggle\_state}(\sphinxstyleemphasis{Enum ToggleState}) \textendash{} used when control mode 
is set to TOGGLE to determine the current output state

\item {} 
\sphinxAtStartPar
\sphinxstylestrong{names\_input}(\sphinxstyleemphasis{{[}str{]}}) \textendash{} list of the input names used by 
the object, this allows the object to know which input value to 
update

\item {} 
\sphinxAtStartPar
\sphinxstylestrong{num\_inputs}(\sphinxstyleemphasis{int}) \textendash{} number of inputs the object uses

\item {} 
\sphinxAtStartPar
\sphinxstylestrong{current\_input}(\sphinxstyleemphasis{{[}int{]}}) \textendash{} since input is given one at a 
time, this list keeps track of previous inputs

\item {} 
\sphinxAtStartPar
\sphinxstylestrong{out\_raw\_min}(\sphinxstyleemphasis{int}) \textendash{} minimum value of an intermediate 
value used to calculate output

\item {} 
\sphinxAtStartPar
\sphinxstylestrong{out\_raw\_max}(\sphinxstyleemphasis{int}) \textendash{} maximum value of an intermediate 
value used to calculate output

\item {} 
\sphinxAtStartPar
\sphinxstylestrong{raw\_output}(\sphinxstyleemphasis{{[}int{]}}) \textendash{} list to store the calculated 
intermediate values that will be mapped to final output values

\end{itemize}

\end{description}\end{quote}

\sphinxAtStartPar
…

\sphinxAtStartPar
\sphinxstylestrong{Methods}
\index{\_\_init\_\_() (MultiInputOutputObject.MultiInputOutputObject method)@\spxentry{\_\_init\_\_()}\spxextra{MultiInputOutputObject.MultiInputOutputObject method}}

\begin{fulllineitems}
\phantomsection\label{\detokenize{base:MultiInputOutputObject.MultiInputOutputObject.__init__}}\pysiglinewithargsret{\sphinxbfcode{\sphinxupquote{\_\_init\_\_}}}{\emph{\DUrole{n}{name}}, \emph{\DUrole{n}{num\_outputs}}, \emph{\DUrole{n}{channels\_output}}, \emph{\DUrole{n}{names\_input}}}{}
\sphinxAtStartPar
Class constructor. Assigns the values passed in and initalizes remaining 
members to default values.
\begin{quote}\begin{description}
\item[{Parameters}] \leavevmode\begin{itemize}
\item {} 
\sphinxAtStartPar
\sphinxstyleliteralstrong{\sphinxupquote{name}} (\sphinxstyleliteralemphasis{\sphinxupquote{string}}) \textendash{} name of the output group represented by output object

\item {} 
\sphinxAtStartPar
\sphinxstyleliteralstrong{\sphinxupquote{num\_outputs}} (\sphinxstyleliteralemphasis{\sphinxupquote{int}}) \textendash{} number of output channels controlled by the output 
object

\item {} 
\sphinxAtStartPar
\sphinxstyleliteralstrong{\sphinxupquote{channels\_output}} (\sphinxstyleliteralemphasis{\sphinxupquote{{[}}}\sphinxstyleliteralemphasis{\sphinxupquote{int}}\sphinxstyleliteralemphasis{\sphinxupquote{{]}}}) \textendash{} list of the output channels used by the object

\item {} 
\sphinxAtStartPar
\sphinxstyleliteralstrong{\sphinxupquote{names\_input}} (\sphinxstyleliteralemphasis{\sphinxupquote{{[}}}\sphinxstyleliteralemphasis{\sphinxupquote{str}}\sphinxstyleliteralemphasis{\sphinxupquote{{]}}}) \textendash{} the names of the associated controller inputs with 
the object

\end{itemize}

\end{description}\end{quote}

\end{fulllineitems}

\index{get\_default\_outputs() (MultiInputOutputObject.MultiInputOutputObject method)@\spxentry{get\_default\_outputs()}\spxextra{MultiInputOutputObject.MultiInputOutputObject method}}

\begin{fulllineitems}
\phantomsection\label{\detokenize{base:MultiInputOutputObject.MultiInputOutputObject.get_default_outputs}}\pysiglinewithargsret{\sphinxbfcode{\sphinxupquote{get\_default\_outputs}}}{}{}
\sphinxAtStartPar
Returns the default output values for the object.
\begin{quote}\begin{description}
\item[{Returns}] \leavevmode
\sphinxAtStartPar
Two lists. The first list is the output channels and the second 
is the default output for those channels.

\item[{Return type}] \leavevmode
\sphinxAtStartPar
{[}{[}int{]}, {[}int{]}{]}

\end{description}\end{quote}

\end{fulllineitems}

\index{get\_num\_channels() (MultiInputOutputObject.MultiInputOutputObject method)@\spxentry{get\_num\_channels()}\spxextra{MultiInputOutputObject.MultiInputOutputObject method}}

\begin{fulllineitems}
\phantomsection\label{\detokenize{base:MultiInputOutputObject.MultiInputOutputObject.get_num_channels}}\pysiglinewithargsret{\sphinxbfcode{\sphinxupquote{get\_num\_channels}}}{}{}
\sphinxAtStartPar
Returns the number of output channels for the object.
\begin{quote}\begin{description}
\item[{Returns}] \leavevmode
\sphinxAtStartPar
number of output channels the object sends output to

\item[{Return type}] \leavevmode
\sphinxAtStartPar
int

\end{description}\end{quote}

\end{fulllineitems}

\index{map\_values() (MultiInputOutputObject.MultiInputOutputObject method)@\spxentry{map\_values()}\spxextra{MultiInputOutputObject.MultiInputOutputObject method}}

\begin{fulllineitems}
\phantomsection\label{\detokenize{base:MultiInputOutputObject.MultiInputOutputObject.map_values}}\pysiglinewithargsret{\sphinxbfcode{\sphinxupquote{map\_values}}}{\emph{\DUrole{n}{value}}, \emph{\DUrole{n}{input\_min}}, \emph{\DUrole{n}{input\_max}}, \emph{\DUrole{n}{out\_min}}, \emph{\DUrole{n}{out\_max}}}{}
\sphinxAtStartPar
Maps an input value to its output.
\begin{quote}\begin{description}
\item[{Parameters}] \leavevmode\begin{itemize}
\item {} 
\sphinxAtStartPar
\sphinxstyleliteralstrong{\sphinxupquote{value}} (\sphinxstyleliteralemphasis{\sphinxupquote{float}}) \textendash{} value of the input to map to an output value

\item {} 
\sphinxAtStartPar
\sphinxstyleliteralstrong{\sphinxupquote{input\_min}} (\sphinxstyleliteralemphasis{\sphinxupquote{float}}) \textendash{} minimun input value in input range

\item {} 
\sphinxAtStartPar
\sphinxstyleliteralstrong{\sphinxupquote{input\_max}} (\sphinxstyleliteralemphasis{\sphinxupquote{float}}) \textendash{} maximun input value in input range

\item {} 
\sphinxAtStartPar
\sphinxstyleliteralstrong{\sphinxupquote{out\_min}} (\sphinxstyleliteralemphasis{\sphinxupquote{int}}) \textendash{} minimun output value in output range

\item {} 
\sphinxAtStartPar
\sphinxstyleliteralstrong{\sphinxupquote{out\_max}} (\sphinxstyleliteralemphasis{\sphinxupquote{int}}) \textendash{} maximun output value in output range

\end{itemize}

\item[{Returns}] \leavevmode
\sphinxAtStartPar
pulse width to output for the given input value

\item[{Return type}] \leavevmode
\sphinxAtStartPar
int

\end{description}\end{quote}

\end{fulllineitems}

\index{set\_control\_direct() (MultiInputOutputObject.MultiInputOutputObject method)@\spxentry{set\_control\_direct()}\spxextra{MultiInputOutputObject.MultiInputOutputObject method}}

\begin{fulllineitems}
\phantomsection\label{\detokenize{base:MultiInputOutputObject.MultiInputOutputObject.set_control_direct}}\pysiglinewithargsret{\sphinxbfcode{\sphinxupquote{set\_control\_direct}}}{}{}
\sphinxAtStartPar
Sets control mode to direct.

\end{fulllineitems}

\index{set\_control\_increment() (MultiInputOutputObject.MultiInputOutputObject method)@\spxentry{set\_control\_increment()}\spxextra{MultiInputOutputObject.MultiInputOutputObject method}}

\begin{fulllineitems}
\phantomsection\label{\detokenize{base:MultiInputOutputObject.MultiInputOutputObject.set_control_increment}}\pysiglinewithargsret{\sphinxbfcode{\sphinxupquote{set\_control\_increment}}}{}{}
\sphinxAtStartPar
Sets control mode to increment.

\end{fulllineitems}

\index{set\_control\_toggle() (MultiInputOutputObject.MultiInputOutputObject method)@\spxentry{set\_control\_toggle()}\spxextra{MultiInputOutputObject.MultiInputOutputObject method}}

\begin{fulllineitems}
\phantomsection\label{\detokenize{base:MultiInputOutputObject.MultiInputOutputObject.set_control_toggle}}\pysiglinewithargsret{\sphinxbfcode{\sphinxupquote{set\_control\_toggle}}}{}{}
\sphinxAtStartPar
Sets control mode to toggle.

\end{fulllineitems}

\index{set\_inversion() (MultiInputOutputObject.MultiInputOutputObject method)@\spxentry{set\_inversion()}\spxextra{MultiInputOutputObject.MultiInputOutputObject method}}

\begin{fulllineitems}
\phantomsection\label{\detokenize{base:MultiInputOutputObject.MultiInputOutputObject.set_inversion}}\pysiglinewithargsret{\sphinxbfcode{\sphinxupquote{set\_inversion}}}{\emph{\DUrole{n}{is\_inverted}}}{}
\sphinxAtStartPar
Sets whether to invert the output signal or not.
\begin{quote}\begin{description}
\item[{Parameters}] \leavevmode
\sphinxAtStartPar
\sphinxstyleliteralstrong{\sphinxupquote{is\_inverted}} (\sphinxstyleliteralemphasis{\sphinxupquote{boolean}}) \textendash{} the state to set the attribute is\_inverted to

\end{description}\end{quote}

\end{fulllineitems}

\index{set\_outputs() (MultiInputOutputObject.MultiInputOutputObject method)@\spxentry{set\_outputs()}\spxextra{MultiInputOutputObject.MultiInputOutputObject method}}

\begin{fulllineitems}
\phantomsection\label{\detokenize{base:MultiInputOutputObject.MultiInputOutputObject.set_outputs}}\pysiglinewithargsret{\sphinxbfcode{\sphinxupquote{set\_outputs}}}{\emph{\DUrole{n}{minimums\_output}}, \emph{\DUrole{n}{default\_output}}, \emph{\DUrole{n}{maximums\_output}}}{}
\sphinxAtStartPar
Sets which channels to output to and the minimun, default, and maximum 
pulse width for each of those channels.

\sphinxAtStartPar
Also sets current outputs to the same values as the default.
\begin{quote}\begin{description}
\item[{Parameters}] \leavevmode\begin{itemize}
\item {} 
\sphinxAtStartPar
\sphinxstyleliteralstrong{\sphinxupquote{minimums\_output}} (\sphinxstyleliteralemphasis{\sphinxupquote{{[}}}\sphinxstyleliteralemphasis{\sphinxupquote{int}}\sphinxstyleliteralemphasis{\sphinxupquote{{]}}}) \textendash{} minimum pulse width values for the corresponding 
servo channel

\item {} 
\sphinxAtStartPar
\sphinxstyleliteralstrong{\sphinxupquote{default\_output}} (\sphinxstyleliteralemphasis{\sphinxupquote{{[}}}\sphinxstyleliteralemphasis{\sphinxupquote{int}}\sphinxstyleliteralemphasis{\sphinxupquote{{]}}}) \textendash{} neutral pulse width values for the corresponding 
servo channel, also determines starting position

\item {} 
\sphinxAtStartPar
\sphinxstyleliteralstrong{\sphinxupquote{maximums\_output}} (\sphinxstyleliteralemphasis{\sphinxupquote{{[}}}\sphinxstyleliteralemphasis{\sphinxupquote{int}}\sphinxstyleliteralemphasis{\sphinxupquote{{]}}}) \textendash{} maximum pulse width values for the corresponding 
servo channel

\end{itemize}

\end{description}\end{quote}

\end{fulllineitems}


\end{fulllineitems}



\section{Generic Objects}
\label{\detokenize{generic:generic-objects}}\label{\detokenize{generic::doc}}

\subsection{Digital Output Object}
\label{\detokenize{generic:module-DigitalOutputObject}}\label{\detokenize{generic:digital-output-object}}\index{module@\spxentry{module}!DigitalOutputObject@\spxentry{DigitalOutputObject}}\index{DigitalOutputObject@\spxentry{DigitalOutputObject}!module@\spxentry{module}}\index{DigitalOutputObject (class in DigitalOutputObject)@\spxentry{DigitalOutputObject}\spxextra{class in DigitalOutputObject}}

\begin{fulllineitems}
\phantomsection\label{\detokenize{generic:DigitalOutputObject.DigitalOutputObject}}\pysiglinewithargsret{\sphinxbfcode{\sphinxupquote{class }}\sphinxcode{\sphinxupquote{DigitalOutputObject.}}\sphinxbfcode{\sphinxupquote{DigitalOutputObject}}}{\emph{\DUrole{n}{name}}, \emph{\DUrole{n}{num\_outputs}}, \emph{\DUrole{n}{channels\_output}}}{}
\sphinxAtStartPar
A class to represent a an output object for a digital controller input.

\sphinxAtStartPar
It inherits attributes and methods from OutputObject.

\sphinxAtStartPar
…
\begin{quote}\begin{description}
\item[{Attributes}] \leavevmode\begin{itemize}
\item {} 
\sphinxAtStartPar
\sphinxstylestrong{name}(\sphinxstyleemphasis{str}) \textendash{} name of the servo group for the output

\item {} 
\sphinxAtStartPar
\sphinxstylestrong{num\_outputs}(\sphinxstyleemphasis{int}) \textendash{} total number of outputs 
calculated from the given input

\item {} 
\sphinxAtStartPar
\sphinxstylestrong{channels\_output}(\sphinxstyleemphasis{{[}int{]}}) \textendash{} channels corresponding to 
the servos controlled by the outputs

\item {} 
\sphinxAtStartPar
\sphinxstylestrong{maximums\_output}(\sphinxstyleemphasis{{[}int{]}}) \textendash{} maximum pulse width values 
for the corresponding servo channel

\item {} 
\sphinxAtStartPar
\sphinxstylestrong{minimums\_output}(\sphinxstyleemphasis{{[}int{]}}) \textendash{} minimum pulse width values 
for the corresponding servo channel

\item {} 
\sphinxAtStartPar
\sphinxstylestrong{default\_output}(\sphinxstyleemphasis{{[}int{]}}) \textendash{} default output position of 
the servos for start up position and used for some multi input 
objects

\item {} 
\sphinxAtStartPar
\sphinxstylestrong{current\_output}(\sphinxstyleemphasis{{[}int{]}}) \textendash{} current output position of the 
servos to be used for increment mode or multi input objects

\item {} 
\sphinxAtStartPar
\sphinxstylestrong{maximum\_input}(\sphinxstyleemphasis{int}) \textendash{} maximum input value used for mapping 
inputs to outputs, 255 for analog and 1 for digital

\item {} 
\sphinxAtStartPar
\sphinxstylestrong{minimum\_input}(\sphinxstyleemphasis{int}) \textendash{} minimum input value used for mapping 
inputs to outputs, 0 for both analog and digital

\item {} 
\sphinxAtStartPar
\sphinxstylestrong{is\_inverted}(\sphinxstyleemphasis{{[}Boolean{]}}) \textendash{} whether to invert the output 
mapping

\item {} 
\sphinxAtStartPar
\sphinxstylestrong{control\_type}(\sphinxstyleemphasis{Enum ControlType}) \textendash{} mode on how to determine 
the output

\item {} 
\sphinxAtStartPar
\sphinxstylestrong{toggle\_state}(\sphinxstyleemphasis{Enum ToggleState}) \textendash{} used when control mode 
is set to TOGGLE to determine the current output state

\end{itemize}

\end{description}\end{quote}

\sphinxAtStartPar
…

\sphinxAtStartPar
\sphinxstylestrong{Methods}
\index{\_\_init\_\_() (DigitalOutputObject.DigitalOutputObject method)@\spxentry{\_\_init\_\_()}\spxextra{DigitalOutputObject.DigitalOutputObject method}}

\begin{fulllineitems}
\phantomsection\label{\detokenize{generic:DigitalOutputObject.DigitalOutputObject.__init__}}\pysiglinewithargsret{\sphinxbfcode{\sphinxupquote{\_\_init\_\_}}}{\emph{\DUrole{n}{name}}, \emph{\DUrole{n}{num\_outputs}}, \emph{\DUrole{n}{channels\_output}}}{}
\sphinxAtStartPar
Class constructor. Assigns the values passed in and initalizes remaining 
members to default values.
\begin{quote}\begin{description}
\item[{Parameters}] \leavevmode\begin{itemize}
\item {} 
\sphinxAtStartPar
\sphinxstyleliteralstrong{\sphinxupquote{name}} (\sphinxstyleliteralemphasis{\sphinxupquote{string}}) \textendash{} name of the output group represented by output object

\item {} 
\sphinxAtStartPar
\sphinxstyleliteralstrong{\sphinxupquote{num\_outputs}} (\sphinxstyleliteralemphasis{\sphinxupquote{int}}) \textendash{} number of output channels controlled by the output 
object

\item {} 
\sphinxAtStartPar
\sphinxstyleliteralstrong{\sphinxupquote{channels\_output}} (\sphinxstyleliteralemphasis{\sphinxupquote{{[}}}\sphinxstyleliteralemphasis{\sphinxupquote{int}}\sphinxstyleliteralemphasis{\sphinxupquote{{]}}}) \textendash{} list of the output channels used by the object

\end{itemize}

\end{description}\end{quote}

\end{fulllineitems}

\index{get\_default\_outputs() (DigitalOutputObject.DigitalOutputObject method)@\spxentry{get\_default\_outputs()}\spxextra{DigitalOutputObject.DigitalOutputObject method}}

\begin{fulllineitems}
\phantomsection\label{\detokenize{generic:DigitalOutputObject.DigitalOutputObject.get_default_outputs}}\pysiglinewithargsret{\sphinxbfcode{\sphinxupquote{get\_default\_outputs}}}{}{}
\sphinxAtStartPar
Returns the default output values for the object.
\begin{quote}\begin{description}
\item[{Returns}] \leavevmode
\sphinxAtStartPar
Two lists. The first list is the output channels and the second 
is the default output for those channels.

\item[{Return type}] \leavevmode
\sphinxAtStartPar
{[}{[}int{]}, {[}int{]}{]}

\end{description}\end{quote}

\end{fulllineitems}

\index{get\_num\_channels() (DigitalOutputObject.DigitalOutputObject method)@\spxentry{get\_num\_channels()}\spxextra{DigitalOutputObject.DigitalOutputObject method}}

\begin{fulllineitems}
\phantomsection\label{\detokenize{generic:DigitalOutputObject.DigitalOutputObject.get_num_channels}}\pysiglinewithargsret{\sphinxbfcode{\sphinxupquote{get\_num\_channels}}}{}{}
\sphinxAtStartPar
Returns the number of output channels for the object.
\begin{quote}\begin{description}
\item[{Returns}] \leavevmode
\sphinxAtStartPar
number of output channels the object sends output to

\item[{Return type}] \leavevmode
\sphinxAtStartPar
int

\end{description}\end{quote}

\end{fulllineitems}

\index{get\_output() (DigitalOutputObject.DigitalOutputObject method)@\spxentry{get\_output()}\spxextra{DigitalOutputObject.DigitalOutputObject method}}

\begin{fulllineitems}
\phantomsection\label{\detokenize{generic:DigitalOutputObject.DigitalOutputObject.get_output}}\pysiglinewithargsret{\sphinxbfcode{\sphinxupquote{get\_output}}}{\emph{\DUrole{n}{input\_name}}, \emph{\DUrole{n}{input\_value}}}{}
\sphinxAtStartPar
Calculate and returns the output based on the given input value and 
current control mode.
\begin{quote}\begin{description}
\item[{Parameters}] \leavevmode\begin{itemize}
\item {} 
\sphinxAtStartPar
\sphinxstyleliteralstrong{\sphinxupquote{input\_name}} (\sphinxstyleliteralemphasis{\sphinxupquote{str}}) \textendash{} name associated with the input, this object does not 
use this value

\item {} 
\sphinxAtStartPar
\sphinxstyleliteralstrong{\sphinxupquote{input\_value}} (\sphinxstyleliteralemphasis{\sphinxupquote{int}}) \textendash{} the input value from the PS4 controller

\end{itemize}

\item[{Returns}] \leavevmode
\sphinxAtStartPar

\sphinxAtStartPar
Two lists. The first list is the output channels and the second 
is output values for those channels (in units of quarter of milliseconds).

\sphinxAtStartPar
How the ouptut is calculated is based off of which control type the output 
object is set to. Direct will map the output directly based on the input 
and the set input and output ranges. Toggle will set the output between the 
max and the min output values and switch between these values whenever the 
input is released. Increment will increment the output value whenever input 
is given.


\item[{Return type}] \leavevmode
\sphinxAtStartPar
{[}{[}int{]}, {[}int{]}{]}

\end{description}\end{quote}

\end{fulllineitems}

\index{map\_values() (DigitalOutputObject.DigitalOutputObject method)@\spxentry{map\_values()}\spxextra{DigitalOutputObject.DigitalOutputObject method}}

\begin{fulllineitems}
\phantomsection\label{\detokenize{generic:DigitalOutputObject.DigitalOutputObject.map_values}}\pysiglinewithargsret{\sphinxbfcode{\sphinxupquote{map\_values}}}{\emph{\DUrole{n}{value}}, \emph{\DUrole{n}{input\_min}}, \emph{\DUrole{n}{input\_max}}, \emph{\DUrole{n}{out\_min}}, \emph{\DUrole{n}{out\_max}}}{}
\sphinxAtStartPar
Maps an input value to its output.
\begin{quote}\begin{description}
\item[{Parameters}] \leavevmode\begin{itemize}
\item {} 
\sphinxAtStartPar
\sphinxstyleliteralstrong{\sphinxupquote{value}} (\sphinxstyleliteralemphasis{\sphinxupquote{float}}) \textendash{} value of the input to map to an output value

\item {} 
\sphinxAtStartPar
\sphinxstyleliteralstrong{\sphinxupquote{input\_min}} (\sphinxstyleliteralemphasis{\sphinxupquote{float}}) \textendash{} minimun input value in input range

\item {} 
\sphinxAtStartPar
\sphinxstyleliteralstrong{\sphinxupquote{input\_max}} (\sphinxstyleliteralemphasis{\sphinxupquote{float}}) \textendash{} maximun input value in input range

\item {} 
\sphinxAtStartPar
\sphinxstyleliteralstrong{\sphinxupquote{out\_min}} (\sphinxstyleliteralemphasis{\sphinxupquote{int}}) \textendash{} minimun output value in output range

\item {} 
\sphinxAtStartPar
\sphinxstyleliteralstrong{\sphinxupquote{out\_max}} (\sphinxstyleliteralemphasis{\sphinxupquote{int}}) \textendash{} maximun output value in output range

\end{itemize}

\item[{Returns}] \leavevmode
\sphinxAtStartPar
pulse width to output for the given input value

\item[{Return type}] \leavevmode
\sphinxAtStartPar
int

\end{description}\end{quote}

\end{fulllineitems}

\index{set\_control\_direct() (DigitalOutputObject.DigitalOutputObject method)@\spxentry{set\_control\_direct()}\spxextra{DigitalOutputObject.DigitalOutputObject method}}

\begin{fulllineitems}
\phantomsection\label{\detokenize{generic:DigitalOutputObject.DigitalOutputObject.set_control_direct}}\pysiglinewithargsret{\sphinxbfcode{\sphinxupquote{set\_control\_direct}}}{}{}
\sphinxAtStartPar
Sets control mode to direct.

\end{fulllineitems}

\index{set\_control\_increment() (DigitalOutputObject.DigitalOutputObject method)@\spxentry{set\_control\_increment()}\spxextra{DigitalOutputObject.DigitalOutputObject method}}

\begin{fulllineitems}
\phantomsection\label{\detokenize{generic:DigitalOutputObject.DigitalOutputObject.set_control_increment}}\pysiglinewithargsret{\sphinxbfcode{\sphinxupquote{set\_control\_increment}}}{}{}
\sphinxAtStartPar
Sets control mode to increment.

\end{fulllineitems}

\index{set\_control\_toggle() (DigitalOutputObject.DigitalOutputObject method)@\spxentry{set\_control\_toggle()}\spxextra{DigitalOutputObject.DigitalOutputObject method}}

\begin{fulllineitems}
\phantomsection\label{\detokenize{generic:DigitalOutputObject.DigitalOutputObject.set_control_toggle}}\pysiglinewithargsret{\sphinxbfcode{\sphinxupquote{set\_control\_toggle}}}{}{}
\sphinxAtStartPar
Sets control mode to toggle.

\end{fulllineitems}

\index{set\_inversion() (DigitalOutputObject.DigitalOutputObject method)@\spxentry{set\_inversion()}\spxextra{DigitalOutputObject.DigitalOutputObject method}}

\begin{fulllineitems}
\phantomsection\label{\detokenize{generic:DigitalOutputObject.DigitalOutputObject.set_inversion}}\pysiglinewithargsret{\sphinxbfcode{\sphinxupquote{set\_inversion}}}{\emph{\DUrole{n}{is\_inverted}}}{}
\sphinxAtStartPar
Sets whether to invert the output signal or not.
\begin{quote}\begin{description}
\item[{Parameters}] \leavevmode
\sphinxAtStartPar
\sphinxstyleliteralstrong{\sphinxupquote{is\_inverted}} (\sphinxstyleliteralemphasis{\sphinxupquote{boolean}}) \textendash{} the state to set the attribute is\_inverted to

\end{description}\end{quote}

\end{fulllineitems}

\index{set\_outputs() (DigitalOutputObject.DigitalOutputObject method)@\spxentry{set\_outputs()}\spxextra{DigitalOutputObject.DigitalOutputObject method}}

\begin{fulllineitems}
\phantomsection\label{\detokenize{generic:DigitalOutputObject.DigitalOutputObject.set_outputs}}\pysiglinewithargsret{\sphinxbfcode{\sphinxupquote{set\_outputs}}}{\emph{\DUrole{n}{minimums\_output}}, \emph{\DUrole{n}{default\_output}}, \emph{\DUrole{n}{maximums\_output}}}{}
\sphinxAtStartPar
Sets which channels to output to and the minimun, default, and maximum 
pulse width for each of those channels.

\sphinxAtStartPar
Also sets current outputs to the same values as the default.
\begin{quote}\begin{description}
\item[{Parameters}] \leavevmode\begin{itemize}
\item {} 
\sphinxAtStartPar
\sphinxstyleliteralstrong{\sphinxupquote{minimums\_output}} (\sphinxstyleliteralemphasis{\sphinxupquote{{[}}}\sphinxstyleliteralemphasis{\sphinxupquote{int}}\sphinxstyleliteralemphasis{\sphinxupquote{{]}}}) \textendash{} minimum pulse width values for the corresponding 
servo channel

\item {} 
\sphinxAtStartPar
\sphinxstyleliteralstrong{\sphinxupquote{default\_output}} (\sphinxstyleliteralemphasis{\sphinxupquote{{[}}}\sphinxstyleliteralemphasis{\sphinxupquote{int}}\sphinxstyleliteralemphasis{\sphinxupquote{{]}}}) \textendash{} neutral pulse width values for the corresponding 
servo channel, also determines starting position

\item {} 
\sphinxAtStartPar
\sphinxstyleliteralstrong{\sphinxupquote{maximums\_output}} (\sphinxstyleliteralemphasis{\sphinxupquote{{[}}}\sphinxstyleliteralemphasis{\sphinxupquote{int}}\sphinxstyleliteralemphasis{\sphinxupquote{{]}}}) \textendash{} maximum pulse width values for the corresponding 
servo channel

\end{itemize}

\end{description}\end{quote}

\end{fulllineitems}


\end{fulllineitems}



\subsection{Analog Output Object}
\label{\detokenize{generic:module-AnalogOutputObject}}\label{\detokenize{generic:analog-output-object}}\index{module@\spxentry{module}!AnalogOutputObject@\spxentry{AnalogOutputObject}}\index{AnalogOutputObject@\spxentry{AnalogOutputObject}!module@\spxentry{module}}\index{AnalogOutputObject (class in AnalogOutputObject)@\spxentry{AnalogOutputObject}\spxextra{class in AnalogOutputObject}}

\begin{fulllineitems}
\phantomsection\label{\detokenize{generic:AnalogOutputObject.AnalogOutputObject}}\pysiglinewithargsret{\sphinxbfcode{\sphinxupquote{class }}\sphinxcode{\sphinxupquote{AnalogOutputObject.}}\sphinxbfcode{\sphinxupquote{AnalogOutputObject}}}{\emph{\DUrole{n}{name}}, \emph{\DUrole{n}{num\_outputs}}, \emph{\DUrole{n}{channels\_output}}}{}
\sphinxAtStartPar
A class to represent a an output object for an analog controller input.

\sphinxAtStartPar
It inherits attributes and methods from OutputObject.

\sphinxAtStartPar
…
\begin{quote}\begin{description}
\item[{Attributes}] \leavevmode\begin{itemize}
\item {} 
\sphinxAtStartPar
\sphinxstylestrong{name}(\sphinxstyleemphasis{str}) \textendash{} name of the servo group for the output

\item {} 
\sphinxAtStartPar
\sphinxstylestrong{num\_outputs}(\sphinxstyleemphasis{int}) \textendash{} total number of outputs 
calculated from the given input

\item {} 
\sphinxAtStartPar
\sphinxstylestrong{channels\_output}(\sphinxstyleemphasis{{[}int{]}}) \textendash{} channels corresponding to 
the servos controlled by the outputs

\item {} 
\sphinxAtStartPar
\sphinxstylestrong{maximums\_output}(\sphinxstyleemphasis{{[}int{]}}) \textendash{} maximum pulse width values 
for the corresponding servo channel

\item {} 
\sphinxAtStartPar
\sphinxstylestrong{minimums\_output}(\sphinxstyleemphasis{{[}int{]}}) \textendash{} minimum pulse width values 
for the corresponding servo channel

\item {} 
\sphinxAtStartPar
\sphinxstylestrong{default\_output}(\sphinxstyleemphasis{{[}int{]}}) \textendash{} default output position of 
the servos for start up position and used for some multi input 
objects

\item {} 
\sphinxAtStartPar
\sphinxstylestrong{current\_output}(\sphinxstyleemphasis{{[}int{]}}) \textendash{} current output position of the 
servos to be used for increment mode or multi input objects

\item {} 
\sphinxAtStartPar
\sphinxstylestrong{maximum\_input}(\sphinxstyleemphasis{int}) \textendash{} maximum input value used for mapping 
inputs to outputs, 255 for analog and 1 for digital

\item {} 
\sphinxAtStartPar
\sphinxstylestrong{minimum\_input}(\sphinxstyleemphasis{int}) \textendash{} minimum input value used for mapping 
inputs to outputs, 0 for both analog and digital

\item {} 
\sphinxAtStartPar
\sphinxstylestrong{is\_inverted}(\sphinxstyleemphasis{{[}Boolean{]}}) \textendash{} whether to invert the output 
mapping

\item {} 
\sphinxAtStartPar
\sphinxstylestrong{control\_type}(\sphinxstyleemphasis{Enum ControlType}) \textendash{} mode on how to determine 
the output

\item {} 
\sphinxAtStartPar
\sphinxstylestrong{toggle\_state}(\sphinxstyleemphasis{Enum ToggleState}) \textendash{} used when control mode 
is set to TOGGLE to determine the current output state

\end{itemize}

\end{description}\end{quote}

\sphinxAtStartPar
…

\sphinxAtStartPar
\sphinxstylestrong{Methods}
\index{\_\_init\_\_() (AnalogOutputObject.AnalogOutputObject method)@\spxentry{\_\_init\_\_()}\spxextra{AnalogOutputObject.AnalogOutputObject method}}

\begin{fulllineitems}
\phantomsection\label{\detokenize{generic:AnalogOutputObject.AnalogOutputObject.__init__}}\pysiglinewithargsret{\sphinxbfcode{\sphinxupquote{\_\_init\_\_}}}{\emph{\DUrole{n}{name}}, \emph{\DUrole{n}{num\_outputs}}, \emph{\DUrole{n}{channels\_output}}}{}
\sphinxAtStartPar
Class constructor. Assigns the values passed in and initalizes remaining 
members to default values.
\begin{quote}\begin{description}
\item[{Parameters}] \leavevmode\begin{itemize}
\item {} 
\sphinxAtStartPar
\sphinxstyleliteralstrong{\sphinxupquote{name}} (\sphinxstyleliteralemphasis{\sphinxupquote{string}}) \textendash{} name of the output group represented by output object

\item {} 
\sphinxAtStartPar
\sphinxstyleliteralstrong{\sphinxupquote{num\_outputs}} (\sphinxstyleliteralemphasis{\sphinxupquote{int}}) \textendash{} number of output channels controlled by the output 
object

\item {} 
\sphinxAtStartPar
\sphinxstyleliteralstrong{\sphinxupquote{channels\_output}} (\sphinxstyleliteralemphasis{\sphinxupquote{{[}}}\sphinxstyleliteralemphasis{\sphinxupquote{int}}\sphinxstyleliteralemphasis{\sphinxupquote{{]}}}) \textendash{} list of the output channels used by the object

\end{itemize}

\end{description}\end{quote}

\end{fulllineitems}

\index{get\_default\_outputs() (AnalogOutputObject.AnalogOutputObject method)@\spxentry{get\_default\_outputs()}\spxextra{AnalogOutputObject.AnalogOutputObject method}}

\begin{fulllineitems}
\phantomsection\label{\detokenize{generic:AnalogOutputObject.AnalogOutputObject.get_default_outputs}}\pysiglinewithargsret{\sphinxbfcode{\sphinxupquote{get\_default\_outputs}}}{}{}
\sphinxAtStartPar
Returns the default output values for the object.
\begin{quote}\begin{description}
\item[{Returns}] \leavevmode
\sphinxAtStartPar
Two lists. The first list is the output channels and the second 
is the default output for those channels.

\item[{Return type}] \leavevmode
\sphinxAtStartPar
{[}{[}int{]}, {[}int{]}{]}

\end{description}\end{quote}

\end{fulllineitems}

\index{get\_num\_channels() (AnalogOutputObject.AnalogOutputObject method)@\spxentry{get\_num\_channels()}\spxextra{AnalogOutputObject.AnalogOutputObject method}}

\begin{fulllineitems}
\phantomsection\label{\detokenize{generic:AnalogOutputObject.AnalogOutputObject.get_num_channels}}\pysiglinewithargsret{\sphinxbfcode{\sphinxupquote{get\_num\_channels}}}{}{}
\sphinxAtStartPar
Returns the number of output channels for the object.
\begin{quote}\begin{description}
\item[{Returns}] \leavevmode
\sphinxAtStartPar
number of output channels the object sends output to

\item[{Return type}] \leavevmode
\sphinxAtStartPar
int

\end{description}\end{quote}

\end{fulllineitems}

\index{get\_output() (AnalogOutputObject.AnalogOutputObject method)@\spxentry{get\_output()}\spxextra{AnalogOutputObject.AnalogOutputObject method}}

\begin{fulllineitems}
\phantomsection\label{\detokenize{generic:AnalogOutputObject.AnalogOutputObject.get_output}}\pysiglinewithargsret{\sphinxbfcode{\sphinxupquote{get\_output}}}{\emph{\DUrole{n}{input\_name}}, \emph{\DUrole{n}{input\_value}}}{}
\sphinxAtStartPar
Calculate and returns the output based on the given input value and 
current control mode.
\begin{quote}\begin{description}
\item[{Parameters}] \leavevmode\begin{itemize}
\item {} 
\sphinxAtStartPar
\sphinxstyleliteralstrong{\sphinxupquote{input\_name}} (\sphinxstyleliteralemphasis{\sphinxupquote{str}}) \textendash{} name associated with the input, this object does not 
use this value

\item {} 
\sphinxAtStartPar
\sphinxstyleliteralstrong{\sphinxupquote{input\_value}} (\sphinxstyleliteralemphasis{\sphinxupquote{int}}) \textendash{} the input value from the PS4 controller

\end{itemize}

\item[{Returns}] \leavevmode
\sphinxAtStartPar

\sphinxAtStartPar
Two lists. The first list is the output channels and the second 
is output values for those channels (in units of quarter of milliseconds).

\sphinxAtStartPar
How the ouptut is calculated is based off of which control type the output 
object is set to. Direct will map the output directly based on the input 
and the set input and output ranges. Toggle will set the output between the 
max and the min output values and switch between these values whenever the 
input is released. Increment will increment the output value whenever input 
is given.


\item[{Return type}] \leavevmode
\sphinxAtStartPar
{[}{[}int{]}, {[}int{]}{]}

\end{description}\end{quote}

\end{fulllineitems}

\index{map\_values() (AnalogOutputObject.AnalogOutputObject method)@\spxentry{map\_values()}\spxextra{AnalogOutputObject.AnalogOutputObject method}}

\begin{fulllineitems}
\phantomsection\label{\detokenize{generic:AnalogOutputObject.AnalogOutputObject.map_values}}\pysiglinewithargsret{\sphinxbfcode{\sphinxupquote{map\_values}}}{\emph{\DUrole{n}{value}}, \emph{\DUrole{n}{input\_min}}, \emph{\DUrole{n}{input\_max}}, \emph{\DUrole{n}{out\_min}}, \emph{\DUrole{n}{out\_max}}}{}
\sphinxAtStartPar
Maps an input value to its output.
\begin{quote}\begin{description}
\item[{Parameters}] \leavevmode\begin{itemize}
\item {} 
\sphinxAtStartPar
\sphinxstyleliteralstrong{\sphinxupquote{value}} (\sphinxstyleliteralemphasis{\sphinxupquote{float}}) \textendash{} value of the input to map to an output value

\item {} 
\sphinxAtStartPar
\sphinxstyleliteralstrong{\sphinxupquote{input\_min}} (\sphinxstyleliteralemphasis{\sphinxupquote{float}}) \textendash{} minimun input value in input range

\item {} 
\sphinxAtStartPar
\sphinxstyleliteralstrong{\sphinxupquote{input\_max}} (\sphinxstyleliteralemphasis{\sphinxupquote{float}}) \textendash{} maximun input value in input range

\item {} 
\sphinxAtStartPar
\sphinxstyleliteralstrong{\sphinxupquote{out\_min}} (\sphinxstyleliteralemphasis{\sphinxupquote{int}}) \textendash{} minimun output value in output range

\item {} 
\sphinxAtStartPar
\sphinxstyleliteralstrong{\sphinxupquote{out\_max}} (\sphinxstyleliteralemphasis{\sphinxupquote{int}}) \textendash{} maximun output value in output range

\end{itemize}

\item[{Returns}] \leavevmode
\sphinxAtStartPar
pulse width to output for the given input value

\item[{Return type}] \leavevmode
\sphinxAtStartPar
int

\end{description}\end{quote}

\end{fulllineitems}

\index{set\_control\_direct() (AnalogOutputObject.AnalogOutputObject method)@\spxentry{set\_control\_direct()}\spxextra{AnalogOutputObject.AnalogOutputObject method}}

\begin{fulllineitems}
\phantomsection\label{\detokenize{generic:AnalogOutputObject.AnalogOutputObject.set_control_direct}}\pysiglinewithargsret{\sphinxbfcode{\sphinxupquote{set\_control\_direct}}}{}{}
\sphinxAtStartPar
Sets control mode to direct.

\end{fulllineitems}

\index{set\_control\_increment() (AnalogOutputObject.AnalogOutputObject method)@\spxentry{set\_control\_increment()}\spxextra{AnalogOutputObject.AnalogOutputObject method}}

\begin{fulllineitems}
\phantomsection\label{\detokenize{generic:AnalogOutputObject.AnalogOutputObject.set_control_increment}}\pysiglinewithargsret{\sphinxbfcode{\sphinxupquote{set\_control\_increment}}}{}{}
\sphinxAtStartPar
Sets control mode to increment.

\end{fulllineitems}

\index{set\_control\_toggle() (AnalogOutputObject.AnalogOutputObject method)@\spxentry{set\_control\_toggle()}\spxextra{AnalogOutputObject.AnalogOutputObject method}}

\begin{fulllineitems}
\phantomsection\label{\detokenize{generic:AnalogOutputObject.AnalogOutputObject.set_control_toggle}}\pysiglinewithargsret{\sphinxbfcode{\sphinxupquote{set\_control\_toggle}}}{}{}
\sphinxAtStartPar
Sets control mode to toggle.

\end{fulllineitems}

\index{set\_inversion() (AnalogOutputObject.AnalogOutputObject method)@\spxentry{set\_inversion()}\spxextra{AnalogOutputObject.AnalogOutputObject method}}

\begin{fulllineitems}
\phantomsection\label{\detokenize{generic:AnalogOutputObject.AnalogOutputObject.set_inversion}}\pysiglinewithargsret{\sphinxbfcode{\sphinxupquote{set\_inversion}}}{\emph{\DUrole{n}{is\_inverted}}}{}
\sphinxAtStartPar
Sets whether to invert the output signal or not.
\begin{quote}\begin{description}
\item[{Parameters}] \leavevmode
\sphinxAtStartPar
\sphinxstyleliteralstrong{\sphinxupquote{is\_inverted}} (\sphinxstyleliteralemphasis{\sphinxupquote{boolean}}) \textendash{} the state to set the attribute is\_inverted to

\end{description}\end{quote}

\end{fulllineitems}

\index{set\_outputs() (AnalogOutputObject.AnalogOutputObject method)@\spxentry{set\_outputs()}\spxextra{AnalogOutputObject.AnalogOutputObject method}}

\begin{fulllineitems}
\phantomsection\label{\detokenize{generic:AnalogOutputObject.AnalogOutputObject.set_outputs}}\pysiglinewithargsret{\sphinxbfcode{\sphinxupquote{set\_outputs}}}{\emph{\DUrole{n}{minimums\_output}}, \emph{\DUrole{n}{default\_output}}, \emph{\DUrole{n}{maximums\_output}}}{}
\sphinxAtStartPar
Sets which channels to output to and the minimun, default, and maximum 
pulse width for each of those channels.

\sphinxAtStartPar
Also sets current outputs to the same values as the default.
\begin{quote}\begin{description}
\item[{Parameters}] \leavevmode\begin{itemize}
\item {} 
\sphinxAtStartPar
\sphinxstyleliteralstrong{\sphinxupquote{minimums\_output}} (\sphinxstyleliteralemphasis{\sphinxupquote{{[}}}\sphinxstyleliteralemphasis{\sphinxupquote{int}}\sphinxstyleliteralemphasis{\sphinxupquote{{]}}}) \textendash{} minimum pulse width values for the corresponding 
servo channel

\item {} 
\sphinxAtStartPar
\sphinxstyleliteralstrong{\sphinxupquote{default\_output}} (\sphinxstyleliteralemphasis{\sphinxupquote{{[}}}\sphinxstyleliteralemphasis{\sphinxupquote{int}}\sphinxstyleliteralemphasis{\sphinxupquote{{]}}}) \textendash{} neutral pulse width values for the corresponding 
servo channel, also determines starting position

\item {} 
\sphinxAtStartPar
\sphinxstyleliteralstrong{\sphinxupquote{maximums\_output}} (\sphinxstyleliteralemphasis{\sphinxupquote{{[}}}\sphinxstyleliteralemphasis{\sphinxupquote{int}}\sphinxstyleliteralemphasis{\sphinxupquote{{]}}}) \textendash{} maximum pulse width values for the corresponding 
servo channel

\end{itemize}

\end{description}\end{quote}

\end{fulllineitems}


\end{fulllineitems}



\subsection{Analog Mixer Output}
\label{\detokenize{generic:module-AnalogMixerOutput}}\label{\detokenize{generic:analog-mixer-output}}\index{module@\spxentry{module}!AnalogMixerOutput@\spxentry{AnalogMixerOutput}}\index{AnalogMixerOutput@\spxentry{AnalogMixerOutput}!module@\spxentry{module}}\index{AnalogMixerOutput (class in AnalogMixerOutput)@\spxentry{AnalogMixerOutput}\spxextra{class in AnalogMixerOutput}}

\begin{fulllineitems}
\phantomsection\label{\detokenize{generic:AnalogMixerOutput.AnalogMixerOutput}}\pysiglinewithargsret{\sphinxbfcode{\sphinxupquote{class }}\sphinxcode{\sphinxupquote{AnalogMixerOutput.}}\sphinxbfcode{\sphinxupquote{AnalogMixerOutput}}}{\emph{\DUrole{n}{name}}, \emph{\DUrole{n}{num\_outputs}}, \emph{\DUrole{n}{channels\_output}}, \emph{\DUrole{n}{names\_input}}}{}
\sphinxAtStartPar
A class to represent an output object that mixes two analog inputs.

\sphinxAtStartPar
It inherits attributes and methods from MultiInputOutputObject.

\sphinxAtStartPar
…
\begin{quote}\begin{description}
\item[{Attributes}] \leavevmode\begin{itemize}
\item {} 
\sphinxAtStartPar
\sphinxstylestrong{name}(\sphinxstyleemphasis{str}) \textendash{} name of the servo group for the output

\item {} 
\sphinxAtStartPar
\sphinxstylestrong{num\_outputs}(\sphinxstyleemphasis{int}) \textendash{} total number of outputs 
calculated from the given input

\item {} 
\sphinxAtStartPar
\sphinxstylestrong{channels\_output}(\sphinxstyleemphasis{{[}int{]}}) \textendash{} channels corresponding to 
the servos controlled by the outputs

\item {} 
\sphinxAtStartPar
\sphinxstylestrong{maximums\_output}(\sphinxstyleemphasis{{[}int{]}}) \textendash{} maximum pulse width values 
for the corresponding servo channel

\item {} 
\sphinxAtStartPar
\sphinxstylestrong{minimums\_output}(\sphinxstyleemphasis{{[}int{]}}) \textendash{} minimum pulse width values 
for the corresponding servo channel

\item {} 
\sphinxAtStartPar
\sphinxstylestrong{default\_output}(\sphinxstyleemphasis{{[}int{]}}) \textendash{} default output position of 
the servos for start up position and used for some multi input 
objects

\item {} 
\sphinxAtStartPar
\sphinxstylestrong{current\_output}(\sphinxstyleemphasis{{[}int{]}}) \textendash{} current output position of the 
servos to be used for increment mode or multi input objects

\item {} 
\sphinxAtStartPar
\sphinxstylestrong{maximum\_input}(\sphinxstyleemphasis{int}) \textendash{} maximum input value used for mapping 
inputs to outputs, 255 for analog and 1 for digital

\item {} 
\sphinxAtStartPar
\sphinxstylestrong{minimum\_input}(\sphinxstyleemphasis{int}) \textendash{} minimum input value used for mapping 
inputs to outputs, 0 for both analog and digital

\item {} 
\sphinxAtStartPar
\sphinxstylestrong{is\_inverted}(\sphinxstyleemphasis{{[}Boolean{]}}) \textendash{} whether to invert the output 
mapping

\item {} 
\sphinxAtStartPar
\sphinxstylestrong{control\_type}(\sphinxstyleemphasis{Enum ControlType}) \textendash{} mode on how to determine 
the output

\item {} 
\sphinxAtStartPar
\sphinxstylestrong{toggle\_state}(\sphinxstyleemphasis{Enum ToggleState}) \textendash{} used when control mode 
is set to TOGGLE to determine the current output state

\item {} 
\sphinxAtStartPar
\sphinxstylestrong{names\_input}(\sphinxstyleemphasis{{[}str{]}}) \textendash{} list of the input names used by 
the object, this allows the object to know which input value to 
update

\item {} 
\sphinxAtStartPar
\sphinxstylestrong{num\_inputs}(\sphinxstyleemphasis{int}) \textendash{} number of inputs the object uses

\item {} 
\sphinxAtStartPar
\sphinxstylestrong{current\_input}(\sphinxstyleemphasis{{[}int{]}}) \textendash{} since input is given one at a 
time, this list keeps track of previous inputs

\item {} 
\sphinxAtStartPar
\sphinxstylestrong{out\_raw\_min}(\sphinxstyleemphasis{int}) \textendash{} minimum value of an intermediate 
value used to calculate output

\item {} 
\sphinxAtStartPar
\sphinxstylestrong{out\_raw\_max}(\sphinxstyleemphasis{int}) \textendash{} maximum value of an intermediate 
value used to calculate output

\item {} 
\sphinxAtStartPar
\sphinxstylestrong{raw\_output}(\sphinxstyleemphasis{{[}int{]}}) \textendash{} list to store the calculated 
intermediate values that will be mapped to final output values

\end{itemize}

\end{description}\end{quote}

\sphinxAtStartPar
…

\sphinxAtStartPar
\sphinxstylestrong{Methods}
\index{\_\_init\_\_() (AnalogMixerOutput.AnalogMixerOutput method)@\spxentry{\_\_init\_\_()}\spxextra{AnalogMixerOutput.AnalogMixerOutput method}}

\begin{fulllineitems}
\phantomsection\label{\detokenize{generic:AnalogMixerOutput.AnalogMixerOutput.__init__}}\pysiglinewithargsret{\sphinxbfcode{\sphinxupquote{\_\_init\_\_}}}{\emph{\DUrole{n}{name}}, \emph{\DUrole{n}{num\_outputs}}, \emph{\DUrole{n}{channels\_output}}, \emph{\DUrole{n}{names\_input}}}{}
\sphinxAtStartPar
Class constructor. Assigns the values passed in and initalizes remaining 
members to default values.
\begin{quote}\begin{description}
\item[{Parameters}] \leavevmode\begin{itemize}
\item {} 
\sphinxAtStartPar
\sphinxstyleliteralstrong{\sphinxupquote{name}} (\sphinxstyleliteralemphasis{\sphinxupquote{string}}) \textendash{} name of the output group represented by output object

\item {} 
\sphinxAtStartPar
\sphinxstyleliteralstrong{\sphinxupquote{num\_outputs}} (\sphinxstyleliteralemphasis{\sphinxupquote{int}}) \textendash{} number of output channels controlled by the output 
object

\item {} 
\sphinxAtStartPar
\sphinxstyleliteralstrong{\sphinxupquote{channels\_output}} (\sphinxstyleliteralemphasis{\sphinxupquote{{[}}}\sphinxstyleliteralemphasis{\sphinxupquote{int}}\sphinxstyleliteralemphasis{\sphinxupquote{{]}}}) \textendash{} list of the output channels used by the object

\item {} 
\sphinxAtStartPar
\sphinxstyleliteralstrong{\sphinxupquote{names\_input}} (\sphinxstyleliteralemphasis{\sphinxupquote{{[}}}\sphinxstyleliteralemphasis{\sphinxupquote{str}}\sphinxstyleliteralemphasis{\sphinxupquote{{]}}}) \textendash{} the names of the associated controller inputs with 
the object

\end{itemize}

\end{description}\end{quote}

\end{fulllineitems}

\index{get\_default\_outputs() (AnalogMixerOutput.AnalogMixerOutput method)@\spxentry{get\_default\_outputs()}\spxextra{AnalogMixerOutput.AnalogMixerOutput method}}

\begin{fulllineitems}
\phantomsection\label{\detokenize{generic:AnalogMixerOutput.AnalogMixerOutput.get_default_outputs}}\pysiglinewithargsret{\sphinxbfcode{\sphinxupquote{get\_default\_outputs}}}{}{}
\sphinxAtStartPar
Returns the default output values for the object.
\begin{quote}\begin{description}
\item[{Returns}] \leavevmode
\sphinxAtStartPar
Two lists. The first list is the output channels and the second 
is the default output for those channels.

\item[{Return type}] \leavevmode
\sphinxAtStartPar
{[}{[}int{]}, {[}int{]}{]}

\end{description}\end{quote}

\end{fulllineitems}

\index{get\_num\_channels() (AnalogMixerOutput.AnalogMixerOutput method)@\spxentry{get\_num\_channels()}\spxextra{AnalogMixerOutput.AnalogMixerOutput method}}

\begin{fulllineitems}
\phantomsection\label{\detokenize{generic:AnalogMixerOutput.AnalogMixerOutput.get_num_channels}}\pysiglinewithargsret{\sphinxbfcode{\sphinxupquote{get\_num\_channels}}}{}{}
\sphinxAtStartPar
Returns the number of output channels for the object.
\begin{quote}\begin{description}
\item[{Returns}] \leavevmode
\sphinxAtStartPar
number of output channels the object sends output to

\item[{Return type}] \leavevmode
\sphinxAtStartPar
int

\end{description}\end{quote}

\end{fulllineitems}

\index{get\_output() (AnalogMixerOutput.AnalogMixerOutput method)@\spxentry{get\_output()}\spxextra{AnalogMixerOutput.AnalogMixerOutput method}}

\begin{fulllineitems}
\phantomsection\label{\detokenize{generic:AnalogMixerOutput.AnalogMixerOutput.get_output}}\pysiglinewithargsret{\sphinxbfcode{\sphinxupquote{get\_output}}}{\emph{\DUrole{n}{input\_name}}, \emph{\DUrole{n}{input\_value}}}{}
\sphinxAtStartPar
Calculate and returns the output based on the given input value and 
current control mode.
\begin{quote}\begin{description}
\item[{Parameters}] \leavevmode\begin{itemize}
\item {} 
\sphinxAtStartPar
\sphinxstyleliteralstrong{\sphinxupquote{input\_name}} (\sphinxstyleliteralemphasis{\sphinxupquote{str}}) \textendash{} name associated with the input, this object does not 
use this value

\item {} 
\sphinxAtStartPar
\sphinxstyleliteralstrong{\sphinxupquote{input\_value}} (\sphinxstyleliteralemphasis{\sphinxupquote{int}}) \textendash{} the input value from the PS4 controller

\end{itemize}

\item[{Returns}] \leavevmode
\sphinxAtStartPar

\sphinxAtStartPar
Two lists. The first list is the output channels and the second 
is output values for those channels (in units of quarter of milliseconds).

\sphinxAtStartPar
How the ouptut is calculated is based off of which control type the output 
object is set to. Direct will map the output directly based on the input 
and the set input and output ranges. Toggle will set the output between the 
max and the min output values and switch between these values whenever the 
input is released. Increment will increment the output value whenever input 
is given.


\item[{Return type}] \leavevmode
\sphinxAtStartPar
{[}{[}int{]}, {[}int{]}{]}

\end{description}\end{quote}

\end{fulllineitems}

\index{map\_values() (AnalogMixerOutput.AnalogMixerOutput method)@\spxentry{map\_values()}\spxextra{AnalogMixerOutput.AnalogMixerOutput method}}

\begin{fulllineitems}
\phantomsection\label{\detokenize{generic:AnalogMixerOutput.AnalogMixerOutput.map_values}}\pysiglinewithargsret{\sphinxbfcode{\sphinxupquote{map\_values}}}{\emph{\DUrole{n}{value}}, \emph{\DUrole{n}{input\_min}}, \emph{\DUrole{n}{input\_max}}, \emph{\DUrole{n}{out\_min}}, \emph{\DUrole{n}{out\_max}}}{}
\sphinxAtStartPar
Maps an input value to its output.
\begin{quote}\begin{description}
\item[{Parameters}] \leavevmode\begin{itemize}
\item {} 
\sphinxAtStartPar
\sphinxstyleliteralstrong{\sphinxupquote{value}} (\sphinxstyleliteralemphasis{\sphinxupquote{float}}) \textendash{} value of the input to map to an output value

\item {} 
\sphinxAtStartPar
\sphinxstyleliteralstrong{\sphinxupquote{input\_min}} (\sphinxstyleliteralemphasis{\sphinxupquote{float}}) \textendash{} minimun input value in input range

\item {} 
\sphinxAtStartPar
\sphinxstyleliteralstrong{\sphinxupquote{input\_max}} (\sphinxstyleliteralemphasis{\sphinxupquote{float}}) \textendash{} maximun input value in input range

\item {} 
\sphinxAtStartPar
\sphinxstyleliteralstrong{\sphinxupquote{out\_min}} (\sphinxstyleliteralemphasis{\sphinxupquote{int}}) \textendash{} minimun output value in output range

\item {} 
\sphinxAtStartPar
\sphinxstyleliteralstrong{\sphinxupquote{out\_max}} (\sphinxstyleliteralemphasis{\sphinxupquote{int}}) \textendash{} maximun output value in output range

\end{itemize}

\item[{Returns}] \leavevmode
\sphinxAtStartPar
pulse width to output for the given input value

\item[{Return type}] \leavevmode
\sphinxAtStartPar
int

\end{description}\end{quote}

\end{fulllineitems}

\index{set\_control\_direct() (AnalogMixerOutput.AnalogMixerOutput method)@\spxentry{set\_control\_direct()}\spxextra{AnalogMixerOutput.AnalogMixerOutput method}}

\begin{fulllineitems}
\phantomsection\label{\detokenize{generic:AnalogMixerOutput.AnalogMixerOutput.set_control_direct}}\pysiglinewithargsret{\sphinxbfcode{\sphinxupquote{set\_control\_direct}}}{}{}
\sphinxAtStartPar
Sets control mode to direct.

\end{fulllineitems}

\index{set\_control\_increment() (AnalogMixerOutput.AnalogMixerOutput method)@\spxentry{set\_control\_increment()}\spxextra{AnalogMixerOutput.AnalogMixerOutput method}}

\begin{fulllineitems}
\phantomsection\label{\detokenize{generic:AnalogMixerOutput.AnalogMixerOutput.set_control_increment}}\pysiglinewithargsret{\sphinxbfcode{\sphinxupquote{set\_control\_increment}}}{}{}
\sphinxAtStartPar
Sets control mode to increment.

\end{fulllineitems}

\index{set\_control\_toggle() (AnalogMixerOutput.AnalogMixerOutput method)@\spxentry{set\_control\_toggle()}\spxextra{AnalogMixerOutput.AnalogMixerOutput method}}

\begin{fulllineitems}
\phantomsection\label{\detokenize{generic:AnalogMixerOutput.AnalogMixerOutput.set_control_toggle}}\pysiglinewithargsret{\sphinxbfcode{\sphinxupquote{set\_control\_toggle}}}{}{}
\sphinxAtStartPar
Sets control mode to toggle.

\end{fulllineitems}

\index{set\_inversion() (AnalogMixerOutput.AnalogMixerOutput method)@\spxentry{set\_inversion()}\spxextra{AnalogMixerOutput.AnalogMixerOutput method}}

\begin{fulllineitems}
\phantomsection\label{\detokenize{generic:AnalogMixerOutput.AnalogMixerOutput.set_inversion}}\pysiglinewithargsret{\sphinxbfcode{\sphinxupquote{set\_inversion}}}{\emph{\DUrole{n}{is\_inverted}}}{}
\sphinxAtStartPar
Sets whether to invert the output signal or not.
\begin{quote}\begin{description}
\item[{Parameters}] \leavevmode
\sphinxAtStartPar
\sphinxstyleliteralstrong{\sphinxupquote{is\_inverted}} (\sphinxstyleliteralemphasis{\sphinxupquote{boolean}}) \textendash{} the state to set the attribute is\_inverted to

\end{description}\end{quote}

\end{fulllineitems}

\index{set\_outputs() (AnalogMixerOutput.AnalogMixerOutput method)@\spxentry{set\_outputs()}\spxextra{AnalogMixerOutput.AnalogMixerOutput method}}

\begin{fulllineitems}
\phantomsection\label{\detokenize{generic:AnalogMixerOutput.AnalogMixerOutput.set_outputs}}\pysiglinewithargsret{\sphinxbfcode{\sphinxupquote{set\_outputs}}}{\emph{\DUrole{n}{minimums\_output}}, \emph{\DUrole{n}{default\_output}}, \emph{\DUrole{n}{maximums\_output}}}{}
\sphinxAtStartPar
Sets which channels to output to and the minimun, default, and maximum 
pulse width for each of those channels.

\sphinxAtStartPar
Also sets current outputs to the same values as the default.
\begin{quote}\begin{description}
\item[{Parameters}] \leavevmode\begin{itemize}
\item {} 
\sphinxAtStartPar
\sphinxstyleliteralstrong{\sphinxupquote{minimums\_output}} (\sphinxstyleliteralemphasis{\sphinxupquote{{[}}}\sphinxstyleliteralemphasis{\sphinxupquote{int}}\sphinxstyleliteralemphasis{\sphinxupquote{{]}}}) \textendash{} minimum pulse width values for the corresponding 
servo channel

\item {} 
\sphinxAtStartPar
\sphinxstyleliteralstrong{\sphinxupquote{default\_output}} (\sphinxstyleliteralemphasis{\sphinxupquote{{[}}}\sphinxstyleliteralemphasis{\sphinxupquote{int}}\sphinxstyleliteralemphasis{\sphinxupquote{{]}}}) \textendash{} neutral pulse width values for the corresponding 
servo channel, also determines starting position

\item {} 
\sphinxAtStartPar
\sphinxstyleliteralstrong{\sphinxupquote{maximums\_output}} (\sphinxstyleliteralemphasis{\sphinxupquote{{[}}}\sphinxstyleliteralemphasis{\sphinxupquote{int}}\sphinxstyleliteralemphasis{\sphinxupquote{{]}}}) \textendash{} maximum pulse width values for the corresponding 
servo channel

\end{itemize}

\end{description}\end{quote}

\end{fulllineitems}


\end{fulllineitems}



\section{Specific Objects}
\label{\detokenize{specific:specific-objects}}\label{\detokenize{specific::doc}}

\subsection{Ear Output}
\label{\detokenize{specific:module-EarOutput}}\label{\detokenize{specific:ear-output}}\index{module@\spxentry{module}!EarOutput@\spxentry{EarOutput}}\index{EarOutput@\spxentry{EarOutput}!module@\spxentry{module}}\index{EarOutput (class in EarOutput)@\spxentry{EarOutput}\spxextra{class in EarOutput}}

\begin{fulllineitems}
\phantomsection\label{\detokenize{specific:EarOutput.EarOutput}}\pysiglinewithargsret{\sphinxbfcode{\sphinxupquote{class }}\sphinxcode{\sphinxupquote{EarOutput.}}\sphinxbfcode{\sphinxupquote{EarOutput}}}{\emph{\DUrole{n}{name}}, \emph{\DUrole{n}{num\_outputs}}, \emph{\DUrole{n}{channels\_output}}, \emph{\DUrole{n}{names\_input}}}{}
\sphinxAtStartPar
A class to represent the output object for the left or right ear that 
will take multiple inputs.

\sphinxAtStartPar
It inherits attributes and methods from MultiInputOutputObject.

\sphinxAtStartPar
…
\begin{quote}\begin{description}
\item[{Attributes}] \leavevmode\begin{itemize}
\item {} 
\sphinxAtStartPar
\sphinxstylestrong{name}(\sphinxstyleemphasis{str}) \textendash{} name of the servo group for the output

\item {} 
\sphinxAtStartPar
\sphinxstylestrong{num\_outputs}(\sphinxstyleemphasis{int}) \textendash{} total number of outputs 
calculated from the given input

\item {} 
\sphinxAtStartPar
\sphinxstylestrong{channels\_output}(\sphinxstyleemphasis{{[}int{]}}) \textendash{} channels corresponding to 
the servos controlled by the outputs

\item {} 
\sphinxAtStartPar
\sphinxstylestrong{maximums\_output}(\sphinxstyleemphasis{{[}int{]}}) \textendash{} maximum pulse width values 
for the corresponding servo channel

\item {} 
\sphinxAtStartPar
\sphinxstylestrong{minimums\_output}(\sphinxstyleemphasis{{[}int{]}}) \textendash{} minimum pulse width values 
for the corresponding servo channel

\item {} 
\sphinxAtStartPar
\sphinxstylestrong{default\_output}(\sphinxstyleemphasis{{[}int{]}}) \textendash{} default output position of 
the servos for start up position and used for some multi input 
objects

\item {} 
\sphinxAtStartPar
\sphinxstylestrong{current\_output}(\sphinxstyleemphasis{{[}int{]}}) \textendash{} current output position of the 
servos to be used for increment mode or multi input objects

\item {} 
\sphinxAtStartPar
\sphinxstylestrong{maximum\_input}(\sphinxstyleemphasis{int}) \textendash{} maximum input value used for mapping 
inputs to outputs, 255 for analog and 1 for digital

\item {} 
\sphinxAtStartPar
\sphinxstylestrong{minimum\_input}(\sphinxstyleemphasis{int}) \textendash{} minimum input value used for mapping 
inputs to outputs, 0 for both analog and digital

\item {} 
\sphinxAtStartPar
\sphinxstylestrong{is\_inverted}(\sphinxstyleemphasis{{[}Boolean{]}}) \textendash{} whether to invert the output 
mapping

\item {} 
\sphinxAtStartPar
\sphinxstylestrong{control\_type}(\sphinxstyleemphasis{Enum ControlType}) \textendash{} mode on how to determine 
the output

\item {} 
\sphinxAtStartPar
\sphinxstylestrong{toggle\_state}(\sphinxstyleemphasis{Enum ToggleState}) \textendash{} used when control mode 
is set to TOGGLE to determine the current output state

\item {} 
\sphinxAtStartPar
\sphinxstylestrong{names\_input}(\sphinxstyleemphasis{{[}str{]}}) \textendash{} list of the input names used by 
the object, this allows the object to know which input value to 
update

\item {} 
\sphinxAtStartPar
\sphinxstylestrong{num\_inputs}(\sphinxstyleemphasis{int}) \textendash{} number of inputs the object uses

\item {} 
\sphinxAtStartPar
\sphinxstylestrong{current\_input}(\sphinxstyleemphasis{{[}int{]}}) \textendash{} since input is given one at a 
time, this list keeps track of previous inputs

\item {} 
\sphinxAtStartPar
\sphinxstylestrong{out\_raw\_min}(\sphinxstyleemphasis{int}) \textendash{} minimum value of an intermediate 
value used to calculate output

\item {} 
\sphinxAtStartPar
\sphinxstylestrong{out\_raw\_max}(\sphinxstyleemphasis{int}) \textendash{} maximum value of an intermediate 
value used to calculate output

\item {} 
\sphinxAtStartPar
\sphinxstylestrong{raw\_output}(\sphinxstyleemphasis{{[}int{]}}) \textendash{} list to store the calculated 
intermediate values that will be mapped to final output values

\end{itemize}

\end{description}\end{quote}

\sphinxAtStartPar
…

\sphinxAtStartPar
\sphinxstylestrong{Methods}
\index{\_\_init\_\_() (EarOutput.EarOutput method)@\spxentry{\_\_init\_\_()}\spxextra{EarOutput.EarOutput method}}

\begin{fulllineitems}
\phantomsection\label{\detokenize{specific:EarOutput.EarOutput.__init__}}\pysiglinewithargsret{\sphinxbfcode{\sphinxupquote{\_\_init\_\_}}}{\emph{\DUrole{n}{name}}, \emph{\DUrole{n}{num\_outputs}}, \emph{\DUrole{n}{channels\_output}}, \emph{\DUrole{n}{names\_input}}}{}
\sphinxAtStartPar
Class constructor. Assigns the values passed in and initalizes remaining 
members to default values.
\begin{quote}\begin{description}
\item[{Parameters}] \leavevmode\begin{itemize}
\item {} 
\sphinxAtStartPar
\sphinxstyleliteralstrong{\sphinxupquote{name}} (\sphinxstyleliteralemphasis{\sphinxupquote{string}}) \textendash{} name of the output group represented by output object

\item {} 
\sphinxAtStartPar
\sphinxstyleliteralstrong{\sphinxupquote{num\_outputs}} (\sphinxstyleliteralemphasis{\sphinxupquote{int}}) \textendash{} number of output channels controlled by the output 
object

\item {} 
\sphinxAtStartPar
\sphinxstyleliteralstrong{\sphinxupquote{channels\_output}} (\sphinxstyleliteralemphasis{\sphinxupquote{{[}}}\sphinxstyleliteralemphasis{\sphinxupquote{int}}\sphinxstyleliteralemphasis{\sphinxupquote{{]}}}) \textendash{} list of the output channels used by the object

\item {} 
\sphinxAtStartPar
\sphinxstyleliteralstrong{\sphinxupquote{names\_input}} (\sphinxstyleliteralemphasis{\sphinxupquote{{[}}}\sphinxstyleliteralemphasis{\sphinxupquote{str}}\sphinxstyleliteralemphasis{\sphinxupquote{{]}}}) \textendash{} the names of the associated controller inputs with 
the object

\end{itemize}

\end{description}\end{quote}

\end{fulllineitems}

\index{get\_default\_outputs() (EarOutput.EarOutput method)@\spxentry{get\_default\_outputs()}\spxextra{EarOutput.EarOutput method}}

\begin{fulllineitems}
\phantomsection\label{\detokenize{specific:EarOutput.EarOutput.get_default_outputs}}\pysiglinewithargsret{\sphinxbfcode{\sphinxupquote{get\_default\_outputs}}}{}{}
\sphinxAtStartPar
Returns the default output values for the object.
\begin{quote}\begin{description}
\item[{Returns}] \leavevmode
\sphinxAtStartPar
Two lists. The first list is the output channels and the second 
is the default output for those channels.

\item[{Return type}] \leavevmode
\sphinxAtStartPar
{[}{[}int{]}, {[}int{]}{]}

\end{description}\end{quote}

\end{fulllineitems}

\index{get\_num\_channels() (EarOutput.EarOutput method)@\spxentry{get\_num\_channels()}\spxextra{EarOutput.EarOutput method}}

\begin{fulllineitems}
\phantomsection\label{\detokenize{specific:EarOutput.EarOutput.get_num_channels}}\pysiglinewithargsret{\sphinxbfcode{\sphinxupquote{get\_num\_channels}}}{}{}
\sphinxAtStartPar
Returns the number of output channels for the object.
\begin{quote}\begin{description}
\item[{Returns}] \leavevmode
\sphinxAtStartPar
number of output channels the object sends output to

\item[{Return type}] \leavevmode
\sphinxAtStartPar
int

\end{description}\end{quote}

\end{fulllineitems}

\index{get\_output() (EarOutput.EarOutput method)@\spxentry{get\_output()}\spxextra{EarOutput.EarOutput method}}

\begin{fulllineitems}
\phantomsection\label{\detokenize{specific:EarOutput.EarOutput.get_output}}\pysiglinewithargsret{\sphinxbfcode{\sphinxupquote{get\_output}}}{\emph{\DUrole{n}{input\_name}}, \emph{\DUrole{n}{input\_value}}}{}
\sphinxAtStartPar
Calculate and returns the output based on the given input value and 
current control mode.
\begin{quote}\begin{description}
\item[{Parameters}] \leavevmode\begin{itemize}
\item {} 
\sphinxAtStartPar
\sphinxstyleliteralstrong{\sphinxupquote{input\_name}} (\sphinxstyleliteralemphasis{\sphinxupquote{str}}) \textendash{} name associated with the input, this object does not 
use this value

\item {} 
\sphinxAtStartPar
\sphinxstyleliteralstrong{\sphinxupquote{input\_value}} (\sphinxstyleliteralemphasis{\sphinxupquote{int}}) \textendash{} the input value from the PS4 controller

\end{itemize}

\item[{Returns}] \leavevmode
\sphinxAtStartPar

\sphinxAtStartPar
Two lists. The first list is the output channels and the second 
is output values for those channels (in units of quarter of milliseconds).

\sphinxAtStartPar
How the ouptut is calculated is based off of which control type the output 
object is set to. Direct will map the output directly based on the input 
and the set input and output ranges. Toggle will set the output between the 
max and the min output values and switch between these values whenever the 
input is released. Increment will increment the output value whenever input 
is given.


\item[{Return type}] \leavevmode
\sphinxAtStartPar
{[}{[}int{]}, {[}int{]}{]}

\end{description}\end{quote}

\end{fulllineitems}

\index{map\_values() (EarOutput.EarOutput method)@\spxentry{map\_values()}\spxextra{EarOutput.EarOutput method}}

\begin{fulllineitems}
\phantomsection\label{\detokenize{specific:EarOutput.EarOutput.map_values}}\pysiglinewithargsret{\sphinxbfcode{\sphinxupquote{map\_values}}}{\emph{\DUrole{n}{value}}, \emph{\DUrole{n}{input\_min}}, \emph{\DUrole{n}{input\_max}}, \emph{\DUrole{n}{out\_min}}, \emph{\DUrole{n}{out\_max}}}{}
\sphinxAtStartPar
Maps an input value to its output.
\begin{quote}\begin{description}
\item[{Parameters}] \leavevmode\begin{itemize}
\item {} 
\sphinxAtStartPar
\sphinxstyleliteralstrong{\sphinxupquote{value}} (\sphinxstyleliteralemphasis{\sphinxupquote{float}}) \textendash{} value of the input to map to an output value

\item {} 
\sphinxAtStartPar
\sphinxstyleliteralstrong{\sphinxupquote{input\_min}} (\sphinxstyleliteralemphasis{\sphinxupquote{float}}) \textendash{} minimun input value in input range

\item {} 
\sphinxAtStartPar
\sphinxstyleliteralstrong{\sphinxupquote{input\_max}} (\sphinxstyleliteralemphasis{\sphinxupquote{float}}) \textendash{} maximun input value in input range

\item {} 
\sphinxAtStartPar
\sphinxstyleliteralstrong{\sphinxupquote{out\_min}} (\sphinxstyleliteralemphasis{\sphinxupquote{int}}) \textendash{} minimun output value in output range

\item {} 
\sphinxAtStartPar
\sphinxstyleliteralstrong{\sphinxupquote{out\_max}} (\sphinxstyleliteralemphasis{\sphinxupquote{int}}) \textendash{} maximun output value in output range

\end{itemize}

\item[{Returns}] \leavevmode
\sphinxAtStartPar
pulse width to output for the given input value

\item[{Return type}] \leavevmode
\sphinxAtStartPar
int

\end{description}\end{quote}

\end{fulllineitems}

\index{set\_control\_direct() (EarOutput.EarOutput method)@\spxentry{set\_control\_direct()}\spxextra{EarOutput.EarOutput method}}

\begin{fulllineitems}
\phantomsection\label{\detokenize{specific:EarOutput.EarOutput.set_control_direct}}\pysiglinewithargsret{\sphinxbfcode{\sphinxupquote{set\_control\_direct}}}{}{}
\sphinxAtStartPar
Sets control mode to direct.

\end{fulllineitems}

\index{set\_control\_increment() (EarOutput.EarOutput method)@\spxentry{set\_control\_increment()}\spxextra{EarOutput.EarOutput method}}

\begin{fulllineitems}
\phantomsection\label{\detokenize{specific:EarOutput.EarOutput.set_control_increment}}\pysiglinewithargsret{\sphinxbfcode{\sphinxupquote{set\_control\_increment}}}{}{}
\sphinxAtStartPar
Sets control mode to increment.

\end{fulllineitems}

\index{set\_control\_toggle() (EarOutput.EarOutput method)@\spxentry{set\_control\_toggle()}\spxextra{EarOutput.EarOutput method}}

\begin{fulllineitems}
\phantomsection\label{\detokenize{specific:EarOutput.EarOutput.set_control_toggle}}\pysiglinewithargsret{\sphinxbfcode{\sphinxupquote{set\_control\_toggle}}}{}{}
\sphinxAtStartPar
Sets control mode to toggle.

\end{fulllineitems}

\index{set\_inversion() (EarOutput.EarOutput method)@\spxentry{set\_inversion()}\spxextra{EarOutput.EarOutput method}}

\begin{fulllineitems}
\phantomsection\label{\detokenize{specific:EarOutput.EarOutput.set_inversion}}\pysiglinewithargsret{\sphinxbfcode{\sphinxupquote{set\_inversion}}}{\emph{\DUrole{n}{is\_inverted}}}{}
\sphinxAtStartPar
Sets whether to invert the output signal or not.
\begin{quote}\begin{description}
\item[{Parameters}] \leavevmode
\sphinxAtStartPar
\sphinxstyleliteralstrong{\sphinxupquote{is\_inverted}} (\sphinxstyleliteralemphasis{\sphinxupquote{boolean}}) \textendash{} the state to set the attribute is\_inverted to

\end{description}\end{quote}

\end{fulllineitems}

\index{set\_outputs() (EarOutput.EarOutput method)@\spxentry{set\_outputs()}\spxextra{EarOutput.EarOutput method}}

\begin{fulllineitems}
\phantomsection\label{\detokenize{specific:EarOutput.EarOutput.set_outputs}}\pysiglinewithargsret{\sphinxbfcode{\sphinxupquote{set\_outputs}}}{\emph{\DUrole{n}{minimums\_output}}, \emph{\DUrole{n}{default\_output}}, \emph{\DUrole{n}{maximums\_output}}}{}
\sphinxAtStartPar
Sets which channels to output to and the minimun, default, and maximum 
pulse width for each of those channels.

\sphinxAtStartPar
Also sets current outputs to the same values as the default.
\begin{quote}\begin{description}
\item[{Parameters}] \leavevmode\begin{itemize}
\item {} 
\sphinxAtStartPar
\sphinxstyleliteralstrong{\sphinxupquote{minimums\_output}} (\sphinxstyleliteralemphasis{\sphinxupquote{{[}}}\sphinxstyleliteralemphasis{\sphinxupquote{int}}\sphinxstyleliteralemphasis{\sphinxupquote{{]}}}) \textendash{} minimum pulse width values for the corresponding 
servo channel

\item {} 
\sphinxAtStartPar
\sphinxstyleliteralstrong{\sphinxupquote{default\_output}} (\sphinxstyleliteralemphasis{\sphinxupquote{{[}}}\sphinxstyleliteralemphasis{\sphinxupquote{int}}\sphinxstyleliteralemphasis{\sphinxupquote{{]}}}) \textendash{} neutral pulse width values for the corresponding 
servo channel, also determines starting position

\item {} 
\sphinxAtStartPar
\sphinxstyleliteralstrong{\sphinxupquote{maximums\_output}} (\sphinxstyleliteralemphasis{\sphinxupquote{{[}}}\sphinxstyleliteralemphasis{\sphinxupquote{int}}\sphinxstyleliteralemphasis{\sphinxupquote{{]}}}) \textendash{} maximum pulse width values for the corresponding 
servo channel

\end{itemize}

\end{description}\end{quote}

\end{fulllineitems}


\end{fulllineitems}



\subsection{Eyebrows Output}
\label{\detokenize{specific:module-EyebrowsOutput}}\label{\detokenize{specific:eyebrows-output}}\index{module@\spxentry{module}!EyebrowsOutput@\spxentry{EyebrowsOutput}}\index{EyebrowsOutput@\spxentry{EyebrowsOutput}!module@\spxentry{module}}\index{EyebrowsOutput (class in EyebrowsOutput)@\spxentry{EyebrowsOutput}\spxextra{class in EyebrowsOutput}}

\begin{fulllineitems}
\phantomsection\label{\detokenize{specific:EyebrowsOutput.EyebrowsOutput}}\pysiglinewithargsret{\sphinxbfcode{\sphinxupquote{class }}\sphinxcode{\sphinxupquote{EyebrowsOutput.}}\sphinxbfcode{\sphinxupquote{EyebrowsOutput}}}{\emph{\DUrole{n}{name}}, \emph{\DUrole{n}{num\_outputs}}, \emph{\DUrole{n}{channels\_output}}, \emph{\DUrole{n}{names\_input}}}{}
\sphinxAtStartPar
A class to represent the output object for the eyebrows that will 
take multiple inputs.

\sphinxAtStartPar
It inherits attributes and methods from MultiInputOutputObject.

\sphinxAtStartPar
…
\begin{quote}\begin{description}
\item[{Attributes}] \leavevmode\begin{itemize}
\item {} 
\sphinxAtStartPar
\sphinxstylestrong{name}(\sphinxstyleemphasis{str}) \textendash{} name of the servo group for the output

\item {} 
\sphinxAtStartPar
\sphinxstylestrong{num\_outputs}(\sphinxstyleemphasis{int}) \textendash{} total number of outputs 
calculated from the given input

\item {} 
\sphinxAtStartPar
\sphinxstylestrong{channels\_output}(\sphinxstyleemphasis{{[}int{]}}) \textendash{} channels corresponding to 
the servos controlled by the outputs

\item {} 
\sphinxAtStartPar
\sphinxstylestrong{maximums\_output}(\sphinxstyleemphasis{{[}int{]}}) \textendash{} maximum pulse width values 
for the corresponding servo channel

\item {} 
\sphinxAtStartPar
\sphinxstylestrong{minimums\_output}(\sphinxstyleemphasis{{[}int{]}}) \textendash{} minimum pulse width values 
for the corresponding servo channel

\item {} 
\sphinxAtStartPar
\sphinxstylestrong{default\_output}(\sphinxstyleemphasis{{[}int{]}}) \textendash{} default output position of 
the servos for start up position and used for some multi input 
objects

\item {} 
\sphinxAtStartPar
\sphinxstylestrong{current\_output}(\sphinxstyleemphasis{{[}int{]}}) \textendash{} current output position of the 
servos to be used for increment mode or multi input objects

\item {} 
\sphinxAtStartPar
\sphinxstylestrong{maximum\_input}(\sphinxstyleemphasis{int}) \textendash{} maximum input value used for mapping 
inputs to outputs, 255 for analog and 1 for digital

\item {} 
\sphinxAtStartPar
\sphinxstylestrong{minimum\_input}(\sphinxstyleemphasis{int}) \textendash{} minimum input value used for mapping 
inputs to outputs, 0 for both analog and digital

\item {} 
\sphinxAtStartPar
\sphinxstylestrong{is\_inverted}(\sphinxstyleemphasis{{[}Boolean{]}}) \textendash{} whether to invert the output 
mapping

\item {} 
\sphinxAtStartPar
\sphinxstylestrong{control\_type}(\sphinxstyleemphasis{Enum ControlType}) \textendash{} mode on how to determine 
the output

\item {} 
\sphinxAtStartPar
\sphinxstylestrong{toggle\_state}(\sphinxstyleemphasis{Enum ToggleState}) \textendash{} used when control mode 
is set to TOGGLE to determine the current output state

\item {} 
\sphinxAtStartPar
\sphinxstylestrong{names\_input}(\sphinxstyleemphasis{{[}str{]}}) \textendash{} list of the input names used by 
the object, this allows the object to know which input value to 
update

\item {} 
\sphinxAtStartPar
\sphinxstylestrong{num\_inputs}(\sphinxstyleemphasis{int}) \textendash{} number of inputs the object uses

\item {} 
\sphinxAtStartPar
\sphinxstylestrong{current\_input}(\sphinxstyleemphasis{{[}int{]}}) \textendash{} since input is given one at a 
time, this list keeps track of previous inputs

\item {} 
\sphinxAtStartPar
\sphinxstylestrong{out\_raw\_min}(\sphinxstyleemphasis{int}) \textendash{} minimum value of an intermediate 
value used to calculate output

\item {} 
\sphinxAtStartPar
\sphinxstylestrong{out\_raw\_max}(\sphinxstyleemphasis{int}) \textendash{} maximum value of an intermediate 
value used to calculate output

\item {} 
\sphinxAtStartPar
\sphinxstylestrong{raw\_output}(\sphinxstyleemphasis{{[}int{]}}) \textendash{} list to store the calculated 
intermediate values that will be mapped to final output values

\end{itemize}

\end{description}\end{quote}

\sphinxAtStartPar
…

\sphinxAtStartPar
\sphinxstylestrong{Methods}
\index{\_\_init\_\_() (EyebrowsOutput.EyebrowsOutput method)@\spxentry{\_\_init\_\_()}\spxextra{EyebrowsOutput.EyebrowsOutput method}}

\begin{fulllineitems}
\phantomsection\label{\detokenize{specific:EyebrowsOutput.EyebrowsOutput.__init__}}\pysiglinewithargsret{\sphinxbfcode{\sphinxupquote{\_\_init\_\_}}}{\emph{\DUrole{n}{name}}, \emph{\DUrole{n}{num\_outputs}}, \emph{\DUrole{n}{channels\_output}}, \emph{\DUrole{n}{names\_input}}}{}
\sphinxAtStartPar
Class constructor. Assigns the values passed in and initalizes remaining 
members to default values.
\begin{quote}\begin{description}
\item[{Parameters}] \leavevmode\begin{itemize}
\item {} 
\sphinxAtStartPar
\sphinxstyleliteralstrong{\sphinxupquote{name}} (\sphinxstyleliteralemphasis{\sphinxupquote{string}}) \textendash{} name of the output group represented by output object

\item {} 
\sphinxAtStartPar
\sphinxstyleliteralstrong{\sphinxupquote{num\_outputs}} (\sphinxstyleliteralemphasis{\sphinxupquote{int}}) \textendash{} number of output channels controlled by the output 
object

\item {} 
\sphinxAtStartPar
\sphinxstyleliteralstrong{\sphinxupquote{channels\_output}} (\sphinxstyleliteralemphasis{\sphinxupquote{{[}}}\sphinxstyleliteralemphasis{\sphinxupquote{int}}\sphinxstyleliteralemphasis{\sphinxupquote{{]}}}) \textendash{} list of the output channels used by the object

\item {} 
\sphinxAtStartPar
\sphinxstyleliteralstrong{\sphinxupquote{names\_input}} (\sphinxstyleliteralemphasis{\sphinxupquote{{[}}}\sphinxstyleliteralemphasis{\sphinxupquote{str}}\sphinxstyleliteralemphasis{\sphinxupquote{{]}}}) \textendash{} the names of the associated controller inputs with 
the object

\end{itemize}

\end{description}\end{quote}

\end{fulllineitems}

\index{get\_default\_outputs() (EyebrowsOutput.EyebrowsOutput method)@\spxentry{get\_default\_outputs()}\spxextra{EyebrowsOutput.EyebrowsOutput method}}

\begin{fulllineitems}
\phantomsection\label{\detokenize{specific:EyebrowsOutput.EyebrowsOutput.get_default_outputs}}\pysiglinewithargsret{\sphinxbfcode{\sphinxupquote{get\_default\_outputs}}}{}{}
\sphinxAtStartPar
Returns the default output values for the object.
\begin{quote}\begin{description}
\item[{Returns}] \leavevmode
\sphinxAtStartPar
Two lists. The first list is the output channels and the second 
is the default output for those channels.

\item[{Return type}] \leavevmode
\sphinxAtStartPar
{[}{[}int{]}, {[}int{]}{]}

\end{description}\end{quote}

\end{fulllineitems}

\index{get\_num\_channels() (EyebrowsOutput.EyebrowsOutput method)@\spxentry{get\_num\_channels()}\spxextra{EyebrowsOutput.EyebrowsOutput method}}

\begin{fulllineitems}
\phantomsection\label{\detokenize{specific:EyebrowsOutput.EyebrowsOutput.get_num_channels}}\pysiglinewithargsret{\sphinxbfcode{\sphinxupquote{get\_num\_channels}}}{}{}
\sphinxAtStartPar
Returns the number of output channels for the object.
\begin{quote}\begin{description}
\item[{Returns}] \leavevmode
\sphinxAtStartPar
number of output channels the object sends output to

\item[{Return type}] \leavevmode
\sphinxAtStartPar
int

\end{description}\end{quote}

\end{fulllineitems}

\index{get\_output() (EyebrowsOutput.EyebrowsOutput method)@\spxentry{get\_output()}\spxextra{EyebrowsOutput.EyebrowsOutput method}}

\begin{fulllineitems}
\phantomsection\label{\detokenize{specific:EyebrowsOutput.EyebrowsOutput.get_output}}\pysiglinewithargsret{\sphinxbfcode{\sphinxupquote{get\_output}}}{\emph{\DUrole{n}{input\_name}}, \emph{\DUrole{n}{input\_value}}}{}
\sphinxAtStartPar
Calculate and returns the output based on the given input value and 
current control mode.
\begin{quote}\begin{description}
\item[{Parameters}] \leavevmode\begin{itemize}
\item {} 
\sphinxAtStartPar
\sphinxstyleliteralstrong{\sphinxupquote{input\_name}} (\sphinxstyleliteralemphasis{\sphinxupquote{str}}) \textendash{} name associated with the input, this object does not 
use this value

\item {} 
\sphinxAtStartPar
\sphinxstyleliteralstrong{\sphinxupquote{input\_value}} (\sphinxstyleliteralemphasis{\sphinxupquote{int}}) \textendash{} the input value from the PS4 controller

\end{itemize}

\item[{Returns}] \leavevmode
\sphinxAtStartPar

\sphinxAtStartPar
Two lists. The first list is the output channels and the second 
is output values for those channels (in units of quarter of milliseconds).

\sphinxAtStartPar
How the ouptut is calculated is based off of which control type the output 
object is set to. Direct will map the output directly based on the input 
and the set input and output ranges. Toggle will set the output between the 
max and the min output values and switch between these values whenever the 
input is released. Increment will increment the output value whenever input 
is given.


\item[{Return type}] \leavevmode
\sphinxAtStartPar
{[}{[}int{]}, {[}int{]}{]}

\end{description}\end{quote}

\end{fulllineitems}

\index{map\_values() (EyebrowsOutput.EyebrowsOutput method)@\spxentry{map\_values()}\spxextra{EyebrowsOutput.EyebrowsOutput method}}

\begin{fulllineitems}
\phantomsection\label{\detokenize{specific:EyebrowsOutput.EyebrowsOutput.map_values}}\pysiglinewithargsret{\sphinxbfcode{\sphinxupquote{map\_values}}}{\emph{\DUrole{n}{value}}, \emph{\DUrole{n}{input\_min}}, \emph{\DUrole{n}{input\_max}}, \emph{\DUrole{n}{out\_min}}, \emph{\DUrole{n}{out\_max}}}{}
\sphinxAtStartPar
Maps an input value to its output.
\begin{quote}\begin{description}
\item[{Parameters}] \leavevmode\begin{itemize}
\item {} 
\sphinxAtStartPar
\sphinxstyleliteralstrong{\sphinxupquote{value}} (\sphinxstyleliteralemphasis{\sphinxupquote{float}}) \textendash{} value of the input to map to an output value

\item {} 
\sphinxAtStartPar
\sphinxstyleliteralstrong{\sphinxupquote{input\_min}} (\sphinxstyleliteralemphasis{\sphinxupquote{float}}) \textendash{} minimun input value in input range

\item {} 
\sphinxAtStartPar
\sphinxstyleliteralstrong{\sphinxupquote{input\_max}} (\sphinxstyleliteralemphasis{\sphinxupquote{float}}) \textendash{} maximun input value in input range

\item {} 
\sphinxAtStartPar
\sphinxstyleliteralstrong{\sphinxupquote{out\_min}} (\sphinxstyleliteralemphasis{\sphinxupquote{int}}) \textendash{} minimun output value in output range

\item {} 
\sphinxAtStartPar
\sphinxstyleliteralstrong{\sphinxupquote{out\_max}} (\sphinxstyleliteralemphasis{\sphinxupquote{int}}) \textendash{} maximun output value in output range

\end{itemize}

\item[{Returns}] \leavevmode
\sphinxAtStartPar
pulse width to output for the given input value

\item[{Return type}] \leavevmode
\sphinxAtStartPar
int

\end{description}\end{quote}

\end{fulllineitems}

\index{set\_control\_direct() (EyebrowsOutput.EyebrowsOutput method)@\spxentry{set\_control\_direct()}\spxextra{EyebrowsOutput.EyebrowsOutput method}}

\begin{fulllineitems}
\phantomsection\label{\detokenize{specific:EyebrowsOutput.EyebrowsOutput.set_control_direct}}\pysiglinewithargsret{\sphinxbfcode{\sphinxupquote{set\_control\_direct}}}{}{}
\sphinxAtStartPar
Sets control mode to direct.

\end{fulllineitems}

\index{set\_control\_increment() (EyebrowsOutput.EyebrowsOutput method)@\spxentry{set\_control\_increment()}\spxextra{EyebrowsOutput.EyebrowsOutput method}}

\begin{fulllineitems}
\phantomsection\label{\detokenize{specific:EyebrowsOutput.EyebrowsOutput.set_control_increment}}\pysiglinewithargsret{\sphinxbfcode{\sphinxupquote{set\_control\_increment}}}{}{}
\sphinxAtStartPar
Sets control mode to increment.

\end{fulllineitems}

\index{set\_control\_toggle() (EyebrowsOutput.EyebrowsOutput method)@\spxentry{set\_control\_toggle()}\spxextra{EyebrowsOutput.EyebrowsOutput method}}

\begin{fulllineitems}
\phantomsection\label{\detokenize{specific:EyebrowsOutput.EyebrowsOutput.set_control_toggle}}\pysiglinewithargsret{\sphinxbfcode{\sphinxupquote{set\_control\_toggle}}}{}{}
\sphinxAtStartPar
Sets control mode to toggle.

\end{fulllineitems}

\index{set\_inversion() (EyebrowsOutput.EyebrowsOutput method)@\spxentry{set\_inversion()}\spxextra{EyebrowsOutput.EyebrowsOutput method}}

\begin{fulllineitems}
\phantomsection\label{\detokenize{specific:EyebrowsOutput.EyebrowsOutput.set_inversion}}\pysiglinewithargsret{\sphinxbfcode{\sphinxupquote{set\_inversion}}}{\emph{\DUrole{n}{is\_inverted}}}{}
\sphinxAtStartPar
Sets whether to invert the output signal or not.
\begin{quote}\begin{description}
\item[{Parameters}] \leavevmode
\sphinxAtStartPar
\sphinxstyleliteralstrong{\sphinxupquote{is\_inverted}} (\sphinxstyleliteralemphasis{\sphinxupquote{boolean}}) \textendash{} the state to set the attribute is\_inverted to

\end{description}\end{quote}

\end{fulllineitems}

\index{set\_outputs() (EyebrowsOutput.EyebrowsOutput method)@\spxentry{set\_outputs()}\spxextra{EyebrowsOutput.EyebrowsOutput method}}

\begin{fulllineitems}
\phantomsection\label{\detokenize{specific:EyebrowsOutput.EyebrowsOutput.set_outputs}}\pysiglinewithargsret{\sphinxbfcode{\sphinxupquote{set\_outputs}}}{\emph{\DUrole{n}{minimums\_output}}, \emph{\DUrole{n}{default\_output}}, \emph{\DUrole{n}{maximums\_output}}}{}
\sphinxAtStartPar
Sets which channels to output to and the minimun, default, and maximum 
pulse width for each of those channels.

\sphinxAtStartPar
Also sets current outputs to the same values as the default.
\begin{quote}\begin{description}
\item[{Parameters}] \leavevmode\begin{itemize}
\item {} 
\sphinxAtStartPar
\sphinxstyleliteralstrong{\sphinxupquote{minimums\_output}} (\sphinxstyleliteralemphasis{\sphinxupquote{{[}}}\sphinxstyleliteralemphasis{\sphinxupquote{int}}\sphinxstyleliteralemphasis{\sphinxupquote{{]}}}) \textendash{} minimum pulse width values for the corresponding 
servo channel

\item {} 
\sphinxAtStartPar
\sphinxstyleliteralstrong{\sphinxupquote{default\_output}} (\sphinxstyleliteralemphasis{\sphinxupquote{{[}}}\sphinxstyleliteralemphasis{\sphinxupquote{int}}\sphinxstyleliteralemphasis{\sphinxupquote{{]}}}) \textendash{} neutral pulse width values for the corresponding 
servo channel, also determines starting position

\item {} 
\sphinxAtStartPar
\sphinxstyleliteralstrong{\sphinxupquote{maximums\_output}} (\sphinxstyleliteralemphasis{\sphinxupquote{{[}}}\sphinxstyleliteralemphasis{\sphinxupquote{int}}\sphinxstyleliteralemphasis{\sphinxupquote{{]}}}) \textendash{} maximum pulse width values for the corresponding 
servo channel

\end{itemize}

\end{description}\end{quote}

\end{fulllineitems}


\end{fulllineitems}



\subsection{Side Lip Output}
\label{\detokenize{specific:module-SideLipOutput}}\label{\detokenize{specific:side-lip-output}}\index{module@\spxentry{module}!SideLipOutput@\spxentry{SideLipOutput}}\index{SideLipOutput@\spxentry{SideLipOutput}!module@\spxentry{module}}\index{SideLipOutput (class in SideLipOutput)@\spxentry{SideLipOutput}\spxextra{class in SideLipOutput}}

\begin{fulllineitems}
\phantomsection\label{\detokenize{specific:SideLipOutput.SideLipOutput}}\pysiglinewithargsret{\sphinxbfcode{\sphinxupquote{class }}\sphinxcode{\sphinxupquote{SideLipOutput.}}\sphinxbfcode{\sphinxupquote{SideLipOutput}}}{\emph{\DUrole{n}{name}}, \emph{\DUrole{n}{num\_outputs}}, \emph{\DUrole{n}{channels\_output}}, \emph{\DUrole{n}{names\_input}}}{}
\sphinxAtStartPar
A class to represent the output object for the left or right lip 
that will take multiple inputs.

\sphinxAtStartPar
It inherits attributes and methods from MultiInputOutputObject.

\sphinxAtStartPar
…
\begin{quote}\begin{description}
\item[{Attributes}] \leavevmode\begin{itemize}
\item {} 
\sphinxAtStartPar
\sphinxstylestrong{name}(\sphinxstyleemphasis{str}) \textendash{} name of the servo group for the output

\item {} 
\sphinxAtStartPar
\sphinxstylestrong{num\_outputs}(\sphinxstyleemphasis{int}) \textendash{} total number of outputs 
calculated from the given input

\item {} 
\sphinxAtStartPar
\sphinxstylestrong{channels\_output}(\sphinxstyleemphasis{{[}int{]}}) \textendash{} channels corresponding to 
the servos controlled by the outputs

\item {} 
\sphinxAtStartPar
\sphinxstylestrong{maximums\_output}(\sphinxstyleemphasis{{[}int{]}}) \textendash{} maximum pulse width values 
for the corresponding servo channel

\item {} 
\sphinxAtStartPar
\sphinxstylestrong{minimums\_output}(\sphinxstyleemphasis{{[}int{]}}) \textendash{} minimum pulse width values 
for the corresponding servo channel

\item {} 
\sphinxAtStartPar
\sphinxstylestrong{default\_output}(\sphinxstyleemphasis{{[}int{]}}) \textendash{} default output position of 
the servos for start up position and used for some multi input 
objects

\item {} 
\sphinxAtStartPar
\sphinxstylestrong{current\_output}(\sphinxstyleemphasis{{[}int{]}}) \textendash{} current output position of the 
servos to be used for increment mode or multi input objects

\item {} 
\sphinxAtStartPar
\sphinxstylestrong{maximum\_input}(\sphinxstyleemphasis{int}) \textendash{} maximum input value used for mapping 
inputs to outputs, 255 for analog and 1 for digital

\item {} 
\sphinxAtStartPar
\sphinxstylestrong{minimum\_input}(\sphinxstyleemphasis{int}) \textendash{} minimum input value used for mapping 
inputs to outputs, 0 for both analog and digital

\item {} 
\sphinxAtStartPar
\sphinxstylestrong{is\_inverted}(\sphinxstyleemphasis{{[}Boolean{]}}) \textendash{} whether to invert the output 
mapping

\item {} 
\sphinxAtStartPar
\sphinxstylestrong{control\_type}(\sphinxstyleemphasis{Enum ControlType}) \textendash{} mode on how to determine 
the output

\item {} 
\sphinxAtStartPar
\sphinxstylestrong{toggle\_state}(\sphinxstyleemphasis{Enum ToggleState}) \textendash{} used when control mode 
is set to TOGGLE to determine the current output state

\item {} 
\sphinxAtStartPar
\sphinxstylestrong{names\_input}(\sphinxstyleemphasis{{[}str{]}}) \textendash{} list of the input names used by 
the object, this allows the object to know which input value to 
update

\item {} 
\sphinxAtStartPar
\sphinxstylestrong{num\_inputs}(\sphinxstyleemphasis{int}) \textendash{} number of inputs the object uses

\item {} 
\sphinxAtStartPar
\sphinxstylestrong{current\_input}(\sphinxstyleemphasis{{[}int{]}}) \textendash{} since input is given one at a 
time, this list keeps track of previous inputs

\item {} 
\sphinxAtStartPar
\sphinxstylestrong{out\_raw\_min}(\sphinxstyleemphasis{int}) \textendash{} minimum value of an intermediate 
value used to calculate output

\item {} 
\sphinxAtStartPar
\sphinxstylestrong{out\_raw\_max}(\sphinxstyleemphasis{int}) \textendash{} maximum value of an intermediate 
value used to calculate output

\item {} 
\sphinxAtStartPar
\sphinxstylestrong{raw\_output}(\sphinxstyleemphasis{{[}int{]}}) \textendash{} list to store the calculated 
intermediate values that will be mapped to final output values

\end{itemize}

\end{description}\end{quote}

\sphinxAtStartPar
…

\sphinxAtStartPar
\sphinxstylestrong{Methods}
\index{\_\_init\_\_() (SideLipOutput.SideLipOutput method)@\spxentry{\_\_init\_\_()}\spxextra{SideLipOutput.SideLipOutput method}}

\begin{fulllineitems}
\phantomsection\label{\detokenize{specific:SideLipOutput.SideLipOutput.__init__}}\pysiglinewithargsret{\sphinxbfcode{\sphinxupquote{\_\_init\_\_}}}{\emph{\DUrole{n}{name}}, \emph{\DUrole{n}{num\_outputs}}, \emph{\DUrole{n}{channels\_output}}, \emph{\DUrole{n}{names\_input}}}{}
\sphinxAtStartPar
Class constructor. Assigns the values passed in and initalizes remaining 
members to default values.
\begin{quote}\begin{description}
\item[{Parameters}] \leavevmode\begin{itemize}
\item {} 
\sphinxAtStartPar
\sphinxstyleliteralstrong{\sphinxupquote{name}} (\sphinxstyleliteralemphasis{\sphinxupquote{string}}) \textendash{} name of the output group represented by output object

\item {} 
\sphinxAtStartPar
\sphinxstyleliteralstrong{\sphinxupquote{num\_outputs}} (\sphinxstyleliteralemphasis{\sphinxupquote{int}}) \textendash{} number of output channels controlled by the output 
object

\item {} 
\sphinxAtStartPar
\sphinxstyleliteralstrong{\sphinxupquote{channels\_output}} (\sphinxstyleliteralemphasis{\sphinxupquote{{[}}}\sphinxstyleliteralemphasis{\sphinxupquote{int}}\sphinxstyleliteralemphasis{\sphinxupquote{{]}}}) \textendash{} list of the output channels used by the object

\item {} 
\sphinxAtStartPar
\sphinxstyleliteralstrong{\sphinxupquote{names\_input}} (\sphinxstyleliteralemphasis{\sphinxupquote{{[}}}\sphinxstyleliteralemphasis{\sphinxupquote{str}}\sphinxstyleliteralemphasis{\sphinxupquote{{]}}}) \textendash{} the names of the associated controller inputs with 
the object

\end{itemize}

\end{description}\end{quote}

\end{fulllineitems}

\index{get\_default\_outputs() (SideLipOutput.SideLipOutput method)@\spxentry{get\_default\_outputs()}\spxextra{SideLipOutput.SideLipOutput method}}

\begin{fulllineitems}
\phantomsection\label{\detokenize{specific:SideLipOutput.SideLipOutput.get_default_outputs}}\pysiglinewithargsret{\sphinxbfcode{\sphinxupquote{get\_default\_outputs}}}{}{}
\sphinxAtStartPar
Returns the default output values for the object.
\begin{quote}\begin{description}
\item[{Returns}] \leavevmode
\sphinxAtStartPar
Two lists. The first list is the output channels and the second 
is the default output for those channels.

\item[{Return type}] \leavevmode
\sphinxAtStartPar
{[}{[}int{]}, {[}int{]}{]}

\end{description}\end{quote}

\end{fulllineitems}

\index{get\_num\_channels() (SideLipOutput.SideLipOutput method)@\spxentry{get\_num\_channels()}\spxextra{SideLipOutput.SideLipOutput method}}

\begin{fulllineitems}
\phantomsection\label{\detokenize{specific:SideLipOutput.SideLipOutput.get_num_channels}}\pysiglinewithargsret{\sphinxbfcode{\sphinxupquote{get\_num\_channels}}}{}{}
\sphinxAtStartPar
Returns the number of output channels for the object.
\begin{quote}\begin{description}
\item[{Returns}] \leavevmode
\sphinxAtStartPar
number of output channels the object sends output to

\item[{Return type}] \leavevmode
\sphinxAtStartPar
int

\end{description}\end{quote}

\end{fulllineitems}

\index{get\_output() (SideLipOutput.SideLipOutput method)@\spxentry{get\_output()}\spxextra{SideLipOutput.SideLipOutput method}}

\begin{fulllineitems}
\phantomsection\label{\detokenize{specific:SideLipOutput.SideLipOutput.get_output}}\pysiglinewithargsret{\sphinxbfcode{\sphinxupquote{get\_output}}}{\emph{\DUrole{n}{input\_name}}, \emph{\DUrole{n}{input\_value}}}{}
\sphinxAtStartPar
Calculate and returns the output based on the given input value and 
current control mode.
\begin{quote}\begin{description}
\item[{Parameters}] \leavevmode\begin{itemize}
\item {} 
\sphinxAtStartPar
\sphinxstyleliteralstrong{\sphinxupquote{input\_name}} (\sphinxstyleliteralemphasis{\sphinxupquote{str}}) \textendash{} name associated with the input, this object does not 
use this value

\item {} 
\sphinxAtStartPar
\sphinxstyleliteralstrong{\sphinxupquote{input\_value}} (\sphinxstyleliteralemphasis{\sphinxupquote{int}}) \textendash{} the input value from the PS4 controller

\end{itemize}

\item[{Returns}] \leavevmode
\sphinxAtStartPar

\sphinxAtStartPar
Two lists. The first list is the output channels and the second 
is output values for those channels (in units of quarter of milliseconds).

\sphinxAtStartPar
How the ouptut is calculated is based off of which control type the output 
object is set to. Direct will map the output directly based on the input 
and the set input and output ranges. Toggle will set the output between the 
max and the min output values and switch between these values whenever the 
input is released. Increment will increment the output value whenever input 
is given.


\item[{Return type}] \leavevmode
\sphinxAtStartPar
{[}{[}int{]}, {[}int{]}{]}

\end{description}\end{quote}

\end{fulllineitems}

\index{map\_values() (SideLipOutput.SideLipOutput method)@\spxentry{map\_values()}\spxextra{SideLipOutput.SideLipOutput method}}

\begin{fulllineitems}
\phantomsection\label{\detokenize{specific:SideLipOutput.SideLipOutput.map_values}}\pysiglinewithargsret{\sphinxbfcode{\sphinxupquote{map\_values}}}{\emph{\DUrole{n}{value}}, \emph{\DUrole{n}{input\_min}}, \emph{\DUrole{n}{input\_max}}, \emph{\DUrole{n}{out\_min}}, \emph{\DUrole{n}{out\_max}}}{}
\sphinxAtStartPar
Maps an input value to its output.
\begin{quote}\begin{description}
\item[{Parameters}] \leavevmode\begin{itemize}
\item {} 
\sphinxAtStartPar
\sphinxstyleliteralstrong{\sphinxupquote{value}} (\sphinxstyleliteralemphasis{\sphinxupquote{float}}) \textendash{} value of the input to map to an output value

\item {} 
\sphinxAtStartPar
\sphinxstyleliteralstrong{\sphinxupquote{input\_min}} (\sphinxstyleliteralemphasis{\sphinxupquote{float}}) \textendash{} minimun input value in input range

\item {} 
\sphinxAtStartPar
\sphinxstyleliteralstrong{\sphinxupquote{input\_max}} (\sphinxstyleliteralemphasis{\sphinxupquote{float}}) \textendash{} maximun input value in input range

\item {} 
\sphinxAtStartPar
\sphinxstyleliteralstrong{\sphinxupquote{out\_min}} (\sphinxstyleliteralemphasis{\sphinxupquote{int}}) \textendash{} minimun output value in output range

\item {} 
\sphinxAtStartPar
\sphinxstyleliteralstrong{\sphinxupquote{out\_max}} (\sphinxstyleliteralemphasis{\sphinxupquote{int}}) \textendash{} maximun output value in output range

\end{itemize}

\item[{Returns}] \leavevmode
\sphinxAtStartPar
pulse width to output for the given input value

\item[{Return type}] \leavevmode
\sphinxAtStartPar
int

\end{description}\end{quote}

\end{fulllineitems}

\index{set\_control\_direct() (SideLipOutput.SideLipOutput method)@\spxentry{set\_control\_direct()}\spxextra{SideLipOutput.SideLipOutput method}}

\begin{fulllineitems}
\phantomsection\label{\detokenize{specific:SideLipOutput.SideLipOutput.set_control_direct}}\pysiglinewithargsret{\sphinxbfcode{\sphinxupquote{set\_control\_direct}}}{}{}
\sphinxAtStartPar
Sets control mode to direct.

\end{fulllineitems}

\index{set\_control\_increment() (SideLipOutput.SideLipOutput method)@\spxentry{set\_control\_increment()}\spxextra{SideLipOutput.SideLipOutput method}}

\begin{fulllineitems}
\phantomsection\label{\detokenize{specific:SideLipOutput.SideLipOutput.set_control_increment}}\pysiglinewithargsret{\sphinxbfcode{\sphinxupquote{set\_control\_increment}}}{}{}
\sphinxAtStartPar
Sets control mode to increment.

\end{fulllineitems}

\index{set\_control\_toggle() (SideLipOutput.SideLipOutput method)@\spxentry{set\_control\_toggle()}\spxextra{SideLipOutput.SideLipOutput method}}

\begin{fulllineitems}
\phantomsection\label{\detokenize{specific:SideLipOutput.SideLipOutput.set_control_toggle}}\pysiglinewithargsret{\sphinxbfcode{\sphinxupquote{set\_control\_toggle}}}{}{}
\sphinxAtStartPar
Sets control mode to toggle.

\end{fulllineitems}

\index{set\_inversion() (SideLipOutput.SideLipOutput method)@\spxentry{set\_inversion()}\spxextra{SideLipOutput.SideLipOutput method}}

\begin{fulllineitems}
\phantomsection\label{\detokenize{specific:SideLipOutput.SideLipOutput.set_inversion}}\pysiglinewithargsret{\sphinxbfcode{\sphinxupquote{set\_inversion}}}{\emph{\DUrole{n}{is\_inverted}}}{}
\sphinxAtStartPar
Sets whether to invert the output signal or not.
\begin{quote}\begin{description}
\item[{Parameters}] \leavevmode
\sphinxAtStartPar
\sphinxstyleliteralstrong{\sphinxupquote{is\_inverted}} (\sphinxstyleliteralemphasis{\sphinxupquote{boolean}}) \textendash{} the state to set the attribute is\_inverted to

\end{description}\end{quote}

\end{fulllineitems}

\index{set\_outputs() (SideLipOutput.SideLipOutput method)@\spxentry{set\_outputs()}\spxextra{SideLipOutput.SideLipOutput method}}

\begin{fulllineitems}
\phantomsection\label{\detokenize{specific:SideLipOutput.SideLipOutput.set_outputs}}\pysiglinewithargsret{\sphinxbfcode{\sphinxupquote{set\_outputs}}}{\emph{\DUrole{n}{minimums\_output}}, \emph{\DUrole{n}{default\_output}}, \emph{\DUrole{n}{maximums\_output}}}{}
\sphinxAtStartPar
Sets which channels to output to and the minimun, default, and maximum 
pulse width for each of those channels.

\sphinxAtStartPar
Also sets current outputs to the same values as the default.
\begin{quote}\begin{description}
\item[{Parameters}] \leavevmode\begin{itemize}
\item {} 
\sphinxAtStartPar
\sphinxstyleliteralstrong{\sphinxupquote{minimums\_output}} (\sphinxstyleliteralemphasis{\sphinxupquote{{[}}}\sphinxstyleliteralemphasis{\sphinxupquote{int}}\sphinxstyleliteralemphasis{\sphinxupquote{{]}}}) \textendash{} minimum pulse width values for the corresponding 
servo channel

\item {} 
\sphinxAtStartPar
\sphinxstyleliteralstrong{\sphinxupquote{default\_output}} (\sphinxstyleliteralemphasis{\sphinxupquote{{[}}}\sphinxstyleliteralemphasis{\sphinxupquote{int}}\sphinxstyleliteralemphasis{\sphinxupquote{{]}}}) \textendash{} neutral pulse width values for the corresponding 
servo channel, also determines starting position

\item {} 
\sphinxAtStartPar
\sphinxstyleliteralstrong{\sphinxupquote{maximums\_output}} (\sphinxstyleliteralemphasis{\sphinxupquote{{[}}}\sphinxstyleliteralemphasis{\sphinxupquote{int}}\sphinxstyleliteralemphasis{\sphinxupquote{{]}}}) \textendash{} maximum pulse width values for the corresponding 
servo channel

\end{itemize}

\end{description}\end{quote}

\end{fulllineitems}


\end{fulllineitems}



\subsection{Neck Tilt Output}
\label{\detokenize{specific:module-NeckTiltOutput}}\label{\detokenize{specific:neck-tilt-output}}\index{module@\spxentry{module}!NeckTiltOutput@\spxentry{NeckTiltOutput}}\index{NeckTiltOutput@\spxentry{NeckTiltOutput}!module@\spxentry{module}}\index{NeckTiltOutput (class in NeckTiltOutput)@\spxentry{NeckTiltOutput}\spxextra{class in NeckTiltOutput}}

\begin{fulllineitems}
\phantomsection\label{\detokenize{specific:NeckTiltOutput.NeckTiltOutput}}\pysiglinewithargsret{\sphinxbfcode{\sphinxupquote{class }}\sphinxcode{\sphinxupquote{NeckTiltOutput.}}\sphinxbfcode{\sphinxupquote{NeckTiltOutput}}}{\emph{\DUrole{n}{name}}, \emph{\DUrole{n}{num\_outputs}}, \emph{\DUrole{n}{channels\_output}}, \emph{\DUrole{n}{names\_input}}}{}
\sphinxAtStartPar
A class to represent the output object for the two axes of neck tilting 
that will take multiple inputs.

\sphinxAtStartPar
It inherits attributes and methods from MultiInputOutputObject.

\sphinxAtStartPar
…
\begin{quote}\begin{description}
\item[{Attributes}] \leavevmode\begin{itemize}
\item {} 
\sphinxAtStartPar
\sphinxstylestrong{name}(\sphinxstyleemphasis{str}) \textendash{} name of the servo group for the output

\item {} 
\sphinxAtStartPar
\sphinxstylestrong{num\_outputs}(\sphinxstyleemphasis{int}) \textendash{} total number of outputs 
calculated from the given input

\item {} 
\sphinxAtStartPar
\sphinxstylestrong{channels\_output}(\sphinxstyleemphasis{{[}int{]}}) \textendash{} channels corresponding to 
the servos controlled by the outputs

\item {} 
\sphinxAtStartPar
\sphinxstylestrong{maximums\_output}(\sphinxstyleemphasis{{[}int{]}}) \textendash{} maximum pulse width values 
for the corresponding servo channel

\item {} 
\sphinxAtStartPar
\sphinxstylestrong{minimums\_output}(\sphinxstyleemphasis{{[}int{]}}) \textendash{} minimum pulse width values 
for the corresponding servo channel

\item {} 
\sphinxAtStartPar
\sphinxstylestrong{default\_output}(\sphinxstyleemphasis{{[}int{]}}) \textendash{} default output position of 
the servos for start up position and used for some multi input 
objects

\item {} 
\sphinxAtStartPar
\sphinxstylestrong{current\_output}(\sphinxstyleemphasis{{[}int{]}}) \textendash{} current output position of the 
servos to be used for increment mode or multi input objects

\item {} 
\sphinxAtStartPar
\sphinxstylestrong{maximum\_input}(\sphinxstyleemphasis{int}) \textendash{} maximum input value used for mapping 
inputs to outputs, 255 for analog and 1 for digital

\item {} 
\sphinxAtStartPar
\sphinxstylestrong{minimum\_input}(\sphinxstyleemphasis{int}) \textendash{} minimum input value used for mapping 
inputs to outputs, 0 for both analog and digital

\item {} 
\sphinxAtStartPar
\sphinxstylestrong{is\_inverted}(\sphinxstyleemphasis{{[}Boolean{]}}) \textendash{} whether to invert the output 
mapping

\item {} 
\sphinxAtStartPar
\sphinxstylestrong{control\_type}(\sphinxstyleemphasis{Enum ControlType}) \textendash{} mode on how to determine 
the output

\item {} 
\sphinxAtStartPar
\sphinxstylestrong{toggle\_state}(\sphinxstyleemphasis{Enum ToggleState}) \textendash{} used when control mode 
is set to TOGGLE to determine the current output state

\item {} 
\sphinxAtStartPar
\sphinxstylestrong{names\_input}(\sphinxstyleemphasis{{[}str{]}}) \textendash{} list of the input names used by 
the object, this allows the object to know which input value to 
update

\item {} 
\sphinxAtStartPar
\sphinxstylestrong{num\_inputs}(\sphinxstyleemphasis{int}) \textendash{} number of inputs the object uses

\item {} 
\sphinxAtStartPar
\sphinxstylestrong{current\_input}(\sphinxstyleemphasis{{[}int{]}}) \textendash{} since input is given one at a 
time, this list keeps track of previous inputs

\item {} 
\sphinxAtStartPar
\sphinxstylestrong{out\_raw\_min}(\sphinxstyleemphasis{int}) \textendash{} minimum value of an intermediate 
value used to calculate output

\item {} 
\sphinxAtStartPar
\sphinxstylestrong{out\_raw\_max}(\sphinxstyleemphasis{int}) \textendash{} maximum value of an intermediate 
value used to calculate output

\item {} 
\sphinxAtStartPar
\sphinxstylestrong{raw\_output}(\sphinxstyleemphasis{{[}int{]}}) \textendash{} list to store the calculated 
intermediate values that will be mapped to final output values

\end{itemize}

\end{description}\end{quote}

\sphinxAtStartPar
…

\sphinxAtStartPar
\sphinxstylestrong{Methods}
\index{\_\_init\_\_() (NeckTiltOutput.NeckTiltOutput method)@\spxentry{\_\_init\_\_()}\spxextra{NeckTiltOutput.NeckTiltOutput method}}

\begin{fulllineitems}
\phantomsection\label{\detokenize{specific:NeckTiltOutput.NeckTiltOutput.__init__}}\pysiglinewithargsret{\sphinxbfcode{\sphinxupquote{\_\_init\_\_}}}{\emph{\DUrole{n}{name}}, \emph{\DUrole{n}{num\_outputs}}, \emph{\DUrole{n}{channels\_output}}, \emph{\DUrole{n}{names\_input}}}{}
\sphinxAtStartPar
Class constructor. Assigns the values passed in and initalizes remaining 
members to default values.
\begin{quote}\begin{description}
\item[{Parameters}] \leavevmode\begin{itemize}
\item {} 
\sphinxAtStartPar
\sphinxstyleliteralstrong{\sphinxupquote{name}} (\sphinxstyleliteralemphasis{\sphinxupquote{string}}) \textendash{} name of the output group represented by output object

\item {} 
\sphinxAtStartPar
\sphinxstyleliteralstrong{\sphinxupquote{num\_outputs}} (\sphinxstyleliteralemphasis{\sphinxupquote{int}}) \textendash{} number of output channels controlled by the output 
object

\item {} 
\sphinxAtStartPar
\sphinxstyleliteralstrong{\sphinxupquote{channels\_output}} (\sphinxstyleliteralemphasis{\sphinxupquote{{[}}}\sphinxstyleliteralemphasis{\sphinxupquote{int}}\sphinxstyleliteralemphasis{\sphinxupquote{{]}}}) \textendash{} list of the output channels used by the object

\item {} 
\sphinxAtStartPar
\sphinxstyleliteralstrong{\sphinxupquote{names\_input}} (\sphinxstyleliteralemphasis{\sphinxupquote{{[}}}\sphinxstyleliteralemphasis{\sphinxupquote{str}}\sphinxstyleliteralemphasis{\sphinxupquote{{]}}}) \textendash{} the names of the associated controller inputs with 
the object

\end{itemize}

\end{description}\end{quote}

\end{fulllineitems}

\index{get\_default\_outputs() (NeckTiltOutput.NeckTiltOutput method)@\spxentry{get\_default\_outputs()}\spxextra{NeckTiltOutput.NeckTiltOutput method}}

\begin{fulllineitems}
\phantomsection\label{\detokenize{specific:NeckTiltOutput.NeckTiltOutput.get_default_outputs}}\pysiglinewithargsret{\sphinxbfcode{\sphinxupquote{get\_default\_outputs}}}{}{}
\sphinxAtStartPar
Returns the default output values for the object.
\begin{quote}\begin{description}
\item[{Returns}] \leavevmode
\sphinxAtStartPar
Two lists. The first list is the output channels and the second 
is the default output for those channels.

\item[{Return type}] \leavevmode
\sphinxAtStartPar
{[}{[}int{]}, {[}int{]}{]}

\end{description}\end{quote}

\end{fulllineitems}

\index{get\_num\_channels() (NeckTiltOutput.NeckTiltOutput method)@\spxentry{get\_num\_channels()}\spxextra{NeckTiltOutput.NeckTiltOutput method}}

\begin{fulllineitems}
\phantomsection\label{\detokenize{specific:NeckTiltOutput.NeckTiltOutput.get_num_channels}}\pysiglinewithargsret{\sphinxbfcode{\sphinxupquote{get\_num\_channels}}}{}{}
\sphinxAtStartPar
Returns the number of output channels for the object.
\begin{quote}\begin{description}
\item[{Returns}] \leavevmode
\sphinxAtStartPar
number of output channels the object sends output to

\item[{Return type}] \leavevmode
\sphinxAtStartPar
int

\end{description}\end{quote}

\end{fulllineitems}

\index{get\_output() (NeckTiltOutput.NeckTiltOutput method)@\spxentry{get\_output()}\spxextra{NeckTiltOutput.NeckTiltOutput method}}

\begin{fulllineitems}
\phantomsection\label{\detokenize{specific:NeckTiltOutput.NeckTiltOutput.get_output}}\pysiglinewithargsret{\sphinxbfcode{\sphinxupquote{get\_output}}}{\emph{\DUrole{n}{input\_name}}, \emph{\DUrole{n}{input\_value}}}{}
\sphinxAtStartPar
Calculate and returns the output based on the given input value and 
current control mode.
\begin{quote}\begin{description}
\item[{Parameters}] \leavevmode\begin{itemize}
\item {} 
\sphinxAtStartPar
\sphinxstyleliteralstrong{\sphinxupquote{input\_name}} (\sphinxstyleliteralemphasis{\sphinxupquote{str}}) \textendash{} name associated with the input, this object does not 
use this value

\item {} 
\sphinxAtStartPar
\sphinxstyleliteralstrong{\sphinxupquote{input\_value}} (\sphinxstyleliteralemphasis{\sphinxupquote{int}}) \textendash{} the input value from the PS4 controller

\end{itemize}

\item[{Returns}] \leavevmode
\sphinxAtStartPar

\sphinxAtStartPar
Two lists. The first list is the output channels and the second 
is output values for those channels (in units of quarter of milliseconds).

\sphinxAtStartPar
How the ouptut is calculated is based off of which control type the output 
object is set to. Direct will map the output directly based on the input 
and the set input and output ranges. Toggle will set the output between the 
max and the min output values and switch between these values whenever the 
input is released. Increment will increment the output value whenever input 
is given.


\item[{Return type}] \leavevmode
\sphinxAtStartPar
{[}{[}int{]}, {[}int{]}{]}

\end{description}\end{quote}

\end{fulllineitems}

\index{map\_values() (NeckTiltOutput.NeckTiltOutput method)@\spxentry{map\_values()}\spxextra{NeckTiltOutput.NeckTiltOutput method}}

\begin{fulllineitems}
\phantomsection\label{\detokenize{specific:NeckTiltOutput.NeckTiltOutput.map_values}}\pysiglinewithargsret{\sphinxbfcode{\sphinxupquote{map\_values}}}{\emph{\DUrole{n}{value}}, \emph{\DUrole{n}{input\_min}}, \emph{\DUrole{n}{input\_max}}, \emph{\DUrole{n}{out\_min}}, \emph{\DUrole{n}{out\_max}}}{}
\sphinxAtStartPar
Maps an input value to its output.
\begin{quote}\begin{description}
\item[{Parameters}] \leavevmode\begin{itemize}
\item {} 
\sphinxAtStartPar
\sphinxstyleliteralstrong{\sphinxupquote{value}} (\sphinxstyleliteralemphasis{\sphinxupquote{float}}) \textendash{} value of the input to map to an output value

\item {} 
\sphinxAtStartPar
\sphinxstyleliteralstrong{\sphinxupquote{input\_min}} (\sphinxstyleliteralemphasis{\sphinxupquote{float}}) \textendash{} minimun input value in input range

\item {} 
\sphinxAtStartPar
\sphinxstyleliteralstrong{\sphinxupquote{input\_max}} (\sphinxstyleliteralemphasis{\sphinxupquote{float}}) \textendash{} maximun input value in input range

\item {} 
\sphinxAtStartPar
\sphinxstyleliteralstrong{\sphinxupquote{out\_min}} (\sphinxstyleliteralemphasis{\sphinxupquote{int}}) \textendash{} minimun output value in output range

\item {} 
\sphinxAtStartPar
\sphinxstyleliteralstrong{\sphinxupquote{out\_max}} (\sphinxstyleliteralemphasis{\sphinxupquote{int}}) \textendash{} maximun output value in output range

\end{itemize}

\item[{Returns}] \leavevmode
\sphinxAtStartPar
pulse width to output for the given input value

\item[{Return type}] \leavevmode
\sphinxAtStartPar
int

\end{description}\end{quote}

\end{fulllineitems}

\index{set\_control\_direct() (NeckTiltOutput.NeckTiltOutput method)@\spxentry{set\_control\_direct()}\spxextra{NeckTiltOutput.NeckTiltOutput method}}

\begin{fulllineitems}
\phantomsection\label{\detokenize{specific:NeckTiltOutput.NeckTiltOutput.set_control_direct}}\pysiglinewithargsret{\sphinxbfcode{\sphinxupquote{set\_control\_direct}}}{}{}
\sphinxAtStartPar
Sets control mode to direct.

\end{fulllineitems}

\index{set\_control\_increment() (NeckTiltOutput.NeckTiltOutput method)@\spxentry{set\_control\_increment()}\spxextra{NeckTiltOutput.NeckTiltOutput method}}

\begin{fulllineitems}
\phantomsection\label{\detokenize{specific:NeckTiltOutput.NeckTiltOutput.set_control_increment}}\pysiglinewithargsret{\sphinxbfcode{\sphinxupquote{set\_control\_increment}}}{}{}
\sphinxAtStartPar
Sets control mode to increment.

\end{fulllineitems}

\index{set\_control\_toggle() (NeckTiltOutput.NeckTiltOutput method)@\spxentry{set\_control\_toggle()}\spxextra{NeckTiltOutput.NeckTiltOutput method}}

\begin{fulllineitems}
\phantomsection\label{\detokenize{specific:NeckTiltOutput.NeckTiltOutput.set_control_toggle}}\pysiglinewithargsret{\sphinxbfcode{\sphinxupquote{set\_control\_toggle}}}{}{}
\sphinxAtStartPar
Sets control mode to toggle.

\end{fulllineitems}

\index{set\_inversion() (NeckTiltOutput.NeckTiltOutput method)@\spxentry{set\_inversion()}\spxextra{NeckTiltOutput.NeckTiltOutput method}}

\begin{fulllineitems}
\phantomsection\label{\detokenize{specific:NeckTiltOutput.NeckTiltOutput.set_inversion}}\pysiglinewithargsret{\sphinxbfcode{\sphinxupquote{set\_inversion}}}{\emph{\DUrole{n}{is\_inverted}}}{}
\sphinxAtStartPar
Sets whether to invert the output signal or not.
\begin{quote}\begin{description}
\item[{Parameters}] \leavevmode
\sphinxAtStartPar
\sphinxstyleliteralstrong{\sphinxupquote{is\_inverted}} (\sphinxstyleliteralemphasis{\sphinxupquote{boolean}}) \textendash{} the state to set the attribute is\_inverted to

\end{description}\end{quote}

\end{fulllineitems}

\index{set\_outputs() (NeckTiltOutput.NeckTiltOutput method)@\spxentry{set\_outputs()}\spxextra{NeckTiltOutput.NeckTiltOutput method}}

\begin{fulllineitems}
\phantomsection\label{\detokenize{specific:NeckTiltOutput.NeckTiltOutput.set_outputs}}\pysiglinewithargsret{\sphinxbfcode{\sphinxupquote{set\_outputs}}}{\emph{\DUrole{n}{minimums\_output}}, \emph{\DUrole{n}{default\_output}}, \emph{\DUrole{n}{maximums\_output}}}{}
\sphinxAtStartPar
Sets which channels to output to and the minimun, default, and maximum 
pulse width for each of those channels.

\sphinxAtStartPar
Also sets current outputs to the same values as the default.
\begin{quote}\begin{description}
\item[{Parameters}] \leavevmode\begin{itemize}
\item {} 
\sphinxAtStartPar
\sphinxstyleliteralstrong{\sphinxupquote{minimums\_output}} (\sphinxstyleliteralemphasis{\sphinxupquote{{[}}}\sphinxstyleliteralemphasis{\sphinxupquote{int}}\sphinxstyleliteralemphasis{\sphinxupquote{{]}}}) \textendash{} minimum pulse width values for the corresponding 
servo channel

\item {} 
\sphinxAtStartPar
\sphinxstyleliteralstrong{\sphinxupquote{default\_output}} (\sphinxstyleliteralemphasis{\sphinxupquote{{[}}}\sphinxstyleliteralemphasis{\sphinxupquote{int}}\sphinxstyleliteralemphasis{\sphinxupquote{{]}}}) \textendash{} neutral pulse width values for the corresponding 
servo channel, also determines starting position

\item {} 
\sphinxAtStartPar
\sphinxstyleliteralstrong{\sphinxupquote{maximums\_output}} (\sphinxstyleliteralemphasis{\sphinxupquote{{[}}}\sphinxstyleliteralemphasis{\sphinxupquote{int}}\sphinxstyleliteralemphasis{\sphinxupquote{{]}}}) \textendash{} maximum pulse width values for the corresponding 
servo channel

\end{itemize}

\end{description}\end{quote}

\end{fulllineitems}


\end{fulllineitems}



\renewcommand{\indexname}{Python Module Index}
\begin{sphinxtheindex}
\let\bigletter\sphinxstyleindexlettergroup
\bigletter{a}
\item\relax\sphinxstyleindexentry{AnalogMixerOutput}\sphinxstyleindexpageref{generic:\detokenize{module-AnalogMixerOutput}}
\item\relax\sphinxstyleindexentry{AnalogOutputObject}\sphinxstyleindexpageref{generic:\detokenize{module-AnalogOutputObject}}
\indexspace
\bigletter{d}
\item\relax\sphinxstyleindexentry{DigitalOutputObject}\sphinxstyleindexpageref{generic:\detokenize{module-DigitalOutputObject}}
\indexspace
\bigletter{e}
\item\relax\sphinxstyleindexentry{EarOutput}\sphinxstyleindexpageref{specific:\detokenize{module-EarOutput}}
\item\relax\sphinxstyleindexentry{EyebrowsOutput}\sphinxstyleindexpageref{specific:\detokenize{module-EyebrowsOutput}}
\indexspace
\bigletter{m}
\item\relax\sphinxstyleindexentry{manualControl}\sphinxstyleindexpageref{manualcontrol:\detokenize{module-manualControl}}
\item\relax\sphinxstyleindexentry{MovementMap}\sphinxstyleindexpageref{movementmap:\detokenize{module-MovementMap}}
\item\relax\sphinxstyleindexentry{MultiInputOutputObject}\sphinxstyleindexpageref{base:\detokenize{module-MultiInputOutputObject}}
\indexspace
\bigletter{n}
\item\relax\sphinxstyleindexentry{NeckTiltOutput}\sphinxstyleindexpageref{specific:\detokenize{module-NeckTiltOutput}}
\indexspace
\bigletter{o}
\item\relax\sphinxstyleindexentry{OutputObject}\sphinxstyleindexpageref{base:\detokenize{module-OutputObject}}
\indexspace
\bigletter{s}
\item\relax\sphinxstyleindexentry{SideLipOutput}\sphinxstyleindexpageref{specific:\detokenize{module-SideLipOutput}}
\end{sphinxtheindex}


\end{document}