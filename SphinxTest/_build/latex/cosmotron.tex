%% Generated by Sphinx.
\def\sphinxdocclass{report}
\documentclass[letterpaper,10pt,english]{sphinxmanual}
\ifdefined\pdfpxdimen
   \let\sphinxpxdimen\pdfpxdimen\else\newdimen\sphinxpxdimen
\fi \sphinxpxdimen=.75bp\relax

\PassOptionsToPackage{warn}{textcomp}
\usepackage[utf8]{inputenc}
\ifdefined\DeclareUnicodeCharacter
% support both utf8 and utf8x syntaxes
  \ifdefined\DeclareUnicodeCharacterAsOptional
    \def\sphinxDUC#1{\DeclareUnicodeCharacter{"#1}}
  \else
    \let\sphinxDUC\DeclareUnicodeCharacter
  \fi
  \sphinxDUC{00A0}{\nobreakspace}
  \sphinxDUC{2500}{\sphinxunichar{2500}}
  \sphinxDUC{2502}{\sphinxunichar{2502}}
  \sphinxDUC{2514}{\sphinxunichar{2514}}
  \sphinxDUC{251C}{\sphinxunichar{251C}}
  \sphinxDUC{2572}{\textbackslash}
\fi
\usepackage{cmap}
\usepackage[T1]{fontenc}
\usepackage{amsmath,amssymb,amstext}
\usepackage{babel}



\usepackage{times}
\expandafter\ifx\csname T@LGR\endcsname\relax
\else
% LGR was declared as font encoding
  \substitutefont{LGR}{\rmdefault}{cmr}
  \substitutefont{LGR}{\sfdefault}{cmss}
  \substitutefont{LGR}{\ttdefault}{cmtt}
\fi
\expandafter\ifx\csname T@X2\endcsname\relax
  \expandafter\ifx\csname T@T2A\endcsname\relax
  \else
  % T2A was declared as font encoding
    \substitutefont{T2A}{\rmdefault}{cmr}
    \substitutefont{T2A}{\sfdefault}{cmss}
    \substitutefont{T2A}{\ttdefault}{cmtt}
  \fi
\else
% X2 was declared as font encoding
  \substitutefont{X2}{\rmdefault}{cmr}
  \substitutefont{X2}{\sfdefault}{cmss}
  \substitutefont{X2}{\ttdefault}{cmtt}
\fi


\usepackage[Bjarne]{fncychap}
\usepackage{sphinx}

\fvset{fontsize=\small}
\usepackage{geometry}


% Include hyperref last.
\usepackage{hyperref}
% Fix anchor placement for figures with captions.
\usepackage{hypcap}% it must be loaded after hyperref.
% Set up styles of URL: it should be placed after hyperref.
\urlstyle{same}

\addto\captionsenglish{\renewcommand{\contentsname}{Contents:}}

\usepackage{sphinxmessages}
\setcounter{tocdepth}{1}



\title{Cosmotron}
\date{Apr 12, 2021}
\release{2021}
\author{Parker King, Ethan Harris}
\newcommand{\sphinxlogo}{\vbox{}}
\renewcommand{\releasename}{Release}
\makeindex
\begin{document}

\pagestyle{empty}
\sphinxmaketitle
\pagestyle{plain}
\sphinxtableofcontents
\pagestyle{normal}
\phantomsection\label{\detokenize{index::doc}}



\chapter{Input Event Handling}
\label{\detokenize{input:module-manualControl}}\label{\detokenize{input:input-event-handling}}\label{\detokenize{input::doc}}\index{module@\spxentry{module}!manualControl@\spxentry{manualControl}}\index{manualControl@\spxentry{manualControl}!module@\spxentry{module}}
\sphinxAtStartPar
This module will handle all input from the Bluetooth connected PS4 controllers in an asynchronous manner. It expects two controllers to be connected 
and will not run until then. If either controller is disconnected the program will need to be restarted in order to reconnect.
Sends modified controller input to the MovementMap.
\index{ControllerEvent (class in manualControl)@\spxentry{ControllerEvent}\spxextra{class in manualControl}}

\begin{fulllineitems}
\phantomsection\label{\detokenize{input:manualControl.ControllerEvent}}\pysiglinewithargsret{\sphinxbfcode{\sphinxupquote{class }}\sphinxcode{\sphinxupquote{manualControl.}}\sphinxbfcode{\sphinxupquote{ControllerEvent}}}{\emph{\DUrole{n}{name}}, \emph{\DUrole{n}{value}}}{}
\sphinxAtStartPar
Controller event object. To be sent to the servo handler to convert a button press to movement, for example. 
Contains a predetermined name and the current event’s value. 
For buttons the value is 1 or 0. For analog axes it is 0\sphinxhyphen{}255. For d\sphinxhyphen{}pad hat axes it is \sphinxhyphen{}1,0,1.

\end{fulllineitems}

\index{process\_events() (in module manualControl)@\spxentry{process\_events()}\spxextra{in module manualControl}}

\begin{fulllineitems}
\phantomsection\label{\detokenize{input:manualControl.process_events}}\pysiglinewithargsret{\sphinxbfcode{\sphinxupquote{async }}\sphinxcode{\sphinxupquote{manualControl.}}\sphinxbfcode{\sphinxupquote{process\_events}}}{\emph{\DUrole{n}{device}}}{}
\sphinxAtStartPar
Async helper function. Process PS4 events and create ControllerEvent objects to be passed to the movementMap

\end{fulllineitems}



\chapter{Movement Mapping}
\label{\detokenize{mapping:module-MovementMap}}\label{\detokenize{mapping:movement-mapping}}\label{\detokenize{mapping::doc}}\index{module@\spxentry{module}!MovementMap@\spxentry{MovementMap}}\index{MovementMap@\spxentry{MovementMap}!module@\spxentry{module}}\index{MovementMap (class in MovementMap)@\spxentry{MovementMap}\spxextra{class in MovementMap}}

\begin{fulllineitems}
\phantomsection\label{\detokenize{mapping:MovementMap.MovementMap}}\pysigline{\sphinxbfcode{\sphinxupquote{class }}\sphinxcode{\sphinxupquote{MovementMap.}}\sphinxbfcode{\sphinxupquote{MovementMap}}}
\sphinxAtStartPar
A class that handles controller inputs. Contains a mapping of input objects to output objects.

\sphinxAtStartPar
…
\begin{description}
\item[{servoBoard}] \leavevmode{[}maestro.Controller{]}
\sphinxAtStartPar
object that establishes serial conection to the Maestro and has functions to send commands to the board

\item[{output\_objects}] \leavevmode{[}{[}OutputObject{]}{]}
\sphinxAtStartPar
list of all the output objects, used to activate each output on the Maestro board on startup

\item[{right\_ear}] \leavevmode{[}AnalogOutputObject{]}
\sphinxAtStartPar
object to map inputs to output for the right ear movements

\item[{left\_ear}] \leavevmode{[}AnalogOutputObject{]}
\sphinxAtStartPar
object to map inputs to output for the left ear movements

\item[{eyelids}] \leavevmode{[}AnalogOutputObject{]}
\sphinxAtStartPar
object to map input to outputs for the top and bottom eyelid movements

\item[{eyes\_horizontal}] \leavevmode{[}AnalogOutputObject{]}
\sphinxAtStartPar
object to map input to outputs for the left and right eye horizontal movements

\item[{eyes\_vertical}] \leavevmode{[}AnalogOutputObject{]}
\sphinxAtStartPar
object to map input to outputs for the left and right eye vertical movements

\item[{eyebrows}] \leavevmode{[}EybrowsOutput{]}
\sphinxAtStartPar
object to map inputs to outputs for the eyebrow movements

\item[{nose}] \leavevmode{[}DigitalOutputObject{]}
\sphinxAtStartPar
object to map input to output for the nose movements

\item[{top\_lip}] \leavevmode{[}DigitalOutputObject{]}
\sphinxAtStartPar
object to map input to output for the top lip movements

\item[{right\_lip}] \leavevmode{[}SideLipOutput{]}
\sphinxAtStartPar
object to map inputs to outputs for the right lip movements

\item[{left\_lip}] \leavevmode{[}SideLipOutput{]}
\sphinxAtStartPar
object to map inputs to outputs for the left lip movements

\item[{jaw}] \leavevmode{[}AnalogOutputObject{]}
\sphinxAtStartPar
object to map input to outputs for the jaw movements

\item[{neck\_twist}] \leavevmode{[}AnalogOutputObject{]}
\sphinxAtStartPar
object to map input to output for the neck twist movements

\item[{neck\_tilt}] \leavevmode{[}NeckTiltOutput{]}
\sphinxAtStartPar
object to map inputs to outpus for the neck tilt movements

\item[{input\_map}] \leavevmode{[}dictionary {[}input\_name (string): output\_object (OutputObject){]}{]}
\sphinxAtStartPar
mapping of the controller inputs to movement outputs

\end{description}
\begin{description}
\item[{\_\_init\_():}] \leavevmode
\sphinxAtStartPar
Class constructor. Creates needed output objects and sets their parameters. Also creates the input map.

\item[{send\_outputs(num\_outputs, channel, output):}] \leavevmode
\sphinxAtStartPar
Sends the output to the corresponding channel from the given lists of channels and outputs.

\item[{process\_input(input\_object):}] \leavevmode
\sphinxAtStartPar
Using the input map, determines the correct output object to send the input value to. The returned outputs are passed to the maestro board.

\item[{start\_outputs():}] \leavevmode
\sphinxAtStartPar
Sends the starting outputs for each motor to the Maestro board to activate each channel in default positions.

\end{description}
\index{process\_input() (MovementMap.MovementMap method)@\spxentry{process\_input()}\spxextra{MovementMap.MovementMap method}}

\begin{fulllineitems}
\phantomsection\label{\detokenize{mapping:MovementMap.MovementMap.process_input}}\pysiglinewithargsret{\sphinxbfcode{\sphinxupquote{process\_input}}}{\emph{\DUrole{n}{input\_object}}}{}
\sphinxAtStartPar
Using the input map, determines the correct output object to send the input value to. The returned outputs are passed to the maestro board.
\begin{description}
\item[{input\_object}] \leavevmode{[}ControllerEvent{]}
\sphinxAtStartPar
object containing the name of the input and its associated value to be processed

\end{description}

\end{fulllineitems}

\index{send\_outputs() (MovementMap.MovementMap method)@\spxentry{send\_outputs()}\spxextra{MovementMap.MovementMap method}}

\begin{fulllineitems}
\phantomsection\label{\detokenize{mapping:MovementMap.MovementMap.send_outputs}}\pysiglinewithargsret{\sphinxbfcode{\sphinxupquote{send\_outputs}}}{\emph{\DUrole{n}{num\_outputs}}, \emph{\DUrole{n}{channel}}, \emph{\DUrole{n}{output}}}{}
\sphinxAtStartPar
Sends the output to the corresponding channel from the given lists of channels and outputs.
\begin{description}
\item[{num\_outputs}] \leavevmode{[}int{]}
\sphinxAtStartPar
number of channels and outputs to loop through

\item[{channel}] \leavevmode{[}{[}int{]}{]}
\sphinxAtStartPar
channel numbers on the Maestro board to send outputs to

\item[{output}] \leavevmode{[}{[}double{]}{]}
\sphinxAtStartPar
values to output on the Maestro board, cast to int to ensure proper typing

\end{description}

\sphinxAtStartPar
None

\end{fulllineitems}

\index{start\_outputs() (MovementMap.MovementMap method)@\spxentry{start\_outputs()}\spxextra{MovementMap.MovementMap method}}

\begin{fulllineitems}
\phantomsection\label{\detokenize{mapping:MovementMap.MovementMap.start_outputs}}\pysiglinewithargsret{\sphinxbfcode{\sphinxupquote{start\_outputs}}}{}{}
\sphinxAtStartPar
Sends the starting outputs for each motor to the Maestro board to activate each channel in default positions.

\sphinxAtStartPar
None

\sphinxAtStartPar
None

\end{fulllineitems}


\end{fulllineitems}



\chapter{Output Objects}
\label{\detokenize{output:output-objects}}\label{\detokenize{output::doc}}

\section{Base Objects}
\label{\detokenize{base:base-objects}}\label{\detokenize{base::doc}}

\subsection{Output Object}
\label{\detokenize{base:module-OutputObject}}\label{\detokenize{base:output-object}}\index{module@\spxentry{module}!OutputObject@\spxentry{OutputObject}}\index{OutputObject@\spxentry{OutputObject}!module@\spxentry{module}}\index{ControlType (class in OutputObject)@\spxentry{ControlType}\spxextra{class in OutputObject}}

\begin{fulllineitems}
\phantomsection\label{\detokenize{base:OutputObject.ControlType}}\pysiglinewithargsret{\sphinxbfcode{\sphinxupquote{class }}\sphinxcode{\sphinxupquote{OutputObject.}}\sphinxbfcode{\sphinxupquote{ControlType}}}{\emph{\DUrole{n}{value}}}{}
\sphinxAtStartPar
The output mode of the input. This determines how output is calculated from the inputs.
\begin{description}
\item[{DIRECT}] \leavevmode{[}auto{]}
\sphinxAtStartPar
output is mapped directly to the input value

\item[{TOGGLE}] \leavevmode{[}auto{]}
\sphinxAtStartPar
output is switched between states and is triggered by a false value

\item[{INCREMENT}] \leavevmode{[}auto{]}
\sphinxAtStartPar
output is incremented by the input

\end{description}

\end{fulllineitems}

\index{OutputObject (class in OutputObject)@\spxentry{OutputObject}\spxextra{class in OutputObject}}

\begin{fulllineitems}
\phantomsection\label{\detokenize{base:OutputObject.OutputObject}}\pysiglinewithargsret{\sphinxbfcode{\sphinxupquote{class }}\sphinxcode{\sphinxupquote{OutputObject.}}\sphinxbfcode{\sphinxupquote{OutputObject}}}{\emph{\DUrole{n}{name}}, \emph{\DUrole{n}{num\_outputs}}, \emph{\DUrole{n}{channels\_output}}}{}
\sphinxAtStartPar
A base class to represent a an output object for a controller input.

\sphinxAtStartPar
…
\begin{description}
\item[{name}] \leavevmode{[}str{]}
\sphinxAtStartPar
name of the servo group for the output

\item[{num\_outputs}] \leavevmode{[}int{]}
\sphinxAtStartPar
total number of outputs calculated from the given input

\item[{channels\_output}] \leavevmode{[}{[}int{]}{]}
\sphinxAtStartPar
channels corresponding to the servos controlled by the outputs

\item[{maximums\_output}] \leavevmode{[}{[}int{]}{]}
\sphinxAtStartPar
maximum pulse width values for the corresponding servo channel

\item[{minimums\_output}] \leavevmode{[}{[}int{]}{]}
\sphinxAtStartPar
minimum pulse width values for the corresponding servo channel

\item[{default\_output}] \leavevmode{[}{[}int{]}{]}
\sphinxAtStartPar
default output position of the servos for start up position and used for some multi input objects

\item[{current\_output}] \leavevmode{[}{[}int{]}{]}
\sphinxAtStartPar
current output position of the servos to be used for increment mode or multi input objects

\item[{maximum\_input}] \leavevmode{[}int{]}
\sphinxAtStartPar
maximum input value used for mapping inputs to outputs, 255 for analog and 1 for digital

\item[{minimum\_input}] \leavevmode{[}int{]}
\sphinxAtStartPar
minimum input value used for mapping inputs to outputs, 0 for both analog and digital

\item[{is\_inverted}] \leavevmode{[}Boolean list{]}
\sphinxAtStartPar
whether to invert the output mapping

\item[{control\_type}] \leavevmode{[}Enum ControlType{]}
\sphinxAtStartPar
mode on how to determine the output

\item[{toggle\_state}] \leavevmode{[}Enum ToggleState{]}
\sphinxAtStartPar
used when control mode is set to TOGGLE to determine the current output state

\end{description}
\begin{description}
\item[{\_\_init\_(name, num\_outputs, channels\_output):}] \leavevmode
\sphinxAtStartPar
Class constructor. Assigns the values passed in and initalizes remaining members to default values.

\item[{set\_outputs(channels\_output, maximums\_output, minimums\_output):}] \leavevmode
\sphinxAtStartPar
Sets which channels to output to and the minimun, default, and maximum pulse width for each of those channels.
Also sets current outputs to the same values as the default.

\item[{set\_inversion(is\_inverted):}] \leavevmode
\sphinxAtStartPar
Sets whether to invert the output signal or not.

\item[{set\_control\_direct():}] \leavevmode
\sphinxAtStartPar
Sets control mode to direct.

\item[{set\_control\_toggle():}] \leavevmode
\sphinxAtStartPar
Sets control mode to toggle.

\item[{set\_control\_increment():}] \leavevmode
\sphinxAtStartPar
Sets control mode to increment.

\item[{get\_num\_channels():}] \leavevmode
\sphinxAtStartPar
Returns the number of output channels for the object.

\item[{get\_default\_outputs():}] \leavevmode
\sphinxAtStartPar
Returns the default output values for the object.

\item[{map\_values(value, input\_min, input\_max, out\_min, out\_max):}] \leavevmode
\sphinxAtStartPar
Maps an input value to its output.

\end{description}
\index{get\_default\_outputs() (OutputObject.OutputObject method)@\spxentry{get\_default\_outputs()}\spxextra{OutputObject.OutputObject method}}

\begin{fulllineitems}
\phantomsection\label{\detokenize{base:OutputObject.OutputObject.get_default_outputs}}\pysiglinewithargsret{\sphinxbfcode{\sphinxupquote{get\_default\_outputs}}}{}{}~\begin{quote}

\sphinxAtStartPar
Returns the default output values for the object.
\end{quote}

\sphinxAtStartPar
None
\begin{description}
\item[{{[}channels\_output, default\_output{]}}] \leavevmode{[}{[}{[}int{]}, {[}int{]}{]}{]}
\sphinxAtStartPar
corresponding channels to the default output values for the object

\end{description}

\end{fulllineitems}

\index{get\_num\_channels() (OutputObject.OutputObject method)@\spxentry{get\_num\_channels()}\spxextra{OutputObject.OutputObject method}}

\begin{fulllineitems}
\phantomsection\label{\detokenize{base:OutputObject.OutputObject.get_num_channels}}\pysiglinewithargsret{\sphinxbfcode{\sphinxupquote{get\_num\_channels}}}{}{}
\sphinxAtStartPar
Returns the number of output channels for the object.

\sphinxAtStartPar
None
\begin{description}
\item[{num\_outputs}] \leavevmode{[}int{]}
\sphinxAtStartPar
number of output channels the object sends output to

\end{description}

\end{fulllineitems}

\index{map\_values() (OutputObject.OutputObject method)@\spxentry{map\_values()}\spxextra{OutputObject.OutputObject method}}

\begin{fulllineitems}
\phantomsection\label{\detokenize{base:OutputObject.OutputObject.map_values}}\pysiglinewithargsret{\sphinxbfcode{\sphinxupquote{map\_values}}}{\emph{\DUrole{n}{value}}, \emph{\DUrole{n}{input\_min}}, \emph{\DUrole{n}{input\_max}}, \emph{\DUrole{n}{out\_min}}, \emph{\DUrole{n}{out\_max}}}{}
\sphinxAtStartPar
Maps an input value to its output.
\begin{description}
\item[{value}] \leavevmode{[}float{]}
\sphinxAtStartPar
value of the input to map to an output value

\item[{input\_min}] \leavevmode{[}float{]}
\sphinxAtStartPar
minimun input value in input range

\item[{input\_max}] \leavevmode{[}float{]}
\sphinxAtStartPar
maximun input value in input range

\item[{out\_min}] \leavevmode{[}int{]}
\sphinxAtStartPar
minimun output value in output range

\item[{out\_max}] \leavevmode{[}int{]}
\sphinxAtStartPar
maximun output value in output range

\end{description}
\begin{description}
\item[{mapped\_output}] \leavevmode{[}int{]}
\sphinxAtStartPar
pulse width to output for the given input value

\end{description}

\end{fulllineitems}

\index{set\_control\_direct() (OutputObject.OutputObject method)@\spxentry{set\_control\_direct()}\spxextra{OutputObject.OutputObject method}}

\begin{fulllineitems}
\phantomsection\label{\detokenize{base:OutputObject.OutputObject.set_control_direct}}\pysiglinewithargsret{\sphinxbfcode{\sphinxupquote{set\_control\_direct}}}{}{}
\sphinxAtStartPar
Sets control mode to direct.

\sphinxAtStartPar
None

\sphinxAtStartPar
None

\end{fulllineitems}

\index{set\_control\_increment() (OutputObject.OutputObject method)@\spxentry{set\_control\_increment()}\spxextra{OutputObject.OutputObject method}}

\begin{fulllineitems}
\phantomsection\label{\detokenize{base:OutputObject.OutputObject.set_control_increment}}\pysiglinewithargsret{\sphinxbfcode{\sphinxupquote{set\_control\_increment}}}{}{}
\sphinxAtStartPar
Sets control mode to increment.

\sphinxAtStartPar
None

\sphinxAtStartPar
None

\end{fulllineitems}

\index{set\_control\_toggle() (OutputObject.OutputObject method)@\spxentry{set\_control\_toggle()}\spxextra{OutputObject.OutputObject method}}

\begin{fulllineitems}
\phantomsection\label{\detokenize{base:OutputObject.OutputObject.set_control_toggle}}\pysiglinewithargsret{\sphinxbfcode{\sphinxupquote{set\_control\_toggle}}}{}{}
\sphinxAtStartPar
Sets control mode to toggle.

\sphinxAtStartPar
None

\sphinxAtStartPar
None

\end{fulllineitems}

\index{set\_inversion() (OutputObject.OutputObject method)@\spxentry{set\_inversion()}\spxextra{OutputObject.OutputObject method}}

\begin{fulllineitems}
\phantomsection\label{\detokenize{base:OutputObject.OutputObject.set_inversion}}\pysiglinewithargsret{\sphinxbfcode{\sphinxupquote{set\_inversion}}}{\emph{\DUrole{n}{is\_inverted}}}{}
\sphinxAtStartPar
Sets whether to invert the output signal or not.
\begin{description}
\item[{is\_inverted}] \leavevmode{[}boolean{]}
\sphinxAtStartPar
the state to set the attribute is\_inverted to

\end{description}

\sphinxAtStartPar
None

\end{fulllineitems}

\index{set\_outputs() (OutputObject.OutputObject method)@\spxentry{set\_outputs()}\spxextra{OutputObject.OutputObject method}}

\begin{fulllineitems}
\phantomsection\label{\detokenize{base:OutputObject.OutputObject.set_outputs}}\pysiglinewithargsret{\sphinxbfcode{\sphinxupquote{set\_outputs}}}{\emph{\DUrole{n}{minimums\_output}}, \emph{\DUrole{n}{default\_output}}, \emph{\DUrole{n}{maximums\_output}}}{}
\sphinxAtStartPar
Sets which channels to output to and the minimun, default, and maximum pulse width for each of those channels.
Also sets current outputs to the same values as the default.
\begin{description}
\item[{minimums\_output}] \leavevmode{[}{[}int{]}{]}
\sphinxAtStartPar
minimum pulse width values for the corresponding servo channel

\item[{default\_output}] \leavevmode{[}{[}int{]}{]}
\sphinxAtStartPar
neutral pulse width values for the corresponding servo channel, also determines starting position

\item[{maximums\_output}] \leavevmode{[}{[}int{]}{]}
\sphinxAtStartPar
maximum pulse width values for the corresponding servo channel

\end{description}

\sphinxAtStartPar
None

\end{fulllineitems}


\end{fulllineitems}

\index{ToggleState (class in OutputObject)@\spxentry{ToggleState}\spxextra{class in OutputObject}}

\begin{fulllineitems}
\phantomsection\label{\detokenize{base:OutputObject.ToggleState}}\pysiglinewithargsret{\sphinxbfcode{\sphinxupquote{class }}\sphinxcode{\sphinxupquote{OutputObject.}}\sphinxbfcode{\sphinxupquote{ToggleState}}}{\emph{\DUrole{n}{value}}}{}
\sphinxAtStartPar
Used to determine output behavior when the control type is set to toggle mode.
\begin{description}
\item[{ON}] \leavevmode{[}auto{]}
\sphinxAtStartPar
the output is toggled “on”

\item[{OFF}] \leavevmode{[}auto{]}
\sphinxAtStartPar
the output is toggled “off”

\end{description}

\end{fulllineitems}



\subsection{Multi Input Output Object}
\label{\detokenize{base:module-MultiInputOutputObject}}\label{\detokenize{base:multi-input-output-object}}\index{module@\spxentry{module}!MultiInputOutputObject@\spxentry{MultiInputOutputObject}}\index{MultiInputOutputObject@\spxentry{MultiInputOutputObject}!module@\spxentry{module}}\index{MultiInputOutputObject (class in MultiInputOutputObject)@\spxentry{MultiInputOutputObject}\spxextra{class in MultiInputOutputObject}}

\begin{fulllineitems}
\phantomsection\label{\detokenize{base:MultiInputOutputObject.MultiInputOutputObject}}\pysiglinewithargsret{\sphinxbfcode{\sphinxupquote{class }}\sphinxcode{\sphinxupquote{MultiInputOutputObject.}}\sphinxbfcode{\sphinxupquote{MultiInputOutputObject}}}{\emph{\DUrole{n}{name}}, \emph{\DUrole{n}{num\_outputs}}, \emph{\DUrole{n}{channels\_output}}, \emph{\DUrole{n}{names\_input}}}{}
\sphinxAtStartPar
A base class to represent a an output object for movement groups that take multiple inputs.
It inherits attributes and methods from OutputObject.

\sphinxAtStartPar
…
\begin{description}
\item[{name}] \leavevmode{[}str{]}
\sphinxAtStartPar
name of the servo group for the output

\item[{num\_outputs}] \leavevmode{[}int{]}
\sphinxAtStartPar
total number of outputs calculated from the given input

\item[{channels\_output}] \leavevmode{[}{[}int{]}{]}
\sphinxAtStartPar
channels corresponding to the servos controlled by the outputs

\item[{maximums\_output}] \leavevmode{[}{[}int{]}{]}
\sphinxAtStartPar
maximum pulse width values for the corresponding servo channel

\item[{minimums\_output}] \leavevmode{[}{[}int{]}{]}
\sphinxAtStartPar
minimum pulse width values for the corresponding servo channel

\item[{default\_output}] \leavevmode{[}{[}int{]}{]}
\sphinxAtStartPar
default output position of the servos for start up position

\item[{current\_output}] \leavevmode{[}{[}int{]}{]}
\sphinxAtStartPar
current output position of the servos to be used for increment mode

\item[{maximum\_input}] \leavevmode{[}int{]}
\sphinxAtStartPar
maximum input value used for mapping inputs to outputs

\item[{minimum\_input}] \leavevmode{[}int{]}
\sphinxAtStartPar
minimum input value used for mapping inputs to outputs

\item[{is\_inverted}] \leavevmode{[}Boolean list{]}
\sphinxAtStartPar
whether to invert the output mapping

\item[{control\_type}] \leavevmode{[}Enum ControlType{]}
\sphinxAtStartPar
mode on how to determine the output

\item[{toggle\_state}] \leavevmode{[}Enum ToggleState{]}
\sphinxAtStartPar
used when control mode is set to TOGGLE to determine the current output state

\item[{names\_input}] \leavevmode{[}{[}str{]}{]}
\sphinxAtStartPar
list of the input names used by the object, this allows the object to know which input value to update

\item[{num\_inputs}] \leavevmode{[}int{]}
\sphinxAtStartPar
number of inputs the object uses

\item[{current\_input}] \leavevmode{[}{[}int{]}{]}
\sphinxAtStartPar
since input is given one at a time, this list keeps track of previous inputs

\item[{out\_raw\_min}] \leavevmode{[}int{]}
\sphinxAtStartPar
minimum value of an intermediate value used to calculate output

\item[{out\_raw\_max}] \leavevmode{[}int{]}
\sphinxAtStartPar
maximum value of an intermediate value used to calculate output

\item[{raw\_output}] \leavevmode{[}{[}int{]}{]}
\sphinxAtStartPar
list to store the calculated intermediate values that will be mapped to final output values

\end{description}
\begin{description}
\item[{\_\_init\_(name, num\_outputs, channels\_output, names\_input):}] \leavevmode
\sphinxAtStartPar
Class constructor. Assigns the values passed in and initalizes remaining members to default values.

\item[{set\_outputs(channels\_output, maximums\_output, minimums\_output):}] \leavevmode
\sphinxAtStartPar
Sets which channels to output to and the minimun, default, and maximum pulse width for each of those channels.
Also sets current outputs to the same values as the default.

\item[{set\_inversion(is\_inverted):}] \leavevmode
\sphinxAtStartPar
Sets whether to invert the output signal or not.

\item[{set\_control\_direct():}] \leavevmode
\sphinxAtStartPar
Sets control mode to direct.

\item[{set\_control\_toggle():}] \leavevmode
\sphinxAtStartPar
Sets control mode to toggle.

\item[{set\_control\_increment():}] \leavevmode
\sphinxAtStartPar
Sets control mode to increment.

\item[{get\_num\_channels():}] \leavevmode
\sphinxAtStartPar
Returns the number of output channels for the object.

\item[{get\_default\_outputs():}] \leavevmode
\sphinxAtStartPar
Returns the default output values for the object.

\item[{map\_values(value, input\_min, input\_max, out\_min, out\_max):}] \leavevmode
\sphinxAtStartPar
Maps an input value to its output.

\end{description}

\end{fulllineitems}



\section{Generic Objects}
\label{\detokenize{generic:generic-objects}}\label{\detokenize{generic::doc}}

\subsection{Digital Output Object}
\label{\detokenize{generic:module-DigitalOutputObject}}\label{\detokenize{generic:digital-output-object}}\index{module@\spxentry{module}!DigitalOutputObject@\spxentry{DigitalOutputObject}}\index{DigitalOutputObject@\spxentry{DigitalOutputObject}!module@\spxentry{module}}\index{DigitalOutputObject (class in DigitalOutputObject)@\spxentry{DigitalOutputObject}\spxextra{class in DigitalOutputObject}}

\begin{fulllineitems}
\phantomsection\label{\detokenize{generic:DigitalOutputObject.DigitalOutputObject}}\pysiglinewithargsret{\sphinxbfcode{\sphinxupquote{class }}\sphinxcode{\sphinxupquote{DigitalOutputObject.}}\sphinxbfcode{\sphinxupquote{DigitalOutputObject}}}{\emph{\DUrole{n}{name}}, \emph{\DUrole{n}{num\_outputs}}, \emph{\DUrole{n}{channels\_output}}}{}
\sphinxAtStartPar
A class to represent a an output object for a digital controller input.
It inherits attributes and methods from OutputObject.

\sphinxAtStartPar
…
\begin{description}
\item[{name}] \leavevmode{[}str{]}
\sphinxAtStartPar
name of the servo group for the output

\item[{num\_outputs}] \leavevmode{[}int{]}
\sphinxAtStartPar
total number of outputs calculated from the given input

\item[{channels\_output}] \leavevmode{[}{[}int{]}{]}
\sphinxAtStartPar
channels corresponding to the servos controlled by the outputs

\item[{maximums\_output}] \leavevmode{[}{[}int{]}{]}
\sphinxAtStartPar
maximum pulse width values for the corresponding servo channel

\item[{minimums\_output}] \leavevmode{[}{[}int{]}{]}
\sphinxAtStartPar
minimum pulse width values for the corresponding servo channel

\item[{default\_output}] \leavevmode{[}{[}int{]}{]}
\sphinxAtStartPar
default output position of the servos for start up position

\item[{current\_output}] \leavevmode{[}{[}int{]}{]}
\sphinxAtStartPar
current output position of the servos to be used for increment mode

\item[{maximum\_input}] \leavevmode{[}int{]}
\sphinxAtStartPar
maximum input value used for mapping inputs to outputs

\item[{minimum\_input}] \leavevmode{[}int{]}
\sphinxAtStartPar
minimum input value used for mapping inputs to outputs

\item[{is\_inverted}] \leavevmode{[}Boolean list{]}
\sphinxAtStartPar
whether to invert the output mapping

\item[{control\_type}] \leavevmode{[}Enum ControlType{]}
\sphinxAtStartPar
mode on how to determine the output

\item[{toggle\_state}] \leavevmode{[}Enum ToggleState{]}
\sphinxAtStartPar
used when control mode is set to TOGGLE to determine the current output state

\end{description}
\begin{description}
\item[{\_\_init\_(name, num\_outputs, channels\_output):}] \leavevmode
\sphinxAtStartPar
Class constructor. Assigns the values passed in and initalizes remaining members to default values.

\item[{set\_outputs(channels\_output, maximums\_output, minimums\_output):}] \leavevmode
\sphinxAtStartPar
Sets which channels to output to and the minimun, default, and maximum pulse width for each of those channels.
Also sets current outputs to the same values as the default.

\item[{set\_inversion(is\_inverted):}] \leavevmode
\sphinxAtStartPar
Sets whether to invert the output signal or not.

\item[{set\_control\_direct():}] \leavevmode
\sphinxAtStartPar
Sets control mode to direct.

\item[{set\_control\_toggle():}] \leavevmode
\sphinxAtStartPar
Sets control mode to toggle.

\item[{set\_control\_increment():}] \leavevmode
\sphinxAtStartPar
Sets control mode to increment.

\item[{get\_num\_channels():}] \leavevmode
\sphinxAtStartPar
Returns the number of output channels for the object.

\item[{get\_default\_outputs():}] \leavevmode
\sphinxAtStartPar
Returns the default output values for the object.

\item[{get\_output(input\_name, input\_value):}] \leavevmode
\sphinxAtStartPar
Calculate and returns the output based on the given input value and current control mode.

\item[{map\_values(value, input\_min, input\_max, out\_min, out\_max):}] \leavevmode
\sphinxAtStartPar
Maps an input value to its output.

\end{description}
\index{get\_output() (DigitalOutputObject.DigitalOutputObject method)@\spxentry{get\_output()}\spxextra{DigitalOutputObject.DigitalOutputObject method}}

\begin{fulllineitems}
\phantomsection\label{\detokenize{generic:DigitalOutputObject.DigitalOutputObject.get_output}}\pysiglinewithargsret{\sphinxbfcode{\sphinxupquote{get\_output}}}{\emph{\DUrole{n}{input\_name}}, \emph{\DUrole{n}{input\_value}}}{}
\sphinxAtStartPar
Calculate and returns the output based on the given input value and current control mode.
\begin{description}
\item[{input\_name}] \leavevmode{[}str{]}
\sphinxAtStartPar
name associated with the input, this object does not use this value

\item[{input\_value}] \leavevmode{[}int{]}
\sphinxAtStartPar
the input value from the PS4 controller

\end{description}
\begin{description}
\item[{{[}channels\_output, current\_output{]}}] \leavevmode{[}{[}{[}int{]}, {[}int{]}{]}{]}
\sphinxAtStartPar
current\_output is the pulse widths in quarter microseconds to output, and channels\_output
is which channels those outputs will be sent over. How the ouptut is calculated is based
off of which control type the output object is set to. Direct will map the output directly
based on the input and the set input and output ranges. Toggle will set the output between
the max and the min output values and switch between these values whenever the input is
released. Increment will increment the output value whenever input is given.

\end{description}

\end{fulllineitems}


\end{fulllineitems}



\subsection{Analog Output Object}
\label{\detokenize{generic:module-AnalogOutputObject}}\label{\detokenize{generic:analog-output-object}}\index{module@\spxentry{module}!AnalogOutputObject@\spxentry{AnalogOutputObject}}\index{AnalogOutputObject@\spxentry{AnalogOutputObject}!module@\spxentry{module}}\index{AnalogOutputObject (class in AnalogOutputObject)@\spxentry{AnalogOutputObject}\spxextra{class in AnalogOutputObject}}

\begin{fulllineitems}
\phantomsection\label{\detokenize{generic:AnalogOutputObject.AnalogOutputObject}}\pysiglinewithargsret{\sphinxbfcode{\sphinxupquote{class }}\sphinxcode{\sphinxupquote{AnalogOutputObject.}}\sphinxbfcode{\sphinxupquote{AnalogOutputObject}}}{\emph{\DUrole{n}{name}}, \emph{\DUrole{n}{num\_outputs}}, \emph{\DUrole{n}{channels\_output}}}{}
\sphinxAtStartPar
A class to represent a an output object for an analog controller input.
It inherits attributes and methods from OutputObject.

\sphinxAtStartPar
…
\begin{description}
\item[{name}] \leavevmode{[}str{]}
\sphinxAtStartPar
name of the servo group for the output

\item[{num\_outputs}] \leavevmode{[}int{]}
\sphinxAtStartPar
total number of outputs calculated from the given input

\item[{channels\_output}] \leavevmode{[}{[}int{]}{]}
\sphinxAtStartPar
channels corresponding to the servos controlled by the outputs

\item[{maximums\_output}] \leavevmode{[}{[}int{]}{]}
\sphinxAtStartPar
maximum pulse width values for the corresponding servo channel

\item[{minimums\_output}] \leavevmode{[}{[}int{]}{]}
\sphinxAtStartPar
minimum pulse width values for the corresponding servo channel

\item[{default\_output}] \leavevmode{[}{[}int{]}{]}
\sphinxAtStartPar
default output position of the servos for start up position

\item[{current\_output}] \leavevmode{[}{[}int{]}{]}
\sphinxAtStartPar
current output position of the servos to be used for increment mode

\item[{maximum\_input}] \leavevmode{[}int{]}
\sphinxAtStartPar
maximum input value used for mapping inputs to outputs

\item[{minimum\_input}] \leavevmode{[}int{]}
\sphinxAtStartPar
minimum input value used for mapping inputs to outputs

\item[{is\_inverted}] \leavevmode{[}Boolean list{]}
\sphinxAtStartPar
whether to invert the output mapping

\item[{control\_type}] \leavevmode{[}Enum ControlType{]}
\sphinxAtStartPar
mode on how to determine the output

\item[{toggle\_state}] \leavevmode{[}Enum ToggleState{]}
\sphinxAtStartPar
used when control mode is set to TOGGLE to determine the current output state

\end{description}
\begin{description}
\item[{\_\_init\_(name, num\_outputs, channels\_output):}] \leavevmode
\sphinxAtStartPar
Class constructor. Assigns the values passed in and initalizes remaining members to default values.

\item[{set\_outputs(channels\_output, maximums\_output, minimums\_output):}] \leavevmode
\sphinxAtStartPar
Sets which channels to output to and the minimun, default, and maximum pulse width for each of those channels.
Also sets current outputs to the same values as the default.

\item[{set\_inversion(is\_inverted):}] \leavevmode
\sphinxAtStartPar
Sets whether to invert the output signal or not.

\item[{set\_control\_direct():}] \leavevmode
\sphinxAtStartPar
Sets control mode to direct.

\item[{set\_control\_toggle():}] \leavevmode
\sphinxAtStartPar
Sets control mode to toggle.

\item[{set\_control\_increment():}] \leavevmode
\sphinxAtStartPar
Sets control mode to increment.

\item[{get\_num\_channels():}] \leavevmode
\sphinxAtStartPar
Returns the number of output channels for the object.

\item[{get\_default\_outputs():}] \leavevmode
\sphinxAtStartPar
Returns the default output values for the object.

\item[{get\_output(input\_name, input\_value):}] \leavevmode
\sphinxAtStartPar
Calculate and returns the output based on the given input value and current control mode.

\item[{map\_values(value, input\_min, input\_max, out\_min, out\_max):}] \leavevmode
\sphinxAtStartPar
Maps an input value to its output.

\end{description}
\index{get\_output() (AnalogOutputObject.AnalogOutputObject method)@\spxentry{get\_output()}\spxextra{AnalogOutputObject.AnalogOutputObject method}}

\begin{fulllineitems}
\phantomsection\label{\detokenize{generic:AnalogOutputObject.AnalogOutputObject.get_output}}\pysiglinewithargsret{\sphinxbfcode{\sphinxupquote{get\_output}}}{\emph{\DUrole{n}{input\_name}}, \emph{\DUrole{n}{input\_value}}}{}
\sphinxAtStartPar
Calculate and returns the output based on the given input value and current control mode.
\begin{description}
\item[{input\_name}] \leavevmode{[}str{]}
\sphinxAtStartPar
name associated with the input, this object does not use this value

\item[{input\_value}] \leavevmode{[}int{]}
\sphinxAtStartPar
the input value from the PS4 controller

\end{description}
\begin{description}
\item[{{[}channels\_output, current\_output{]}}] \leavevmode{[}{[}{[}int{]}, {[}int{]}{]}{]}
\sphinxAtStartPar
current\_output is the pulse widths in quarter microseconds to output, and channels\_output
is which channels those outputs will be sent over. How the ouptut is calculated is based
off of which control type the output object is set to. Direct will map the output directly
based on the input and the set input and output ranges. Toggle will set the output between
the max and the min output values and switch between these values whenever the input is
released. Increment will increment the output value whenever input is given.

\end{description}

\end{fulllineitems}


\end{fulllineitems}



\subsection{Analog Mixer Output}
\label{\detokenize{generic:module-AnalogMixerOutput}}\label{\detokenize{generic:analog-mixer-output}}\index{module@\spxentry{module}!AnalogMixerOutput@\spxentry{AnalogMixerOutput}}\index{AnalogMixerOutput@\spxentry{AnalogMixerOutput}!module@\spxentry{module}}\index{AnalogMixerOutput (class in AnalogMixerOutput)@\spxentry{AnalogMixerOutput}\spxextra{class in AnalogMixerOutput}}

\begin{fulllineitems}
\phantomsection\label{\detokenize{generic:AnalogMixerOutput.AnalogMixerOutput}}\pysiglinewithargsret{\sphinxbfcode{\sphinxupquote{class }}\sphinxcode{\sphinxupquote{AnalogMixerOutput.}}\sphinxbfcode{\sphinxupquote{AnalogMixerOutput}}}{\emph{\DUrole{n}{name}}, \emph{\DUrole{n}{num\_outputs}}, \emph{\DUrole{n}{channels\_output}}, \emph{\DUrole{n}{names\_input}}}{}
\sphinxAtStartPar
A class to represent an output object that mixes two analog inputs.
It inherits attributes and methods from MultiInputOutputObject.

\sphinxAtStartPar
…
\begin{description}
\item[{name}] \leavevmode{[}str{]}
\sphinxAtStartPar
name of the servo group for the output

\item[{num\_outputs}] \leavevmode{[}int{]}
\sphinxAtStartPar
total number of outputs calculated from the given input

\item[{channels\_output}] \leavevmode{[}{[}int{]}{]}
\sphinxAtStartPar
channels corresponding to the servos controlled by the outputs

\item[{maximums\_output}] \leavevmode{[}{[}int{]}{]}
\sphinxAtStartPar
maximum pulse width values for the corresponding servo channel

\item[{minimums\_output}] \leavevmode{[}{[}int{]}{]}
\sphinxAtStartPar
minimum pulse width values for the corresponding servo channel

\item[{default\_output}] \leavevmode{[}{[}int{]}{]}
\sphinxAtStartPar
default output position of the servos for start up position

\item[{current\_output}] \leavevmode{[}{[}int{]}{]}
\sphinxAtStartPar
current output position of the servos to be used for increment mode

\item[{maximum\_input}] \leavevmode{[}int{]}
\sphinxAtStartPar
maximum input value used for mapping inputs to outputs

\item[{minimum\_input}] \leavevmode{[}int{]}
\sphinxAtStartPar
minimum input value used for mapping inputs to outputs

\item[{is\_inverted}] \leavevmode{[}Boolean list{]}
\sphinxAtStartPar
whether to invert the output mapping

\item[{control\_type}] \leavevmode{[}Enum ControlType{]}
\sphinxAtStartPar
mode on how to determine the output

\item[{toggle\_state}] \leavevmode{[}Enum ToggleState{]}
\sphinxAtStartPar
used when control mode is set to TOGGLE to determine the current output state

\item[{names\_input}] \leavevmode{[}{[}str{]}{]}
\sphinxAtStartPar
list of the input names used by the object, this allows the object to know which input value to update

\item[{num\_inputs}] \leavevmode{[}int{]}
\sphinxAtStartPar
number of inputs the object uses

\item[{current\_input}] \leavevmode{[}{[}int{]}{]}
\sphinxAtStartPar
since input is given one at a time, this list keeps track of previous inputs

\item[{out\_raw\_min}] \leavevmode{[}int{]}
\sphinxAtStartPar
minimum value of an intermediate value used to calculate output

\item[{out\_raw\_max}] \leavevmode{[}int{]}
\sphinxAtStartPar
maximum value of an intermediate value used to calculate output

\item[{raw\_output}] \leavevmode{[}{[}int{]}{]}
\sphinxAtStartPar
list to store the calculated intermediate values that will be mapped to final output values

\end{description}
\begin{description}
\item[{\_\_init\_(name, num\_outputs, channels\_output, names\_input):}] \leavevmode
\sphinxAtStartPar
Class constructor. Assigns the values passed in and initalizes remaining members to default values.

\item[{set\_outputs(channels\_output, maximums\_output, minimums\_output):}] \leavevmode
\sphinxAtStartPar
Sets which channels to output to and the minimun, default, and maximum pulse width for each of those channels.
Also sets current outputs to the same values as the default.

\item[{set\_inversion(is\_inverted):}] \leavevmode
\sphinxAtStartPar
Sets whether to invert the output signal or not.

\item[{set\_control\_direct():}] \leavevmode
\sphinxAtStartPar
Sets control mode to direct.

\item[{set\_control\_toggle():}] \leavevmode
\sphinxAtStartPar
Sets control mode to toggle.

\item[{set\_control\_increment():}] \leavevmode
\sphinxAtStartPar
Sets control mode to increment.

\item[{get\_num\_channels():}] \leavevmode
\sphinxAtStartPar
Returns the number of output channels for the object.

\item[{get\_default\_outputs():}] \leavevmode
\sphinxAtStartPar
Returns the default output values for the object.

\item[{get\_output(input\_name, input\_value):}] \leavevmode
\sphinxAtStartPar
Calculate and returns the output based on the given input value and current control mode.

\item[{map\_values(value, input\_min, input\_max, out\_min, out\_max):}] \leavevmode
\sphinxAtStartPar
Maps an input value to its output.

\end{description}
\index{get\_output() (AnalogMixerOutput.AnalogMixerOutput method)@\spxentry{get\_output()}\spxextra{AnalogMixerOutput.AnalogMixerOutput method}}

\begin{fulllineitems}
\phantomsection\label{\detokenize{generic:AnalogMixerOutput.AnalogMixerOutput.get_output}}\pysiglinewithargsret{\sphinxbfcode{\sphinxupquote{get\_output}}}{\emph{\DUrole{n}{input\_name}}, \emph{\DUrole{n}{input\_value}}}{}
\sphinxAtStartPar
Calculate and returns the output based on the given input value and current control mode.
\begin{description}
\item[{input\_name}] \leavevmode{[}str{]}
\sphinxAtStartPar
name associated with the input

\item[{input\_value}] \leavevmode{[}int{]}
\sphinxAtStartPar
the input value from the PS4 controller

\end{description}
\begin{description}
\item[{{[}channels\_output, current\_output{]}}] \leavevmode{[}{[}{[}int{]}, {[}int{]}{]}{]}
\sphinxAtStartPar
current\_output is the pulse widths in quarter microseconds to output, and channels\_output
is which channels those outputs will be sent over. How the ouptut is calculated is based
off of which control type the output object is set to. Direct will map the output directly
based on the input and the set input and output ranges. Toggle will set the output between
the max and the min output values and switch between these values whenever the input is
released. Increment will increment the output value whenever input is given.

\end{description}

\end{fulllineitems}


\end{fulllineitems}



\section{Specific Objects}
\label{\detokenize{specific:specific-objects}}\label{\detokenize{specific::doc}}

\subsection{Ear Output}
\label{\detokenize{specific:module-EarOutput}}\label{\detokenize{specific:ear-output}}\index{module@\spxentry{module}!EarOutput@\spxentry{EarOutput}}\index{EarOutput@\spxentry{EarOutput}!module@\spxentry{module}}\index{EarOutput (class in EarOutput)@\spxentry{EarOutput}\spxextra{class in EarOutput}}

\begin{fulllineitems}
\phantomsection\label{\detokenize{specific:EarOutput.EarOutput}}\pysiglinewithargsret{\sphinxbfcode{\sphinxupquote{class }}\sphinxcode{\sphinxupquote{EarOutput.}}\sphinxbfcode{\sphinxupquote{EarOutput}}}{\emph{\DUrole{n}{name}}, \emph{\DUrole{n}{num\_outputs}}, \emph{\DUrole{n}{channels\_output}}, \emph{\DUrole{n}{names\_input}}}{}
\sphinxAtStartPar
A class to represent the output object for the left or right ear that will take multiple inputs.
It inherits attributes and methods from MultiInputOutputObject.

\sphinxAtStartPar
…
\begin{description}
\item[{name}] \leavevmode{[}str{]}
\sphinxAtStartPar
name of the servo group for the output

\item[{num\_outputs}] \leavevmode{[}int{]}
\sphinxAtStartPar
total number of outputs calculated from the given input

\item[{channels\_output}] \leavevmode{[}{[}int{]}{]}
\sphinxAtStartPar
channels corresponding to the servos controlled by the outputs

\item[{maximums\_output}] \leavevmode{[}{[}int{]}{]}
\sphinxAtStartPar
maximum pulse width values for the corresponding servo channel

\item[{minimums\_output}] \leavevmode{[}{[}int{]}{]}
\sphinxAtStartPar
minimum pulse width values for the corresponding servo channel

\item[{default\_output}] \leavevmode{[}{[}int{]}{]}
\sphinxAtStartPar
default output position of the servos for start up position

\item[{current\_output}] \leavevmode{[}{[}int{]}{]}
\sphinxAtStartPar
current output position of the servos to be used for increment mode

\item[{maximum\_input}] \leavevmode{[}int{]}
\sphinxAtStartPar
maximum input value used for mapping inputs to outputs

\item[{minimum\_input}] \leavevmode{[}int{]}
\sphinxAtStartPar
minimum input value used for mapping inputs to outputs

\item[{is\_inverted}] \leavevmode{[}Boolean list{]}
\sphinxAtStartPar
whether to invert the output mapping

\item[{control\_type}] \leavevmode{[}Enum ControlType{]}
\sphinxAtStartPar
mode on how to determine the output

\item[{toggle\_state}] \leavevmode{[}Enum ToggleState{]}
\sphinxAtStartPar
used when control mode is set to TOGGLE to determine the current output state

\item[{names\_input}] \leavevmode{[}{[}str{]}{]}
\sphinxAtStartPar
list of the input names used by the object, this allows the object to know which input value to update

\item[{num\_inputs}] \leavevmode{[}int{]}
\sphinxAtStartPar
number of inputs the object uses

\item[{current\_input}] \leavevmode{[}{[}int{]}{]}
\sphinxAtStartPar
since input is given one at a time, this list keeps track of previous inputs

\item[{out\_raw\_min}] \leavevmode{[}int{]}
\sphinxAtStartPar
minimum value of an intermediate value used to calculate output

\item[{out\_raw\_max}] \leavevmode{[}int{]}
\sphinxAtStartPar
maximum value of an intermediate value used to calculate output

\item[{raw\_output}] \leavevmode{[}{[}int{]}{]}
\sphinxAtStartPar
list to store the calculated intermediate values that will be mapped to final output values

\end{description}
\begin{description}
\item[{\_\_init\_(name, num\_outputs, channels\_output, names\_input):}] \leavevmode
\sphinxAtStartPar
Class constructor. Assigns the values passed in and initalizes remaining members to default values.

\item[{set\_outputs(channels\_output, maximums\_output, minimums\_output):}] \leavevmode
\sphinxAtStartPar
Sets which channels to output to and the minimun, default, and maximum pulse width for each of those channels.
Also sets current outputs to the same values as the default.

\item[{set\_inversion(is\_inverted):}] \leavevmode
\sphinxAtStartPar
Sets whether to invert the output signal or not.

\item[{set\_control\_direct():}] \leavevmode
\sphinxAtStartPar
Sets control mode to direct.

\item[{set\_control\_toggle():}] \leavevmode
\sphinxAtStartPar
Sets control mode to toggle.

\item[{set\_control\_increment():}] \leavevmode
\sphinxAtStartPar
Sets control mode to increment.

\item[{get\_num\_channels():}] \leavevmode
\sphinxAtStartPar
Returns the number of output channels for the object.

\item[{get\_default\_outputs():}] \leavevmode
\sphinxAtStartPar
Returns the default output values for the object.

\item[{get\_output(input\_name, input\_value):}] \leavevmode
\sphinxAtStartPar
Calculate and returns the output based on the given input value and current control mode.

\item[{map\_values(value, input\_min, input\_max, out\_min, out\_max):}] \leavevmode
\sphinxAtStartPar
Maps an input value to its output.

\end{description}
\index{get\_output() (EarOutput.EarOutput method)@\spxentry{get\_output()}\spxextra{EarOutput.EarOutput method}}

\begin{fulllineitems}
\phantomsection\label{\detokenize{specific:EarOutput.EarOutput.get_output}}\pysiglinewithargsret{\sphinxbfcode{\sphinxupquote{get\_output}}}{\emph{\DUrole{n}{input\_name}}, \emph{\DUrole{n}{input\_value}}}{}
\sphinxAtStartPar
Calculate and returns the output based on the given input value and current control mode.
\begin{description}
\item[{input\_name}] \leavevmode{[}str{]}
\sphinxAtStartPar
name associated with the input

\item[{input\_value}] \leavevmode{[}int{]}
\sphinxAtStartPar
the input value from the PS4 controller

\end{description}
\begin{description}
\item[{{[}channels\_output, current\_output{]}}] \leavevmode{[}{[}{[}int{]}, {[}int{]}{]}{]}
\sphinxAtStartPar
current\_output is the pulse widths in quarter microseconds to output, and channels\_output
is which channels those outputs will be sent over. How the ouptut is calculated is based
off of which control type the output object is set to. Direct will map the output directly
based on the input and the set input and output ranges. Toggle will set the output between
the max and the min output values and switch between these values whenever the input is
released. Increment will increment the output value whenever input is given.

\end{description}

\end{fulllineitems}


\end{fulllineitems}



\subsection{Eyebrows Output}
\label{\detokenize{specific:module-EyebrowsOutput}}\label{\detokenize{specific:eyebrows-output}}\index{module@\spxentry{module}!EyebrowsOutput@\spxentry{EyebrowsOutput}}\index{EyebrowsOutput@\spxentry{EyebrowsOutput}!module@\spxentry{module}}\index{EyebrowsOutput (class in EyebrowsOutput)@\spxentry{EyebrowsOutput}\spxextra{class in EyebrowsOutput}}

\begin{fulllineitems}
\phantomsection\label{\detokenize{specific:EyebrowsOutput.EyebrowsOutput}}\pysiglinewithargsret{\sphinxbfcode{\sphinxupquote{class }}\sphinxcode{\sphinxupquote{EyebrowsOutput.}}\sphinxbfcode{\sphinxupquote{EyebrowsOutput}}}{\emph{\DUrole{n}{name}}, \emph{\DUrole{n}{num\_outputs}}, \emph{\DUrole{n}{channels\_output}}, \emph{\DUrole{n}{names\_input}}}{}
\sphinxAtStartPar
A class to represent the output object for the eyebrows that will take multiple inputs.
It inherits attributes and methods from MultiInputOutputObject.

\sphinxAtStartPar
…
\begin{description}
\item[{name}] \leavevmode{[}str{]}
\sphinxAtStartPar
name of the servo group for the output

\item[{num\_outputs}] \leavevmode{[}int{]}
\sphinxAtStartPar
total number of outputs calculated from the given input

\item[{channels\_output}] \leavevmode{[}{[}int{]}{]}
\sphinxAtStartPar
channels corresponding to the servos controlled by the outputs

\item[{maximums\_output}] \leavevmode{[}{[}int{]}{]}
\sphinxAtStartPar
maximum pulse width values for the corresponding servo channel

\item[{minimums\_output}] \leavevmode{[}{[}int{]}{]}
\sphinxAtStartPar
minimum pulse width values for the corresponding servo channel

\item[{default\_output}] \leavevmode{[}{[}int{]}{]}
\sphinxAtStartPar
default output position of the servos for start up position

\item[{current\_output}] \leavevmode{[}{[}int{]}{]}
\sphinxAtStartPar
current output position of the servos to be used for increment mode

\item[{maximum\_input}] \leavevmode{[}int{]}
\sphinxAtStartPar
maximum input value used for mapping inputs to outputs

\item[{minimum\_input}] \leavevmode{[}int{]}
\sphinxAtStartPar
minimum input value used for mapping inputs to outputs

\item[{is\_inverted}] \leavevmode{[}Boolean list{]}
\sphinxAtStartPar
whether to invert the output mapping

\item[{control\_type}] \leavevmode{[}Enum ControlType{]}
\sphinxAtStartPar
mode on how to determine the output

\item[{toggle\_state}] \leavevmode{[}Enum ToggleState{]}
\sphinxAtStartPar
used when control mode is set to TOGGLE to determine the current output state

\item[{names\_input}] \leavevmode{[}{[}str{]}{]}
\sphinxAtStartPar
list of the input names used by the object, this allows the object to know which input value to update

\item[{num\_inputs}] \leavevmode{[}int{]}
\sphinxAtStartPar
number of inputs the object uses

\item[{current\_input}] \leavevmode{[}{[}int{]}{]}
\sphinxAtStartPar
since input is given one at a time, this list keeps track of previous inputs

\item[{out\_raw\_min}] \leavevmode{[}int{]}
\sphinxAtStartPar
minimum value of an intermediate value used to calculate output

\item[{out\_raw\_max}] \leavevmode{[}int{]}
\sphinxAtStartPar
maximum value of an intermediate value used to calculate output

\item[{raw\_output}] \leavevmode{[}{[}int{]}{]}
\sphinxAtStartPar
list to store the calculated intermediate values that will be mapped to final output values

\end{description}
\begin{description}
\item[{\_\_init\_(name, num\_outputs, channels\_output, names\_input):}] \leavevmode
\sphinxAtStartPar
Class constructor. Assigns the values passed in and initalizes remaining members to default values.

\item[{set\_outputs(channels\_output, maximums\_output, minimums\_output):}] \leavevmode
\sphinxAtStartPar
Sets which channels to output to and the minimun, default, and maximum pulse width for each of those channels.
Also sets current outputs to the same values as the default.

\item[{set\_inversion(is\_inverted):}] \leavevmode
\sphinxAtStartPar
Sets whether to invert the output signal or not.

\item[{set\_control\_direct():}] \leavevmode
\sphinxAtStartPar
Sets control mode to direct.

\item[{set\_control\_toggle():}] \leavevmode
\sphinxAtStartPar
Sets control mode to toggle.

\item[{set\_control\_increment():}] \leavevmode
\sphinxAtStartPar
Sets control mode to increment.

\item[{get\_num\_channels():}] \leavevmode
\sphinxAtStartPar
Returns the number of output channels for the object.

\item[{get\_default\_outputs():}] \leavevmode
\sphinxAtStartPar
Returns the default output values for the object.

\item[{get\_output(input\_name, input\_value):}] \leavevmode
\sphinxAtStartPar
Calculate and returns the output based on the given input value and current control mode.

\item[{map\_values(value, input\_min, input\_max, out\_min, out\_max):}] \leavevmode
\sphinxAtStartPar
Maps an input value to its output.

\end{description}
\index{get\_output() (EyebrowsOutput.EyebrowsOutput method)@\spxentry{get\_output()}\spxextra{EyebrowsOutput.EyebrowsOutput method}}

\begin{fulllineitems}
\phantomsection\label{\detokenize{specific:EyebrowsOutput.EyebrowsOutput.get_output}}\pysiglinewithargsret{\sphinxbfcode{\sphinxupquote{get\_output}}}{\emph{\DUrole{n}{input\_name}}, \emph{\DUrole{n}{input\_value}}}{}
\sphinxAtStartPar
Calculate and returns the output based on the given input value and current control mode.
\begin{description}
\item[{input\_name}] \leavevmode{[}str{]}
\sphinxAtStartPar
name associated with the input

\item[{input\_value}] \leavevmode{[}int{]}
\sphinxAtStartPar
the input value from the PS4 controller

\end{description}
\begin{description}
\item[{{[}channels\_output, current\_output{]}}] \leavevmode{[}{[}{[}int{]}, {[}int{]}{]}{]}
\sphinxAtStartPar
current\_output is the pulse widths in quarter microseconds to output, and channels\_output
is which channels those outputs will be sent over. How the ouptut is calculated is based
off of which control type the output object is set to. Direct will map the output directly
based on the input and the set input and output ranges. Toggle will set the output between
the max and the min output values and switch between these values whenever the input is
released. Increment will increment the output value whenever input is given.

\end{description}

\end{fulllineitems}


\end{fulllineitems}



\subsection{Side Lip Output}
\label{\detokenize{specific:module-SideLipOutput}}\label{\detokenize{specific:side-lip-output}}\index{module@\spxentry{module}!SideLipOutput@\spxentry{SideLipOutput}}\index{SideLipOutput@\spxentry{SideLipOutput}!module@\spxentry{module}}\index{SideLipOutput (class in SideLipOutput)@\spxentry{SideLipOutput}\spxextra{class in SideLipOutput}}

\begin{fulllineitems}
\phantomsection\label{\detokenize{specific:SideLipOutput.SideLipOutput}}\pysiglinewithargsret{\sphinxbfcode{\sphinxupquote{class }}\sphinxcode{\sphinxupquote{SideLipOutput.}}\sphinxbfcode{\sphinxupquote{SideLipOutput}}}{\emph{\DUrole{n}{name}}, \emph{\DUrole{n}{num\_outputs}}, \emph{\DUrole{n}{channels\_output}}, \emph{\DUrole{n}{names\_input}}}{}
\sphinxAtStartPar
A class to represent the output object for the left or right lip that will take multiple inputs.
It inherits attributes and methods from MultiInputOutputObject.

\sphinxAtStartPar
…
\begin{description}
\item[{name}] \leavevmode{[}str{]}
\sphinxAtStartPar
name of the servo group for the output

\item[{num\_outputs}] \leavevmode{[}int{]}
\sphinxAtStartPar
total number of outputs calculated from the given input

\item[{channels\_output}] \leavevmode{[}{[}int{]}{]}
\sphinxAtStartPar
channels corresponding to the servos controlled by the outputs

\item[{maximums\_output}] \leavevmode{[}{[}int{]}{]}
\sphinxAtStartPar
maximum pulse width values for the corresponding servo channel

\item[{minimums\_output}] \leavevmode{[}{[}int{]}{]}
\sphinxAtStartPar
minimum pulse width values for the corresponding servo channel

\item[{default\_output}] \leavevmode{[}{[}int{]}{]}
\sphinxAtStartPar
default output position of the servos for start up position

\item[{current\_output}] \leavevmode{[}{[}int{]}{]}
\sphinxAtStartPar
current output position of the servos to be used for increment mode

\item[{maximum\_input}] \leavevmode{[}int{]}
\sphinxAtStartPar
maximum input value used for mapping inputs to outputs

\item[{minimum\_input}] \leavevmode{[}int{]}
\sphinxAtStartPar
minimum input value used for mapping inputs to outputs

\item[{is\_inverted}] \leavevmode{[}Boolean list{]}
\sphinxAtStartPar
whether to invert the output mapping

\item[{control\_type}] \leavevmode{[}Enum ControlType{]}
\sphinxAtStartPar
mode on how to determine the output

\item[{toggle\_state}] \leavevmode{[}Enum ToggleState{]}
\sphinxAtStartPar
used when control mode is set to TOGGLE to determine the current output state

\item[{names\_input}] \leavevmode{[}{[}str{]}{]}
\sphinxAtStartPar
list of the input names used by the object, this allows the object to know which input value to update

\item[{num\_inputs}] \leavevmode{[}int{]}
\sphinxAtStartPar
number of inputs the object uses

\item[{current\_input}] \leavevmode{[}{[}int{]}{]}
\sphinxAtStartPar
since input is given one at a time, this list keeps track of previous inputs

\item[{out\_raw\_min}] \leavevmode{[}int{]}
\sphinxAtStartPar
minimum value of an intermediate value used to calculate output

\item[{out\_raw\_max}] \leavevmode{[}int{]}
\sphinxAtStartPar
maximum value of an intermediate value used to calculate output

\item[{raw\_output}] \leavevmode{[}{[}int{]}{]}
\sphinxAtStartPar
list to store the calculated intermediate values that will be mapped to final output values

\end{description}
\begin{description}
\item[{\_\_init\_(name, num\_outputs, channels\_output, names\_input:}] \leavevmode
\sphinxAtStartPar
Class constructor. Assigns the values passed in and initalizes remaining members to default values.

\item[{set\_outputs(channels\_output, maximums\_output, minimums\_output):}] \leavevmode
\sphinxAtStartPar
Sets which channels to output to and the minimun, default, and maximum pulse width for each of those channels.
Also sets current outputs to the same values as the default.

\item[{set\_inversion(is\_inverted):}] \leavevmode
\sphinxAtStartPar
Sets whether to invert the output signal or not.

\item[{set\_control\_direct():}] \leavevmode
\sphinxAtStartPar
Sets control mode to direct.

\item[{set\_control\_toggle():}] \leavevmode
\sphinxAtStartPar
Sets control mode to toggle.

\item[{set\_control\_increment():}] \leavevmode
\sphinxAtStartPar
Sets control mode to increment.

\item[{get\_num\_channels():}] \leavevmode
\sphinxAtStartPar
Returns the number of output channels for the object.

\item[{get\_default\_outputs():}] \leavevmode
\sphinxAtStartPar
Returns the default output values for the object.

\item[{get\_output(input\_name, input\_value):}] \leavevmode
\sphinxAtStartPar
Calculate and returns the output based on the given input value and current control mode.

\item[{map\_values(value, input\_min, input\_max, out\_min, out\_max):}] \leavevmode
\sphinxAtStartPar
Maps an input value to its output.

\end{description}
\index{get\_output() (SideLipOutput.SideLipOutput method)@\spxentry{get\_output()}\spxextra{SideLipOutput.SideLipOutput method}}

\begin{fulllineitems}
\phantomsection\label{\detokenize{specific:SideLipOutput.SideLipOutput.get_output}}\pysiglinewithargsret{\sphinxbfcode{\sphinxupquote{get\_output}}}{\emph{\DUrole{n}{input\_name}}, \emph{\DUrole{n}{input\_value}}}{}
\sphinxAtStartPar
Calculate and returns the output based on the given input value and current control mode.
\begin{description}
\item[{input\_name}] \leavevmode{[}str{]}
\sphinxAtStartPar
name associated with the input

\item[{input\_value}] \leavevmode{[}int{]}
\sphinxAtStartPar
the input value from the PS4 controller

\end{description}
\begin{description}
\item[{{[}channels\_output, current\_output{]}}] \leavevmode{[}{[}{[}int{]}, {[}int{]}{]}{]}
\sphinxAtStartPar
current\_output is the pulse widths in quarter microseconds to output, and channels\_output
is which channels those outputs will be sent over. How the ouptut is calculated is based
off of which control type the output object is set to. Direct will map the output directly
based on the input and the set input and output ranges. Toggle will set the output between
the max and the min output values and switch between these values whenever the input is
released. Increment will increment the output value whenever input is given.

\end{description}

\end{fulllineitems}


\end{fulllineitems}



\subsection{Neck Tilt Output}
\label{\detokenize{specific:module-NeckTiltOutput}}\label{\detokenize{specific:neck-tilt-output}}\index{module@\spxentry{module}!NeckTiltOutput@\spxentry{NeckTiltOutput}}\index{NeckTiltOutput@\spxentry{NeckTiltOutput}!module@\spxentry{module}}\index{NeckTiltOutput (class in NeckTiltOutput)@\spxentry{NeckTiltOutput}\spxextra{class in NeckTiltOutput}}

\begin{fulllineitems}
\phantomsection\label{\detokenize{specific:NeckTiltOutput.NeckTiltOutput}}\pysiglinewithargsret{\sphinxbfcode{\sphinxupquote{class }}\sphinxcode{\sphinxupquote{NeckTiltOutput.}}\sphinxbfcode{\sphinxupquote{NeckTiltOutput}}}{\emph{\DUrole{n}{name}}, \emph{\DUrole{n}{num\_outputs}}, \emph{\DUrole{n}{channels\_output}}, \emph{\DUrole{n}{names\_input}}}{}
\sphinxAtStartPar
A class to represent the output object for the two axes of neck tilting that will take multiple inputs.
It inherits attributes and methods from MultiInputOutputObject.

\sphinxAtStartPar
…
\begin{description}
\item[{name}] \leavevmode{[}str{]}
\sphinxAtStartPar
name of the servo group for the output

\item[{num\_outputs}] \leavevmode{[}int{]}
\sphinxAtStartPar
total number of outputs calculated from the given input

\item[{channels\_output}] \leavevmode{[}{[}int{]}{]}
\sphinxAtStartPar
channels corresponding to the servos controlled by the outputs

\item[{maximums\_output}] \leavevmode{[}{[}int{]}{]}
\sphinxAtStartPar
maximum pulse width values for the corresponding servo channel

\item[{minimums\_output}] \leavevmode{[}{[}int{]}{]}
\sphinxAtStartPar
minimum pulse width values for the corresponding servo channel

\item[{default\_output}] \leavevmode{[}{[}int{]}{]}
\sphinxAtStartPar
default output position of the servos for start up position

\item[{current\_output}] \leavevmode{[}{[}int{]}{]}
\sphinxAtStartPar
current output position of the servos to be used for increment mode

\item[{maximum\_input}] \leavevmode{[}int{]}
\sphinxAtStartPar
maximum input value used for mapping inputs to outputs

\item[{minimum\_input}] \leavevmode{[}int{]}
\sphinxAtStartPar
minimum input value used for mapping inputs to outputs

\item[{is\_inverted}] \leavevmode{[}Boolean list{]}
\sphinxAtStartPar
whether to invert the output mapping

\item[{control\_type}] \leavevmode{[}Enum ControlType{]}
\sphinxAtStartPar
mode on how to determine the output

\item[{toggle\_state}] \leavevmode{[}Enum ToggleState{]}
\sphinxAtStartPar
used when control mode is set to TOGGLE to determine the current output state

\item[{names\_input}] \leavevmode{[}{[}str{]}{]}
\sphinxAtStartPar
list of the input names used by the object, this allows the object to know which input value to update

\item[{num\_inputs}] \leavevmode{[}int{]}
\sphinxAtStartPar
number of inputs the object uses

\item[{current\_input}] \leavevmode{[}{[}int{]}{]}
\sphinxAtStartPar
since input is given one at a time, this list keeps track of previous inputs

\item[{out\_raw\_min}] \leavevmode{[}int{]}
\sphinxAtStartPar
minimum value of an intermediate value used to calculate output

\item[{out\_raw\_max}] \leavevmode{[}int{]}
\sphinxAtStartPar
maximum value of an intermediate value used to calculate output

\item[{raw\_output}] \leavevmode{[}{[}int{]}{]}
\sphinxAtStartPar
list to store the calculated intermediate values that will be mapped to final output values

\end{description}
\begin{description}
\item[{\_\_init\_(name, num\_outputs, channels\_output, names\_input):}] \leavevmode
\sphinxAtStartPar
Class constructor. Assigns the values passed in and initalizes remaining members to default values.

\item[{set\_outputs(channels\_output, maximums\_output, minimums\_output):}] \leavevmode
\sphinxAtStartPar
Sets which channels to output to and the minimun, default, and maximum pulse width for each of those channels.
Also sets current outputs to the same values as the default.

\item[{set\_inversion(is\_inverted):}] \leavevmode
\sphinxAtStartPar
Sets whether to invert the output signal or not.

\item[{set\_control\_direct():}] \leavevmode
\sphinxAtStartPar
Sets control mode to direct.

\item[{set\_control\_toggle():}] \leavevmode
\sphinxAtStartPar
Sets control mode to toggle.

\item[{set\_control\_increment():}] \leavevmode
\sphinxAtStartPar
Sets control mode to increment.

\item[{get\_num\_channels():}] \leavevmode
\sphinxAtStartPar
Returns the number of output channels for the object.

\item[{get\_default\_outputs():}] \leavevmode
\sphinxAtStartPar
Returns the default output values for the object.

\item[{get\_output(input\_name, input\_value):}] \leavevmode
\sphinxAtStartPar
Calculate and returns the output based on the given input value and current control mode.

\item[{map\_values(value, input\_min, input\_max, out\_min, out\_max):}] \leavevmode
\sphinxAtStartPar
Maps an input value to its output.

\end{description}
\index{get\_output() (NeckTiltOutput.NeckTiltOutput method)@\spxentry{get\_output()}\spxextra{NeckTiltOutput.NeckTiltOutput method}}

\begin{fulllineitems}
\phantomsection\label{\detokenize{specific:NeckTiltOutput.NeckTiltOutput.get_output}}\pysiglinewithargsret{\sphinxbfcode{\sphinxupquote{get\_output}}}{\emph{\DUrole{n}{input\_name}}, \emph{\DUrole{n}{input\_value}}}{}
\sphinxAtStartPar
Calculate and returns the output based on the given input value and current control mode.
\begin{description}
\item[{input\_name}] \leavevmode{[}str{]}
\sphinxAtStartPar
name associated with the input

\item[{input\_value}] \leavevmode{[}int{]}
\sphinxAtStartPar
the input value from the PS4 controller

\end{description}
\begin{description}
\item[{{[}channels\_output, current\_output{]}}] \leavevmode{[}{[}{[}int{]}, {[}int{]}{]}{]}
\sphinxAtStartPar
current\_output is the pulse widths in quarter microseconds to output, and channels\_output
is which channels those outputs will be sent over. How the ouptut is calculated is based
off of which control type the output object is set to. Direct will map the output directly
based on the input and the set input and output ranges. Toggle will set the output between
the max and the min output values and switch between these values whenever the input is
released. Increment will increment the output value whenever input is given.

\end{description}

\end{fulllineitems}


\end{fulllineitems}



\chapter{Indices and tables}
\label{\detokenize{index:indices-and-tables}}\begin{itemize}
\item {} 
\sphinxAtStartPar
\DUrole{xref,std,std-ref}{genindex}

\item {} 
\sphinxAtStartPar
\DUrole{xref,std,std-ref}{modindex}

\item {} 
\sphinxAtStartPar
\DUrole{xref,std,std-ref}{search}

\end{itemize}


\renewcommand{\indexname}{Python Module Index}
\begin{sphinxtheindex}
\let\bigletter\sphinxstyleindexlettergroup
\bigletter{a}
\item\relax\sphinxstyleindexentry{AnalogMixerOutput}\sphinxstyleindexpageref{generic:\detokenize{module-AnalogMixerOutput}}
\item\relax\sphinxstyleindexentry{AnalogOutputObject}\sphinxstyleindexpageref{generic:\detokenize{module-AnalogOutputObject}}
\indexspace
\bigletter{d}
\item\relax\sphinxstyleindexentry{DigitalOutputObject}\sphinxstyleindexpageref{generic:\detokenize{module-DigitalOutputObject}}
\indexspace
\bigletter{e}
\item\relax\sphinxstyleindexentry{EarOutput}\sphinxstyleindexpageref{specific:\detokenize{module-EarOutput}}
\item\relax\sphinxstyleindexentry{EyebrowsOutput}\sphinxstyleindexpageref{specific:\detokenize{module-EyebrowsOutput}}
\indexspace
\bigletter{m}
\item\relax\sphinxstyleindexentry{manualControl}\sphinxstyleindexpageref{input:\detokenize{module-manualControl}}
\item\relax\sphinxstyleindexentry{MovementMap}\sphinxstyleindexpageref{mapping:\detokenize{module-MovementMap}}
\item\relax\sphinxstyleindexentry{MultiInputOutputObject}\sphinxstyleindexpageref{base:\detokenize{module-MultiInputOutputObject}}
\indexspace
\bigletter{n}
\item\relax\sphinxstyleindexentry{NeckTiltOutput}\sphinxstyleindexpageref{specific:\detokenize{module-NeckTiltOutput}}
\indexspace
\bigletter{o}
\item\relax\sphinxstyleindexentry{OutputObject}\sphinxstyleindexpageref{base:\detokenize{module-OutputObject}}
\indexspace
\bigletter{s}
\item\relax\sphinxstyleindexentry{SideLipOutput}\sphinxstyleindexpageref{specific:\detokenize{module-SideLipOutput}}
\end{sphinxtheindex}

\renewcommand{\indexname}{Index}
\printindex
\end{document}